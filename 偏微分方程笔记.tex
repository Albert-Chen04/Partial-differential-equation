\documentclass[12pt,a4paper]{article}
\usepackage[UTF8]{ctex}
\usepackage{geometry}
\usepackage{amsmath,amssymb,amsthm}
\usepackage{mathtools,bm}
\usepackage{empheq}
\usepackage{graphicx}
\usepackage{booktabs}
\usepackage[numbers,sort&compress]{natbib}
\usepackage{caption}
\usepackage{enumitem}
\usepackage{chngcntr}

% ========== 页面布局 ==========
\geometry{left=2.5cm,right=2.5cm,top=2.5cm,bottom=2.5cm}
\setlength{\parskip}{0.5em}
\renewcommand{\baselinestretch}{1.2}

% ========== 数学命令 ==========
\newcommand{\diff}{\mathop{}\!\mathrm{d}}
\newcommand{\R}{\mathbb{R}}
\newcommand{\C}{\mathbb{C}}
\newcommand{\Z}{\mathbb{Z}}
\newcommand{\N}{\mathbb{N}}
\DeclareMathOperator{\supp}{supp}

% ========== 编号系统 ==========
\numberwithin{subsection}{section}   % 子小节按章节编号
\numberwithin{subsubsection}{subsection}
\counterwithin{equation}{subsection} % 公式按子小节编号

% ========== 定理环境 ==========
\theoremstyle{plain}
\newtheorem{theorem}{定理}[section]
\newtheorem{lemma}[theorem]{引理}
\newtheorem{proposition}[theorem]{命题}
\newtheorem{corollary}[theorem]{推论}
\newtheorem{solution}{解}[subsection]  % 解按子小节编号

\theoremstyle{definition}
\newtheorem{definition}[theorem]{定义}
\newtheorem{example}{命题}[subsection]  % 示例按子小节编号

\theoremstyle{remark}
\newtheorem{remark}[theorem]{注记}

\theoremstyle{remark}
\newtheorem{verification}[theorem]{验证}

% 加载hyperref包以实现超链接功能
\usepackage[colorlinks=true, linkcolor=black]{hyperref}

\title{偏微分方程笔记}
\author{陈柏均}
\date{2025年5月12日}

\begin{document}
	
	\maketitle


\newpage
	
	\tableofcontents  % 添加目录
	
	
	\newpage
	
		\section{齐次化}
	\subsection{边界条件齐次化}
	要想利用Duhamamel原理,我们首先将第一边界条件齐次化,即要找到一个恰当的变换将第一边界值变为零。
	对于	(A) 方程问题:
	\begin{equation}
		\begin{cases}
			u_{tt} - a^2 u_{xx} = f(x, t), & 0 < x < l, \ t > 0, \\
			u(x, 0) = \varphi(x), \ u_t(x, 0) = \psi(x), & 0 \leq x \leq l, \\
			u(0, t) = \mu_1(t), \ u(l, t) = \mu_2(t), & t \geq 0.
		\end{cases}
	\end{equation}
	
	由边界条件
	\begin{equation}
		u(0,t) = \mu_1(t), \quad u(l,t) = \mu_2(t), \quad t \geq 0.
	\end{equation}
	
	对方程 (A) 构造关于变量 \(x\) 的线性辅助函数(直线方程):
	\begin{equation}
		U(x, t) = \mu_1(t) + \frac{x}{l}(\mu_2(t) - \mu_1(t)),
	\end{equation}
	
	作变换:
	\begin{equation}
		v(x, t) = u(x, t) - U(x, t),
	\end{equation}
	
	将 \(u(x, t) = v(x, t) + U(x, t)\) 代入方程 (A):
	\begin{equation}
		\begin{cases}
			u_{tt} - a^2 u_{xx} = f(x, t), & 0 < x < l, \ t > 0, \\
			u(x, 0) = \varphi(x), \ u_t(x, 0) = \psi(x), & 0 \leq x \leq l, \\
			u(0, t) = \mu_1(t), \ u(l, t) = \mu_2(t), & t \geq 0.
		\end{cases}
	\end{equation}
	
	得到方程 (B):
	\begin{equation}\label{B}
		\begin{cases}
			v_{tt} - a^2 v_{xx} = f_1(x, t), & 0 < x < l, \ t > 0, \\
			v|_{t=0} = \varphi_1(x), \quad v_t|_{t=0} = \psi_1(x), & 0 \leq x \leq l, \\
			v|_{x=0} = 0, \quad v|_{x=l} = 0, & t \geq 0,
		\end{cases}
	\end{equation}
	其中:
	\begin{equation}
		\begin{cases}
			f_1(x, t) = f(x, t) - \mu_1''(t) - \frac{x}{l}(\mu_2''(t) - \mu_1''(t)), \\
			\varphi_1(x) = \varphi(x) - \mu_1(0) - \frac{x}{l}(\mu_2(0) - \mu_1(0)), \\
			\psi_1(x) = \psi(x) - \mu_1'(0) - \frac{x}{l}(\mu_2'(0) - \mu_1'(0)).
		\end{cases}
	\end{equation}
	
	这样我们就完成了边界条件的齐次化。
	
	\subsection{叠加原理}
	对于问题\eqref{B},我们可以使用叠加原理将其分解为两个子问题.
	\begin{equation}
		\begin{cases}
			v_{tt} - a^2 v_{xx} = f_1(x, t), & 0 < x < l, \ t > 0, \\
			v|_{t=0} = 0, \quad v_t|_{t=0} = 0, & 0 \leq x \leq l, \\
			v|_{x=0} = 0, \quad v|_{x=l} = 0, & t \geq 0.
		\end{cases}
	\end{equation}
	
	\begin{equation}\label{f}
		\begin{cases}
			v_{tt} - a^2 v_{xx} = 0, & 0 < x < l, \ t > 0, \\
			v|_{t=0} = \varphi_1(x), \quad v_t|_{t=0} = \psi_1(x), & 0 \leq x \leq l, \\
			v|_{x=0} = 0, \quad v|_{x=l} = 0, & t \geq 0.
		\end{cases}
	\end{equation}
	
	
	根据叠加原理,原问题\eqref{B}  的解 \(v(x, t)\) 可以表示为子问题1和子问题2的解的和:
	\begin{equation}
		v(x, t) = v^{(1)}(x, t) + v^{(2)}(x, t)
	\end{equation}
	其中 \(v^{(1)}(x, t)\) 是子问题1的解,\(v^{(2)}(x, t)\) 是子问题2的解。
	
	
	
	
	\subsection{Duhamamel原理之方程齐次化}
	对于有界方程非齐次问题:
	\begin{equation}
		\begin{cases}
			v_{tt} - a^2 v_{xx} = f_1(x, t), & 0 < x < l, \ t > 0, \\
			v|_{t=0} = 0, \quad v_t|_{t=0} = 0, & 0 \leq x \leq l, \\
			v|_{x=0} = 0, \quad v|_{x=l} = 0, & t \geq 0,
		\end{cases}
	\end{equation}
	
	若 \( w(x, t, \tau) \) 是以下定解问题的解:
	\begin{equation}
		\begin{cases}
			W_{tt} - a^2 W_{xx} = 0, & t > \tau, \\
			W|_{t=\tau} = 0, \quad \left. \frac{\partial W}{\partial t} \right|_{t=\tau} = f_1(x, \tau), & 0 \leq x \leq l,
		\end{cases}
	\end{equation}
	
	则函数 \( v(x, t) = \int_0^t w(x, t, \tau) \, d\tau \) 就是问题的解。
	然后还有偏导数,那方程的函数转移到最高偏导函数上,其余依旧为0.
	
	
	
	
	
	所以我们最终要解决的就是初始条件非齐次,方程和边界条件齐次\eqref{B}这样的方程.
	
	对于无界区域中的非齐次波动方程:
	
\begin{equation}
	\begin{cases}
		v_{tt} - a^2 v_{xx} = f_1(x, t), & -\infty < x < \infty, \ t > 0 \\
		v|_{t=0} = 0, \quad v_t|_{t=0} = 0, & -\infty < x < \infty
	\end{cases}
\end{equation}
	
	假设 \(w(x, t, \tau)\) 是以下齐次波动方程的解:
	
\begin{equation}
	\begin{cases}
		w_{tt} - a^2 w_{xx} = 0, & t > \tau \\
		w(x, \tau, \tau) = 0, & -\infty < x < \infty \\
		\frac{\partial w}{\partial t}(x, \tau, \tau) = f_1(x, \tau)
	\end{cases}
\end{equation}
	
	那么,原非齐次波动方程的解可以表示为:
	
\begin{equation}
	v(x, t) = \int_0^t w(x, t, \tau) \, d\tau
\end{equation}
	
	有界无界两者齐次化方程形式上是一样的,后面热传导方程也是一样的。
	
	
	
	
	
	
	\section{一阶拟线性方程之传输方程} 
	\subsection{变量替换求解常系数齐次传输方程} 
	\subsubsection{问题描述}
	
	
	假设 $a_1 \neq 0$ 且 $a_2 \neq 0$,我们求解常系数传输方程:
	\begin{equation} \label{eq:pde_original}
		a_1 \frac{\partial u}{\partial t} + a_2 \frac{\partial u}{\partial x} = 0
	\end{equation}
	
	\subsubsection{通解} 
	核心思想:通过变量替换,把二元偏微分转化成一元的常微分求解。
	
	其中 $u = u(t,x)$。引入坐标变换 $(\alpha, \beta)$,使得 $u = u(\alpha, \beta)$,且:
	\begin{equation} \label{eq:coordinate_transform}
		\begin{cases}
			\alpha = ax + bt, \\
			\beta = cx + dt.
		\end{cases}
	\end{equation}
	利用链式法则计算偏导数:
	\begin{align}
		\frac{\partial u}{\partial t} 
		&= \frac{\partial u}{\partial \alpha} \frac{\partial \alpha}{\partial t} + \frac{\partial u}{\partial \beta} \frac{\partial \beta}{\partial t} 
		= b\frac{\partial u}{\partial \alpha} + d\frac{\partial u}{\partial \beta}, \label{eq:u_t_chain_rule} \\
		\frac{\partial u}{\partial x} 
		&= \frac{\partial u}{\partial \alpha} \frac{\partial \alpha}{\partial x} + \frac{\partial u}{\partial \beta} \frac{\partial \beta}{\partial x} 
		= a\frac{\partial u}{\partial \alpha} + c\frac{\partial u}{\partial \beta}. \label{eq:u_x_chain_rule}
	\end{align}
	
	将 \eqref{eq:u_t_chain_rule} 和 \eqref{eq:u_x_chain_rule} 代入原方程 \eqref{eq:pde_original}:
	\begin{equation} \label{eq:pde_transformed}
		a_1 \left( b \frac{\partial u}{\partial \alpha} + d \frac{\partial u}{\partial \beta} \right) + a_2 \left( a \frac{\partial u}{\partial \alpha} + c \frac{\partial u}{\partial \beta} \right) = 0.
	\end{equation}
	整理后得到:
	\begin{equation} \label{eq:pde_collected}
		(a_1 b + a_2 a) \frac{\partial u}{\partial \alpha} + (a_1 d + a_2 c) \frac{\partial u}{\partial \beta} = 0.
	\end{equation}
	为消去一个变量,pde转ode,选择让第二项系数为0,把方程 \eqref{eq:pde_collected} 简化为:
	\begin{equation} \label{eq:pde_final}
		\frac{\partial u}{\partial \alpha} = 0.
	\end{equation}
	选择系数
	\begin{equation} \label{eq:constant_choice}
		\begin{cases}
			a = 0, & b = 1, \\
			c = a_1, & d = -a_2.
		\end{cases}
	\end{equation}
	此时坐标变换为:
	\begin{equation} \label{eq:coordinates_specific}
		\begin{cases}
			\alpha = t, \\
			\beta = a_1 x - a_2 t.
		\end{cases}
	\end{equation}
	由\eqref{eq:pde_final} 表明 $u$ 仅依赖于 $\beta$,即通解为:
	\begin{equation} \label{eq:solution}
		u(t,x) = L(a_1 x - a_2 t),
	\end{equation}
	其中
	$L(\cdot)$ 是任意可微函数。
	
		\subsubsection{特解(初始条件或边界条件)} 
	已知初始条件$	u(x, 0) = e^{-x^2}$,求下面常系数运输方程:
	\begin{equation} 
		\frac{\partial u}{\partial t} +  \frac{\partial u}{\partial x} = 0
	\end{equation}
	
	由\eqref{eq:solution}可知
	\begin{equation}
		u(x, t) = f(x - t)= e^{-(x-t)^2}
	\end{equation}
	
	
	\subsection{波的传播求解常系数齐次传输方程} 
	\subsubsection{问题描述}
	在一阶线性方程中,有一种最简单的形如
	
	\begin{equation}
		u_t + b \cdot \mathrm{D}u = 0, \quad x \in \mathbb{R}^n, \ t \in (0, \infty)
	\end{equation}
	
	的方程,称为传输方程,其中,\(b = (b_1, b_2, \cdots, b_n)\) 是已知 \(n\) 维常向量,\(u = u(x, t)\),\(\mathrm{D}u = (u_{x_1}, u_{x_2}, \cdots, u_{x_n})\)。
	
		\subsubsection{通解} 
			\begin{equation}
			\frac{\partial u}{\partial t} + b \frac{\partial u}{\partial x}=(1, b) \cdot \left( \frac{\partial u}{\partial t}, \frac{\partial u}{\partial x} \right) = 0 
		\end{equation}
		$\left( \frac{\partial u}{\partial x}, \frac{\partial u}{\partial y} \right)$为梯度,$(1, b)$为方向,一整个乘积为方向导数,方向导数为0意味着,$u(t, x)=C$在切向量为$(1, b)$这条曲线上,即
		\begin{equation}
			u(t,x)|_{\Gamma} = C
		\end{equation}
		
	
	由方程的形式可以看出,\(u(x, t)\) 沿$(1, b)$微商等于零。事实上,固定一点 \((x, t) \in \mathbb{R}^{n+1}\),记过该直线$\Gamma$的参数方程为 \((x + bs, t + s), s \in \mathbb{R}\),考查函数 \(u\) 在该直线上的值。令
\begin{equation}
	z(s) = u(x + bs, t + s), \quad s \in \mathbb{R}.
	\end{equation}
	
	于是
	\begin{equation}
	\frac{\mathrm{d}z}{\mathrm{d}s} = \mathrm{D}u(x + sb, t + s) \cdot b + u_t(x + sb, t + s) = 0,
	\end{equation}
	
	因此,函数 \(z(s)\) 在过点 \((x, t)\) 且具有方向 \((b, 1) \in \mathbb{R}^{n+1}\) 的直线上取常数值,特征线上的取值和$s$没有关系(和下文中特征线法求解传输方程的$(1, p(x, y))$含义相同)。所以,如果我们知道解 \(u\) 在这条直线上一点的值,则就得到它沿此直线上的值。这就引出下面求解初值问题的方法。
	
	\subsubsection{初值问题之特解} 
	设 $a \in \mathbb{R}^n$ 是已知常向量,$f: \mathbb{R}^n \rightarrow \mathbb{R}$ 是给定函数。考察传输方程的初值问题
	\begin{equation}
		\begin{cases}
			u_t + a \cdot \mathrm{D}u = 0, & (x, t) \in \mathbb{R}^n \times (0, \infty), \\
			u(x, 0) = f(x), & x \in \mathbb{R}^n.
		\end{cases}
	\end{equation}
	
	如上取定 $(x, t)$,过点 $(x, t)$ 且具有方向 $(a, 1)$ 的直线的参数式为 $(x + a s, t + s)$,$s \in \mathbb{R}$。当 $s = -t$ 时,此直线与平面 $\Gamma: \mathbb{R}^n \times \{t = 0\}$ 相交于点 $(x - a t, 0)$。由上文分析知 $u$ 沿此直线取常数值,而由初值条件便得
	\begin{equation}\label{eq:齐次解}
	u(x,t)=	z(0)=z(-t) =u(x - a t, 0) = f(x - a t), \quad x \in \mathbb{R}^n, \ t \geq 0.
	\end{equation}
	
	\begin{remark}
	这表示对于每一个特定的点都有一条特征线,他的函数为特定的$f$。取遍每个特征线就能取遍域内所有点,对于任意的点都有任意的函数表达式。因为上面的式子,at是任意的,所以x-at是任意的,可以取遍整个
	\end{remark}
		
	所以,如果 有解,必由上式子 表示,因此解是唯一的;反之,若 $f$ 一阶连续可微,则可直接验证由 上式子表示的函数 $u(x, t)$ 是问题的解。这就是齐次传输方程初值问题解的存在唯一性。
	
\subsection{波的传播求解常系数非齐次传输方程} 
\subsubsection{问题描述}
	考察非齐次传输方程的初值问题
	\begin{equation}
		\begin{cases}
			u_t + a \cdot \mathrm{D}u = f, & x \in \mathbb{R}^n, t > 0, \\
			u(x,0) = g(x)
		\end{cases}
	\end{equation}
	
\subsubsection{求解}
	受齐次问题解法的启示,我们仍然先取定 \((x, t) \in \mathbb{R}^{n+1}\),对 \(s \in \mathbb{R}\),令 \(z(s) = u(x + a s, t + s)\),则
		\begin{equation}
	\frac{\mathrm{d}z}{\mathrm{d}s} = \mathrm{D}u(x + a s, t + s) \cdot a + u_t(x + a s, t + s) = f(x + a s, t + s).
		\end{equation}
	
	因此,
	\begin{equation}
	\begin{aligned}
		u(x, t) -	u(x-at,0)&= u(x, t)-g(x - a t) \\
		&= z(0) - z(-t) = \int_{-t}^0 \frac{\mathrm{d}z}{\mathrm{d}s} \, \mathrm{d}s \\
		&= \int_{-t}^0 f(x + a s, t + s) \, \mathrm{d}s \\
		&= \int_0^t f(x + a (s - t), s) \, \mathrm{d}s.
	\end{aligned}
\end{equation}
	
	于是,得到问题 的在 \(x \in \mathbb{R}^n\),\(t \geq 0\) 上的解
	\begin{equation}\label{eq:非齐次解1}
		u(x, t) = g(x - a t) + \int_0^t f(x + a (s - t), s) \, \mathrm{d}s.
	\end{equation}
	
	在下一章,这个公式将被用来求解一维波动方程。
	
	
	
	\subsection{特征线法求解变系数齐次传输方程} 
	\subsubsection{通解} 
	一阶线性变系数偏微分方程如下:
	\begin{equation}\label{eq:pde_original2}
		\frac{\partial u}{\partial x} + p(x,y) \frac{\partial u}{\partial y}=(1, p(x, y)) \cdot \left( \frac{\partial u}{\partial x}, \frac{\partial u}{\partial y} \right) = 0 
	\end{equation}
	其中 $p(x, y)$ 是 $x$ 和 $y$ 的函数。
	$\left( \frac{\partial u}{\partial x}, \frac{\partial u}{\partial y} \right)$为梯度,$(1, p(x, y))$为方向,一整个乘积为方向导数,方向导数为0意味着,$u(x, y)=C$在切向量为$(1, p(x, y))$这条曲线上,即
	\begin{equation}
		u(x,y)|_{\Gamma} = C
	\end{equation}
	\begin{equation}
		u(x,y) =  f(C)
	\end{equation}
	$\Gamma$曲线上,任意点$(x, y)$求导($\Gamma$曲线为$XOY$平面上的曲线,故$y$可表示成$x$的函数),可得切向量$(1,\frac{dy}{dx})$
	
	所以我们找到$\Gamma$曲线,把二元偏微分转化成一元的常微分,令
	
	\begin{equation}
		\frac{dy}{dx} = p(x, y)
	\end{equation}
	可解得
	\begin{equation}
		C=\phi(x,y)
	\end{equation}
	得方程解
	\begin{equation}
		u(x, y)=f(C)=f(\phi(x,y))
	\end{equation}
	$(1, \frac{dy}{dx})$
	为该曲线的切向量。我们称这条曲线叫特征线。只需要取遍所有的特征曲线就可以取遍$XOY$平面上所有的点,若有初始条件或者边界条件可以确定每条特征线在$u(x, y)$对应的取值,就可以完整确定$u(x, y)$这个函数。
	
	\begin{example}求解方程
		\begin{equation}
			\frac{\partial u}{\partial x} + x \frac{\partial u}{\partial y} = 0.
		\end{equation}
		
		此时我们有 $p(x, y) = x$,解 $\frac{dy}{dx} = x$,我们得到特征线 $y = \frac{1}{2}x^2 + C$,或 $y - \frac{1}{2}x^2 = C$。从而 $\phi(x, y) = y - \frac{1}{2}x^2$,偏微分方程的通解为 $u(x, y) = f(\phi(x, y))$,其中 $f$ 是任意函数。把它们代回方程,直接验证,便知是解。
	\end{example}
	
	\newpage
	
	\section{一维无界齐次波动方程}
	
	\subsection{d’Alembert 公式}
	
	\subsubsection{问题描述}
	先考察初值问题
	
	\begin{equation}
		\begin{cases}
			u_{tt} - a^2 u_{xx} = 0, & x \in \mathbb{R}, t > 0, \\
			u(x, 0) = \varphi(x), \quad u_t(x, 0) = \psi(x), & x \in \mathbb{R}.
		\end{cases}
	\end{equation}
	
\subsubsection{求解}
	由算子复合作用的概念,易验证下述算子因式分解
	
	\begin{equation}
		\left( \frac{\partial}{\partial t} + a \frac{\partial}{\partial x} \right) \left( \frac{\partial}{\partial t} - a \frac{\partial}{\partial x} \right) u = u_{tt} - a^2 u_{xx} = 0.
	\end{equation}
	
	令
	
	\begin{equation}
		v(x, t) = \left( \frac{\partial}{\partial t} - a \frac{\partial}{\partial x} \right) u.
	\end{equation}
	
	由 (2.1.2), 得
	\begin{equation}
	v_t(x, t) + a v_x(x, t) = 0, \quad x \in \mathbb{R}, t > 0.
\end{equation}
	
	这是一维传输方程,且由 (2.1.3) 知 \(v\) 满足初值条件
	
	\begin{equation}
	v(x, 0) = \psi(x) - a \varphi'(x).
\end{equation}
	
	由 \eqref{eq:齐次解}, 得
	\begin{equation}
		v(x, t) = \psi(x - a t) - a \varphi'(x - a t).
	\end{equation}
	
	将 \(v\) 代入 (2.1.3),得
	\begin{equation}
		u_t(x, t) - a u_x(x, t) = \psi(x - a t) - a \varphi'(x - a t),
	\end{equation}
	其中 \((x, t) \in \mathbb{R} \times (0, \infty)\)。
	
	对此非齐次传输方程,已知 \(u(x, 0) = \varphi(x)\),用公式\eqref{eq:非齐次解1}得到
	\begin{equation}\label{eq:达朗贝尔公式}
		\begin{aligned}
			u(x, t) &= \varphi(x + a t) + \int_0^t \left[ \psi(x - 2 a s + a t) - a \varphi'(x - 2 a s + a t) \right] \mathrm{d}s \\
			&= \varphi(x + a t) + \frac{1}{2 a} \int_{x - a t}^{x + a t} \left[ \psi(y) - a \varphi'(y) \right] \mathrm{d}y \\
			&= \frac{1}{2} \left[ \varphi(x + a t) + \varphi(x - a t) \right] + \frac{1}{2 a} \int_{x - a t}^{x + a t} \psi(y) \mathrm{d}y.
		\end{aligned}
	\end{equation}
	称此式为 d'Alembert (达朗贝尔) 公式.
	
	\section{一维齐次波动方程的初边值问题}
	\subsection{第一边值条件半直线问题}
	反射法的核心思想:利用达朗贝尔公式把解延拓
	
	\subsubsection{问题描述}
	求解半直线 \(\mathbb{R}_+ = \{x > 0\}\) 上的初边值问题:
	
	\begin{equation}
		\begin{cases}
			u_{tt} - u_{xx} = 0, & x \in \mathbb{R}_+, t > 0, \\
			u(x, 0) = g(x), \quad u_t(x, 0) = h(x), & x \in \mathbb{R}_+, \\
			u(0, t) = 0, & t \geq 0,
		\end{cases}
	\end{equation}
	
	其中,\(g, h\) 是已知函数,满足 \(g(0) = h(0) = 0\)。
	
	\subsubsection{做奇延拓}
先把问题转换到全空间 \(\mathbb{R}\) 上去。为此,对函数 \(u, g, h\) 作奇延拓(或称奇反射)如下:
	
	\begin{equation}
		\bar{u}(x, t) = \begin{cases}
			u(x, t), & x \geq 0, t \geq 0, \\
			-u(-x, t), & x \leq 0, t \geq 0,
		\end{cases}
	\end{equation}
	
	\begin{equation}
		\bar{g}(x) = \begin{cases}
			g(x), & x \geq 0, \\
			-g(-x), & x \leq 0,
		\end{cases}
	\end{equation}
	
	\begin{equation}
		\bar{h}(x) = \begin{cases}
			h(x), & x \geq 0, \\
			-h(-x), & x \leq 0.
		\end{cases}
	\end{equation}
	
\subsubsection{边界条件与方程验证}
设波动方程参数为$a$,考虑有限区间$x \in [0, L]$的延拓问题。已知$f,g$为以$2L$为周期的奇函数,即满足:
\begin{equation}
	\forall y \in \mathbb{R},\quad 
	\begin{cases}
		f(y + 2L) = f(y) \\
		f(-y) = -f(y) \\
		g(y + 2L) = g(y) \\
		g(-y) = -g(y)
	\end{cases}
\end{equation}

\paragraph{达朗贝尔解表达式}
延拓后的解可表示为:
\begin{equation}
	u(x,t) = \frac{1}{2}[f(x + at) + f(x - at)] + \frac{1}{2a}\int_{x-at}^{x+at} g(y) dy
\end{equation}

\paragraph{边界点验证}
\begin{itemize}
	\item \textbf{左端点$x=0$:}
	\begin{align*}
		u(0,t) &= \frac{1}{2}[f(at) + f(-at)] + \frac{1}{2a}\int_{-at}^{at} g(y) dy \\
		&= \frac{1}{2}[f(at) - f(at)] + 0 \quad (\text{奇函数性质}) \\
		&= 0
	\end{align*}
	
	\item \textbf{右端点$x=L$:} 利用周期性与奇性
	\begin{align*}
		u(L,t) &= \frac{1}{2}[f(L+at) + f(L-at)] + \frac{1}{2a}\int_{L-at}^{L+at} g(y) dy \\
		&= \frac{1}{2}[f(L+at) + f(-(at-L))] + \frac{1}{2a}\int_{-at}^{at} g(y+L) dy \quad (y \mapsto y-L) \\
		&= \frac{1}{2}[f(L+at) - f(at-L)] + \frac{1}{2a}\int_{-at}^{at} -g(-y+L) dy \quad (\text{周期奇性}) \\
		&= \frac{1}{2}[f(L+at) - f(L+at-2L)] + 0 \quad (\text{积分对称性}) \\
		&= 0 \quad (\because f\text{的}2L\text{周期性})
	\end{align*}
\end{itemize}

\paragraph{方程验证}
\begin{itemize}
	\item \textbf{正半轴$x \geq 0$:} 直接满足原波动方程
	
	\item \textbf{负半轴$x < 0$:} 令$x = -y,\ y > 0$,则延拓解为
	\[
	\bar{u}(x,t) = -u(y,t) = -u(-x,t)
	\]
	计算二阶导数:
	\begin{align}
		\bar{u}_{xx}(x,t) &= \frac{\partial^2}{\partial x^2}[-u(-x,t)] = -u_{xx}(-x,t) \\
		\bar{u}_{tt}(x,t) &= \frac{\partial^2}{\partial t^2}[-u(-x,t)] = -u_{tt}(-x,t)
	\end{align}
	验证波动方程:
	\[
	\bar{u}_{tt} - a^2\bar{u}_{xx} = -u_{tt}(-x,t) + a^2u_{xx}(-x,t) = 0 
	\]
\end{itemize}






	则 \(\bar{u}(x, t)\) 满足问题:
	
	\begin{equation}
		\begin{cases}
			\bar{u}_{tt} - \bar{u}_{xx} = 0, & (x, t) \in \mathbb{R} \times (0, \infty), \\[8pt]
			\bar{u}(x, 0) = \bar{g}(x), \quad \bar{u}_t(x, 0) = \bar{h}(x), & x \in \mathbb{R}.
		\end{cases}
	\end{equation}




\paragraph{区域分析}
	\begin{equation}
		u(x,t) = 
		\begin{cases}
			\displaystyle
			\frac{1}{2}\left[g(x+at) + g(x-at)\right] + \frac{1}{2a}\int_{x-at}^{x+at} h(s) ds, & x > at \geq 0 \\
			\displaystyle
			\frac{1}{2}\left[g(x+at) - g(at-x)\right] + \frac{1}{2a}\int_{at-x}^{x+at} h(s) ds, & 0 \leq x < at
		\end{cases}
	\end{equation}
	
	
	\begin{remark}
		还可以用特征线法对问题 (3.1.1) 求解,即用初值问题中方程的特征线作自变量的变换,把方程化为双曲型的第二标准型 \(u_{\xi\eta} = 0\) 的形式,对它积分两次求出通解 \(u = F(\xi) + G(\eta)\),其中,\(F\) 和 \(G\) 是任意二次光滑函数。然后利用初值条件确定通解中的两个任意函数,便得 d'Alembert 公式。
	\end{remark}
	
	
	
		\subsection{第二边值条件半直线问题}
		反射法的核心思想:利用达朗贝尔公式把解延拓
		
		\subsubsection{问题描述}
		求解半直线 \(\mathbb{R}_+ = \{x > 0\}\) 上的初边值问题:
		
		\begin{equation}
			\begin{cases}
				u_{tt} - u_{xx} = 0, & x \in \mathbb{R}_+, t > 0, \\
				u(x, 0) = g(x), \quad u_t(x, 0) = h(x), & x \in \mathbb{R}_+, \\
				u_x(0, t) = 0, & t \geq 0,
			\end{cases}
		\end{equation}
		
		其中,\(g, h\) 是已知函数,满足 \(g'(0) = h'(0) = 0\)(自然相容性条件)。
		
		\subsubsection{做偶延拓}
		先把问题转换到全空间 \(\mathbb{R}\) 上去。为此,对函数 \(u, g, h\) 作偶延拓(或称偶反射)如下:
		
		\begin{equation}
			\bar{u}(x, t) = \begin{cases}
				u(x, t), & x \geq 0, t \geq 0, \\
				u(-x, t), & x \leq 0, t \geq 0,
			\end{cases}
		\end{equation}
		
		\begin{equation}
			\bar{g}(x) = \begin{cases}
				g(x), & x \geq 0, \\
				g(-x), & x \leq 0,
			\end{cases}
		\end{equation}
		
		\begin{equation}
			\bar{h}(x) = \begin{cases}
				h(x), & x \geq 0, \\
				h(-x), & x \leq 0.
			\end{cases}
		\end{equation}
		
		(验证过程省略)则 \(\bar{u}(x, t)\) 满足问题:
		
		\begin{equation}
			\begin{cases}
				\bar{u}_{tt} - \bar{u}_{xx} = 0, & (x, t) \in \mathbb{R} \times (0, \infty), \\
				\bar{u}(x, 0) = \bar{g}(x), \quad \bar{u}_t(x, 0) = \bar{h}(x), & x \in \mathbb{R}.
			\end{cases}
		\end{equation}
		
		\begin{remark}
			对于第二边值条件问题,需保证延拓后的函数 \(\bar{g}(x)\) 和 \(\bar{h}(x)\) 在 \(x=0\) 处满足导数连续的条件。通过偶延拓可自然满足 \(u_x(0,t) = 0\) 的边界条件。
		\end{remark}
	
	
\subsection{有界之反射法}

\subsubsection{有界之第一边值条件}
考虑初边值问题:
\begin{equation}
	\begin{cases}
		u_{tt} - a^2 u_{xx} = 0, & 0 < x < L, \ t > 0, \\
		u(x, 0) = f(x), \ u_t(x, 0) = g(x), & 0 \leq x \leq L, \\
		u(0, t) = \alpha(t), \ u(L, t) = \beta(t), & t \geq 0.
	\end{cases}
\end{equation}

其中,弦的两端固定,即 \(\alpha(t) \equiv \beta(t) \equiv 0\)。

\subsubsection{核心思想}
当两端固定时,若 \(f, g \in C^2\) 满足相容条件:
\begin{equation}
	\begin{aligned}
		f(0) &= f'(0) = f''(0) = 0, & f(L) &= f'(L) = f''(L) = 0, \\
		g(0) &= g'(0) = g''(0) = 0, & g(L) &= g'(L) = g''(L) = 0,
	\end{aligned}
\end{equation}
则可将 \(f, g\) 延拓为实轴上以 \(2L\) 为周期的奇函数:
\begin{equation}
	\begin{aligned}
		f(x) &= -f(-x), & f(x + 2L) &= f(x), \\
		g(x) &= -g(-x), & g(x + 2L) &= g(x).
	\end{aligned}
\end{equation}

延拓后,\(f, g \in C^2(\mathbb{R})\),代入达朗贝尔公式得到延拓问题的解,其在区间 \([0, L]\) 上的限制即为原问题的解。

\subsubsection{达朗贝尔公式的应用}
因为$f$,$g$是以$2L$为周期函数,而且是奇函数。
故
\begin{equation}
g(y + L) = g(y - L) = -g(-y + L)
\end{equation}

$f(y + L)$、$g(y + L)$ 是奇函数。



达朗贝尔公式为:
\begin{equation}
	u(x,t) = \frac{1}{2} \left[ f(x + at) + f(x - at) \right] + \frac{1}{2a} \int_{x - at}^{x + at} g(y) \, dy
\end{equation}

由于 \(f, g\) 为 \(\mathbb{R}\) 上以 \(2L\) 为周期的奇函数,代入边界点 \(x = 0\) 和 \(x = L\) 验证:

对于 \(x = 0\):
\begin{equation}
	u(0, t) = \frac{1}{2} \left[ f(at) + f(-at) \right] + \frac{1}{2a} \int_{-at}^{at} g(y) \, dy = 0
\end{equation}

对于 \(x = L\):
\begin{equation}
	\begin{aligned}
		u(L, t) &= \frac{1}{2}[f(L + at) + f(L - at)] + \frac{1}{2a} \int_{L - at}^{L + at} g(y) \, dy \\
		&= \frac{1}{2}[f(L + at) + f(L - at)] + \frac{1}{2a} \int_{-at}^{at} g(y + L) \, dy \\
		&= 0
	\end{aligned}
\end{equation}

	当 \(x \geq 0\) 时,一定满足波动方程。

当 \(x \leq 0\) 时,令 \(x = -y\),\(y > 0\),
\[
\bar{u}(x, t) = \bar{u}(-y, t) = -u(y, t),
\]

对于 \(\bar{u}_{xx}(x, t)\):
\[
\begin{aligned}
	\bar{u}_{xx}(x, t) &= \bar{u}_{xx}(-y, t) = \frac{d^2}{dx^2} [-u(y, t)] = \frac{d^2}{dx^2} [-u(-x, t)] \\
	&= -u_{xx}(-x, t) = -u_{xx}(y, t).
\end{aligned}
\]

对于 \(\bar{u}_{tt}(x, t)\):
\[
\begin{aligned}
	\bar{u}_{tt}(x, t) &= \bar{u}_{tt}(-y, t) = \bar{u}_{tt}(y, t) = -u_{tt}(y, t).
\end{aligned}
\]

验证波动方程:
\[
\bar{u}_{tt} - \bar{u}_{xx} = -u_{tt}(y, t) + u_{xx}(y, t) = 0
\]

故问题延拓到全平面上就可以用达朗贝尔公式,
\begin{equation*}
	\begin{cases}
		\bar{u}_{tt} - a^2 \bar{u}_{xx} = 0, & x \geq  0 , \ t > 0, \\
		\bar{u}(x, 0) = \bar{f}(x), \ \bar{u}_t(x, 0) = \bar{g}(x), & x \geq  0 \\
		\bar{u}(0, t) = 0, \ u(L, t) = 0
	\end{cases}
\end{equation*}

\subsubsection{有界之第二边值条件}
	\begin{equation}
		\begin{cases}
			u_{tt} - a^2 u_{xx} = 0, & 0 < x < L, \ t > 0, \\
			u(x, 0) = f(x), \ u_t(x, 0) = g(x), & 0 \leq x \leq L, \\
			u_x(0, t) = 0, \ u_x(L, t) = 0, & t \geq 0.
		\end{cases}
	\end{equation}
	对于第二边值条件,我们先做偶延拓,再做周期延拓。
	
	
	
	\subsection{有界之分离变量法}
	\subsubsection{第一边值条件}
	\begin{equation} \label{eq:wave_equation}
		\frac{\partial^2 u}{\partial t^2} = c^2 \frac{\partial^2 u}{\partial x^2} \qquad 0 < x < l, \quad t > 0
	\end{equation}
	
	边界条件:
	\begin{equation} \label{eq:boundary_conditions}
		u(0, t) = 0 \quad u(l, t) = 0 \qquad \forall t > 0
	\end{equation}
	
	初始条件:
	\begin{equation} \label{eq:initial_conditions}
		\begin{aligned}
			u(x, 0) &= f(x) \\
			\frac{\partial u}{\partial t}(x, 0) &= g(x) \qquad 0 < x < l
		\end{aligned}
	\end{equation}
	
	\subsubsection{核心思想}
	核心思想:分离变量法把偏微分转成为两个常微分。
	
	设 \(u(x, t) = X(x) \cdot T(t)\),假设解为乘积解。
	
	代入方程:
	\begin{equation} \label{eq:substitution}
		\frac{\partial^2 u}{\partial t^2} = X \cdot T'' \qquad \frac{\partial^2 u}{\partial x^2} = X'' \cdot T
	\end{equation}
	
	代入原方程:
	\begin{equation} \label{eq:original_substitution}
		X \cdot T'' = c^2 \cdot X'' \cdot T
	\end{equation}
	
	转化为可分离变量方程:
	\begin{equation} \label{eq:separation}
		\frac{T''}{c^2 T} = \frac{X''}{X}
	\end{equation}
	
	两个线性无关的变量相等,只能同为常数:
	\begin{equation} \label{eq:constant}
		\frac{T''}{c^2 T} = \frac{X''}{X} = k
	\end{equation}
	
	转化为两个常微分方程:
	\begin{equation} \label{eq:ode}
		\begin{cases}
			X'' = kX \\
			T'' = k c^2 T
		\end{cases}
	\end{equation}
	
	\subsubsection{空间常微分方程的求解}
	\begin{equation}
		X'' - kX = 0 \quad X(0) = 0 \quad (X(l) = 0
	\end{equation}
	
情况 1 \quad 若 \(k > 0\)

通解为 \(X(x) = C_1 \cdot \cosh \mu x + C_2 \cdot \sinh \mu x\),其中 \(k = \mu^2\)
	
	代入初始条件 
	\begin{equation}
		X(0) = C_1 = 0 \quad X(l) = C_2 \cdot \sinh \mu l = 0 \quad \therefore C_2 = 0
	\end{equation}
	
	
	\begin{verification}	
		\begin{equation*}
			\cosh x = \frac{e^x + e^{-x}}{2} \quad \text{双曲余弦} \quad \sinh x = \frac{e^x - e^{-x}}{2} \quad \text{双曲正弦}
		\end{equation*}
		
		\begin{equation*}
			e^{ix} = \cos x + i \sin x \quad e^{-ix} = \cos x - i \sin x
		\end{equation*}
		
		\begin{equation*}
			\therefore \cos x = \frac{e^{ix} + e^{-ix}}{2} \quad \sin x = \frac{e^{ix} - e^{-ix}}{2i}
		\end{equation*}
		
		\begin{equation*}
			(\cosh x)' = \left( \frac{e^x + e^{-x}}{2} \right)' = \frac{e^x - e^{-x}}{2} = \sinh x
		\end{equation*}
		
		\begin{equation*}
			(\sinh x)' = \left( \frac{e^x - e^{-x}}{2} \right)' = \frac{e^x + e^{-x}}{2} = \cosh x
		\end{equation*}
		
		\begin{equation*}
			\therefore X = C_1 \cdot u \cdot \sinh \mu x + C_2 \cdot u \cdot \cosh \mu x
		\end{equation*}
		
		\begin{equation*}
			X'' = C_1 \cdot \mu^2 \cdot \cosh \mu x + C_2 \cdot \mu^2 \cdot \sinh \mu x
		\end{equation*}
		
		\begin{equation*}
			X'' - kX = 0 \quad \therefore k = \mu^2
		\end{equation*}
		
	\end{verification}	
	
	
情况 2 \quad 若 \(k = 0\)

则 \(X'' = 0\)
	\begin{equation}
		X(x) = C_1 x + C_2 \quad \text{且} \quad X(0) = 0 \quad X(l) = 0
	\end{equation}
	\begin{equation}
		\therefore C_1 = C_2 = 0
	\end{equation}
	
情况 3 \quad 若 \(k < 0\)

即 \(X'' + \mu^2 X = 0\) \quad \(X(0) = 0\) \quad \(X(l) = 0\) \quad \(k = -\mu^2\)
	
	通解:
	\begin{equation}
		X = C_1 \cos \mu x + C_2 \sin \mu x
	\end{equation}
	
	
	
	边界条件:
	\begin{equation}
		X(0) = C_1 = 0 \quad X(l) = C_2 \sin \mu l = 0
	\end{equation}
	
	非平凡解要求:
	\begin{equation}
		\sin \mu l = 0 \quad \therefore \mu l = n\pi \quad n \text{ 为任意正整数}
	\end{equation}
	
	特征值:
	\begin{equation}
		\mu_n = \frac{n\pi}{l}
	\end{equation}
	
	特征函数:
	\begin{equation}
		X_n = C_2 \sin \frac{n\pi}{l} x \quad n = 1, 2, 3, \ldots \quad \text{(C₂吸收正负号)}
	\end{equation}
	
	特征值:
	\begin{equation}
		k = -\mu^2 = -\left(\frac{n\pi}{l}\right)^2
	\end{equation}
	
	\begin{verification}	
		一阶导数:
		\begin{equation}
			X' = -C_1 \mu \sin \mu x + C_2 \mu \cos \mu x
		\end{equation}
		
		二阶导数:
		\begin{equation}
			X'' = -C_1 \mu^2 \cos \mu x - C_2 \mu^2 \sin \mu x
		\end{equation}
		
		满足方程:
		\begin{equation}
			X'' + \mu^2 X = 0
		\end{equation}
	\end{verification}	
	
	
	\subsubsection{时间常微分方程的求解}
	\(T'' + \left(c \cdot \frac{n\pi}{l}\right)^2 \cdot T = 0 \implies T'' + (c \mu_n)^2 T = 0\),其中 \(\lambda_n = c \mu_n = \frac{c n \pi}{l}\)
	
	同理可得通解:
	\begin{equation}
		T = C_3 \cos \lambda_n t + C_4 \sin \lambda_n t
	\end{equation}
	
	\subsubsection{得偏微分方程通解}
	因此:
	\begin{equation}
		u_n(x, t) = X \cdot T = \sin \frac{n\pi}{l} x \cdot (a_n \cos \lambda_n t + b_n \sin \lambda_n t)
	\end{equation}
	
	由于方程为线性齐次,故可用叠加原理:
	\begin{equation}
		u(x, t) = \sum_{n=1}^{\infty} \sin \frac{n\pi}{l} x \cdot (a_n \cos \lambda_n t + b_n \sin \lambda_n t)
	\end{equation}
	
\subsubsection{初始条件求系数}
原函数初始条件求$a_n$
	
	\begin{equation}
		u(x, 0) = f(x) \quad \frac{\partial u}{\partial t}(x, 0) = g(x)
	\end{equation}
	
	由初始条件:
	\begin{equation}
		u(x, 0) = \sum_{n=1}^{\infty} \sin \frac{n\pi}{l} x \cdot a_n = f(x)
	\end{equation}
	
	利用内积公式(需要$f \in L^2$):
	\begin{equation}
		a_n = \frac{\langle f(x), \sin \frac{n\pi}{l} x \rangle}{\langle \sin \frac{n\pi}{l} x, \sin \frac{n\pi}{l} x \rangle} = \frac{\int_0^l f(x) \cdot \sin \frac{n\pi}{l} x \, dx}{\int_0^l \sin^2 \frac{n\pi}{l} x \, dx}
	\end{equation}
	
	化简得:
	\begin{equation}
		a_n = \frac{2}{l} \cdot \int_0^l f(x) \cdot \sin \frac{n\pi}{l} x \, dx
	\end{equation}
	
	偏导初始条件求$b_n$
	
	对 \(u_n\) 求偏导:
	\begin{equation}
		\frac{\partial u_n}{\partial t}(x, t) = \sin \frac{n\pi}{l} x \cdot \left( -a_n \lambda_n \sin \lambda_n t + b_n \lambda_n \cos \lambda_n t \right)
	\end{equation}
	
	在 \(t = 0\) 时:
	\begin{equation}
		\frac{\partial u_n}{\partial t}(x, 0) = \sin \frac{n\pi}{l} x \cdot b_n \lambda_n
	\end{equation}
	
	对总解求偏导:
	\begin{equation}
		\frac{\partial u}{\partial t}(x, 0) = \sum_{n=1}^{\infty} \frac{\partial u_n}{\partial t}(x, 0) = \sum_{n=1}^{\infty} b_n \lambda_n \sin \frac{n\pi}{l} x = g(x)
	\end{equation}
	
	利用内积公式(需要$f \in L^2$):
	\begin{equation}
		b_n \lambda_n = \frac{\langle g(x), \sin \frac{n\pi}{l} x \rangle}{\langle \sin \frac{n\pi}{l} x, \sin \frac{n\pi}{l} x \rangle} = \frac{2}{l} \int_0^l g(x) \cdot \sin \frac{n\pi}{l} x \, dx
	\end{equation}
	
	化简得:
	\begin{equation}
		b_n = \frac{2}{l \lambda_n} \cdot \int_0^l g(x) \cdot \sin \frac{n\pi}{l} x \, dx = \frac{2}{c n \pi} \int_0^l g(x) \cdot \sin \frac{n\pi}{l} x \, dx
	\end{equation}
	
	\subsubsection{用数分知识求系数,条件和前面泛函内积不一样}
	
	考虑函数 \( f(t) \) 的傅里叶级数展开:
	\begin{equation}
		f = \frac{a_0}{2} + \sum_{n=1}^{\infty} \left( a_n \cos nt + b_n \sin nt \right)
	\end{equation}
	
	计算 \( a_0 \):
	\begin{equation}
		\frac{a_0}{2} = f - \sum_{n=1}^{\infty} \left( a_n \cos nt + b_n \sin nt \right)
	\end{equation}
	
	\begin{equation}
		a_0 = 2f - 2 \sum_{n=1}^{\infty} \left( a_n \cos nt + b_n \sin nt \right)
	\end{equation}
	
	对 \( a_0 \) 积分,若积分和求和可换序:
	\begin{equation}
		\frac{1}{2\pi} \int_{-\pi}^{\pi} a_0 \, dt = \frac{1}{\pi} \int_{-\pi}^{\pi} f \, dt - \sum_{n=1}^{\infty} \frac{1}{\pi} a_n \int_{-\pi}^{\pi} \cos nt \, dt - \sum_{n=1}^{\infty} \frac{1}{\pi} b_n \int_{-\pi}^{\pi} \sin nt \, dt
	\end{equation}
	
	化简得:
	\begin{equation}
		a_0 = \frac{1}{\pi} \int_{-\pi}^{\pi} f \, dt
	\end{equation}
	
	计算 \( a_n \):
	\begin{equation}
		f \cos nt = \frac{a_0}{2} \cos nt + \sum_{k=1}^{\infty} \left( a_k \cos kt + b_k \sin kt \right) \cos nt
	\end{equation}
	
	积分得,若积分和求和可换序:
	\begin{equation}
		\int_{-\pi}^{\pi} f \cos nt \, dt = \int_{-\pi}^{\pi} \frac{a_0}{2} \cos nt \, dt + \sum_{k=1}^{\infty} \left( a_k \int_{-\pi}^{\pi} \cos kt \cos nt \, dt + b_k \int_{-\pi}^{\pi} \sin kt \cos nt \, dt \right)
	\end{equation}
	
	化简得:
	\begin{equation}
		\int_{-\pi}^{\pi} f \cos nt \, dt = a_n \pi
	\end{equation}
	
	因此:
	\begin{equation}
		a_n = \frac{1}{\pi} \int_{-\pi}^{\pi} f \cos nt \, dt
	\end{equation}
	
	同理可得:
	\begin{equation}
		b_n = \frac{1}{\pi} \int_{-\pi}^{\pi} f \sin nt \, dt
	\end{equation}
	
	级数收敛性:
	\begin{equation}
		\sum_{n=1}^{\infty} a_n \cos nx < \infty \qquad \sum_{n=1}^{\infty} b_n \sin nx < \infty
	\end{equation}
	详细条件可以去看我的傅里叶分析笔记。
	
	\subsubsection{总结}
	
	一维波动方程:
	\begin{equation}
		\frac{\partial^2 u}{\partial t^2} = c^2 \cdot \frac{\partial^2 u}{\partial x^2} \qquad 0 < x < l, \quad t > 0
	\end{equation}
	
	边界条件:
	\begin{equation}
		u(0, t) = 0 \quad u(l, t) = 0 \qquad \forall t > 0
	\end{equation}
	
	初始条件:
	\begin{equation}
		u(x, 0) = f(x) \quad \frac{\partial u}{\partial t}(x, 0) = g(x) \qquad 0 < x < l
	\end{equation}
	
	解为:
	\begin{equation}
		u(x, t) = \sum_{n=1}^{\infty} \sin \frac{n\pi}{l} x \cdot \left( a_n \cos \lambda_n t + b_n \sin \lambda_n t \right)
	\end{equation}
	
	其中:
	\begin{equation}
		a_n = \frac{2}{l} \int_0^l f(x) \cdot \sin \frac{n\pi}{l} x \, dx
	\end{equation}
	
	\begin{equation}
		b_n = \frac{2}{c n \pi} \int_0^l g(x) \cdot \sin \frac{n\pi}{l} x \, dx
	\end{equation}
	
	\begin{equation}
		\lambda_n = c \mu_n = \frac{c n \pi}{l}
	\end{equation}
	
		\subsubsection{第二边值条件}
求解过程一样,就是解不同。
	
	\section{热传导方程}
	\subsection{无界齐次热传导方程}
	对于无界的热传导方程:
	\begin{equation}
		\begin{cases}\label{wujierechuandao}
			u_t - \Delta u = 0, \\
			u(x, 0) = \varphi(x),
		\end{cases}
	\end{equation}
	其中,
	\begin{equation}
		\Delta u = \sum_{i=1}^n \frac{\partial^2 u}{\partial x_i^2}.
	\end{equation}
	
	热传导方程的基本解:
	\begin{equation}
		E(x - y, t) = \frac{1}{t^{\frac{n}{2}}} e^{-\frac{|x - y|^2}{4t}}.
	\end{equation}
	
	\subsection{傅里叶变换}
	傅里叶变换定义为:
	\begin{equation}
		\mathcal{F}(f) = \hat{f}(\xi) = \int_{\mathbb{R}^n} f(x) e^{-i x \cdot \xi} \, dx.
	\end{equation}
	
	\subsubsection{微分性质}
	若 \( f \in C \cap L^p \),则:
	\begin{equation}
		\mathcal{F}(f') = (i \xi_j) \mathcal{F}(f).
	\end{equation}
	这里求导是对于$x_j$
	
	\begin{proof}
		\[
			\mathcal{F}(f') = \hat{f'}(\xi) = \int_{\mathbb{R}^n} f'(x) e^{-i x \cdot \xi} \, dx
		\]
		
		对$x_j$做分部积分
		
	\[
			= \left[ f(x) e^{-i x \cdot \xi} \right]_{-\infty}^{+\infty} - \int_{\mathbb{R}^n} f(x) \cdot \frac{\partial}{\partial x_j} \left( e^{-i x \cdot \xi} \right) dx
		\]	
		
		因为\( f \in C \cap L^p \),则$\lim_{|x| \to \infty} f(x) = 0$(衰减性,紧支撑也有),故可化简为:
		
	\[	
			= (i \xi_j) \mathcal{F}(f)
	\]	
	\end{proof}
	
	若 \( f \in C^\alpha \cap L^p \),用多次分布积分,则傅里叶变换的高阶导数性质为:
	\begin{equation}\label{weifen}
		\mathcal{F}(\partial^\alpha f) = (i \xi)^\alpha \mathcal{F}(f)
	\end{equation}
	其中,\(\alpha\) 为多指标,表示高阶导数。
	
	
		\subsubsection{幂乘性质}
	若 \( f \in L^1 \),则傅里叶变换的幂乘性质为:
	\begin{equation}
		\mathcal{F}[-i x_j f(x)] = \frac{\partial}{\partial \xi_j} \mathcal{F}[f](\xi)
	\end{equation}
	
	\begin{proof}
		\[	
		\frac{\partial}{\partial \xi_j} \int_{\mathbb{R}^n} f(x) e^{-i x \cdot \xi} dx = \int_{\mathbb{R}^n} f(x) \cdot \frac{\partial}{\partial \xi_j} e^{-i x \cdot \xi} dx= (-i x_j) \cdot \int_{\mathbb{R}^n} f(x) e^{-i x \cdot \xi} dx
	\]	
	
\end{proof}
	
	
	\subsubsection{傅里叶变换的卷积性质}
		卷积定义:
	\begin{equation}
		(f * g)(x) = \int_{\mathbb{R}^n} f(y) \cdot g(x - y) \, dy
	\end{equation}
	
	若\( f,g \in L^1 \),卷积性质:
	\begin{equation}
		\mathcal{F}[f * g] = \mathcal{F}[f] \cdot \mathcal{F}[g]
	\end{equation}


		\begin{proof}
	\[	
		\mathcal{F}[f * g] = \int_{\mathbb{R}^n} e^{-i x \cdot \xi} \left( \int_{\mathbb{R}^n} f(y) \cdot g(x - y) \, dy \right) dx
	\]	
	
	
	根据 Fubini 定理,若\( f,g \in L^1 \),我们交换外层关于 \(x\) 的积分和内层关于 \(y\) 的积分:
	
	\[
	\mathcal{F}[f * g] = \int_{\mathbb{R}^n} f(y) \cdot \left( \int_{\mathbb{R}^n} e^{-i x \cdot \xi} g(x - y) dx \right) dy
	\]
	
	变量替换 \( z = x - y \),即 \( x = z + y \),则 \( dz = dx \),代入后:
	\[	
		= \int_{\mathbb{R}^n} f(y) \cdot \left( \int_{\mathbb{R}^n} e^{-i (z + y) \cdot \xi} g(z) \, dz \right) dy
	\]	
	
	化简指数项:
		\[	
		= \int_{\mathbb{R}^n} e^{-i y \cdot \xi} \cdot \left( \int_{\mathbb{R}^n} e^{-i z \cdot \xi} g(z) \, dz \right) f(y) \, dy	= \mathcal{F}[g] \cdot \int_{\mathbb{R}^n} e^{-i y \cdot \xi} f(y) \, dy = \mathcal{F}[g] \cdot \mathcal{F}[f]
	\]	
	
	
\end{proof}
	
		\subsubsection{傅里叶逆变换的卷积性质}
	若\( f,g \in L^1 \)
	\begin{equation}\label{nibianhuanjuanji}
		\mathcal{F}^{-1}[f \cdot g] = \mathcal{F}^{-1}[f] * \mathcal{F}^{-1}[g]
	\end{equation}
	
	\begin{proof}
		首先,根据傅里叶逆变换的定义:
		\[
		\mathcal{F}^{-1}[f \cdot g](x) = \frac{1}{(2\pi)^n} \int_{\mathbb{R}^n} e^{i x \cdot \xi} (f(\xi) \cdot g(\xi)) d\xi
		\]
		
		卷积的定义:
		\[
		(\mathcal{F}^{-1}[f] * \mathcal{F}^{-1}[g])(x) = \int_{\mathbb{R}^n} \mathcal{F}^{-1}[f](y) \cdot \mathcal{F}^{-1}[g](x - y) dy
		\]
		
		根据傅里叶逆变换的定义,将其展开:
		\[
		\mathcal{F}^{-1}[f](y) = \frac{1}{(2\pi)^n} \int_{\mathbb{R}^n} e^{i y \cdot \xi} f(\xi) d\xi
		\]
		\[
		\mathcal{F}^{-1}[g](x - y) = \frac{1}{(2\pi)^n} \int_{\mathbb{R}^n} e^{i (x - y) \cdot \eta} g(\eta) d\eta
		\]
		
		代入卷积表达式:
		\[
		(\mathcal{F}^{-1}[f] * \mathcal{F}^{-1}[g])(x) = \int_{\mathbb{R}^n} \left( \frac{1}{(2\pi)^n} \int_{\mathbb{R}^n} e^{i y \cdot \xi} f(\xi) d\xi \right) \cdot \left( \frac{1}{(2\pi)^n} \int_{\mathbb{R}^n} e^{i (x - y) \cdot \eta} g(\eta) d\eta \right) dy
		\]
		
			根据 Fubini 定理,若\( f,g \in L^1 \),交换积分顺序:
		\[
		(\mathcal{F}^{-1}[f] * \mathcal{F}^{-1}[g])(x) = \frac{1}{(2\pi)^{2n}} \int_{\mathbb{R}^n} \int_{\mathbb{R}^n} \left( \int_{\mathbb{R}^n} e^{i y \cdot \xi} e^{i (x - y) \cdot \eta} dy \right) f(\xi) g(\eta) d\xi d\eta
		\]
		
		合并指数项:
		\[
		= \frac{1}{(2\pi)^{2n}} \int_{\mathbb{R}^n} \int_{\mathbb{R}^n} e^{i x \cdot \eta} \left( \int_{\mathbb{R}^n} e^{i y \cdot (\xi - \eta)} dy \right) f(\xi) g(\eta) d\xi d\eta
		\]
		
		结果:
		\[
		= \frac{1}{(2\pi)^{2n}} \int_{\mathbb{R}^n} e^{i x \cdot \xi} (2\pi)^n f(\xi) g(\xi) d\xi= \frac{1}{(2\pi)^n} \int_{\mathbb{R}^n} e^{i x \cdot \xi} f(\xi) g(\xi) d\xi = \mathcal{F}^{-1}[f \cdot g](x)
		\]

		

	\end{proof}
	
	
		\subsection{高斯型函数的一些积分}
		
		下面是高斯型函数一些积分的计算
	
	
	
	\begin{example}
		\label{ex:1}
		$	\int_{-\infty}^{+\infty} e^{-a(x-b)^2} dx = \sqrt{\frac{\pi}{a}}$
	\end{example}
	
	
	\begin{proof}
		\begin{equation*}
			A^2 = \left( \int_{-\infty}^{+\infty} e^{-a(x-b)^2} dx \right) \left( \int_{-\infty}^{+\infty} e^{-a(y-b)^2} dy \right)
		\end{equation*}
		
		\begin{equation*}
			= \int_{-\infty}^{+\infty} \int_{-\infty}^{+\infty} e^{-a((x-b)^2 + (y-b)^2)} dx dy
		\end{equation*}
		
		由极坐标变量替换可得
		
		\begin{equation*}
			(x - b)^2 + (y - b)^2 = r^2
		\end{equation*}
		
		\begin{equation*}
			\begin{cases}
				x - b = r \cos\theta \\
				y - b = r \sin\theta
			\end{cases}
		\end{equation*}
		
		\begin{equation*}
			\begin{cases}
				dx = \cos\theta \, dr - r \sin\theta \, d\theta \\
				dy = \sin\theta \, dr + r \cos\theta \, d\theta
			\end{cases}
		\end{equation*}
		
		\begin{equation*}
			J = \begin{vmatrix}
				\cos\theta & -r \sin\theta \\
				\sin\theta & r \cos\theta
			\end{vmatrix} = r
		\end{equation*}
		
		\begin{equation*}
			dx dy = r \, d\theta \, dr
		\end{equation*}
		
		\begin{equation*}
			A^2 = \int_{-\pi}^{\pi} \int_{0}^{\infty} e^{-ar^2} r \, dr \, d\theta	= \frac{1}{2} \int_{-\pi}^{\pi} \int_{0}^{\infty} e^{-ar^2} d(r^2) \, d\theta
		\end{equation*}
		
		
		\begin{equation*}
			= \frac{1}{2} \int_{-\pi}^{\pi} \left[ -\frac{1}{a} e^{-ar^2} \Big|_{0}^{+\infty} \right] d\theta= \frac{1}{2} \int_{-\pi}^{\pi} \frac{1}{a} \, d\theta	= \frac{\pi}{a}
		\end{equation*}
		
		
		\begin{equation*}
			\therefore	\int_{-\infty}^{+\infty} e^{-a(x-b)^2} dx = \sqrt{\frac{\pi}{a}}
		\end{equation*}
		
	\end{proof}
	
	\begin{example}
		\label{ex:2}
		$	\int_{c}^{+\infty} e^{-a(x-b)^2} dx$
		积分没有原函数
	\end{example}
	
	\begin{proof}
		
		
		方法1.不成熟的证明方法望指正
		
		
		
		
		
		\begin{equation*}
			A^2 = \left( \int_{0}^{+\infty} e^{-a(x-b)^2} dx \right) \left( \int_{0}^{+\infty} e^{-a(y-b)^2} dy \right)
		\end{equation*}
		
		\begin{equation*}
			\neq \int_{0}^{\frac{\pi}{2}} \int_{0}^{+\infty} e^{-ar^2} \cdot r \, dr \, d\theta
		\end{equation*}
		在这里不能像上面那样做极坐标变量代换,因为圆心为$(b,b)$,而坐标点不是全平面,当$r$大到一定程度,角度再也不能转一圈,角度会随着$r$的增大而变小,无法用二重积分的定义表示。
		
		
	\end{proof}
	
	
	\begin{example}\label{ex:3}
		$\int_{-\infty}^{+\infty} \cos kx \cdot e^{-a x^2} dx=\int_{-\infty}^{+\infty} e^{ikx} \cdot e^{-a x^2} dx= e^{-\frac{k^2}{4a}} \cdot \sqrt{\frac{\pi}{a}}$
	\end{example}
	
	\begin{proof}
		
		\begin{equation*}
			\int_{-\infty}^{+\infty} e^{ikx} \cdot e^{-a x^2} dx =\int_{-\infty}^{+\infty} \cos kx \cdot e^{-a x^2} dx+i\int_{-\infty}^{+\infty} \sin kx \cdot e^{-a x^2} dx
		\end{equation*}
		第二项为奇函数,积分为0,所以,
		\begin{equation*}
			\int_{-\infty}^{+\infty} e^{ikx} \cdot e^{-a x^2} dx =\int_{-\infty}^{+\infty} \cos kx \cdot e^{-a x^2} dx
		\end{equation*}
		
		\begin{equation*}
			\int_{-\infty}^{+\infty} e^{ikx} \cdot e^{-a x^2} dx = \int_{-\infty}^{+\infty} e^{-a x^2 + ikx} dx
		\end{equation*}
		
		配方
		\begin{equation*}
			\begin{split}
				-a x^2 + ikx &= -a \left( x^2 - \frac{ik}{a} x \right) \\
				&= -a \left[ x^2 - \frac{ik}{a} x + \left( \frac{ik}{2a} \right)^2 - \left( \frac{ik}{2a} \right)^2 \right] \\
				&= -a \left( x - \frac{ik}{2a} \right)^2 - \frac{k^2}{4a}
			\end{split}
		\end{equation*}
		所以,
		\begin{equation*}
			\int_{-\infty}^{+\infty} e^{ikx} \cdot e^{-a x^2} dx = \int_{-\infty}^{+\infty} e^{-a x^2 + ikx} dx	= e^{-\frac{k^2}{4a}} \cdot \sqrt{\frac{\pi}{a}}
		\end{equation*}
	\end{proof}
	
	
	\begin{example}\label{ex:4}
		$\int_{-\infty}^{+\infty} z^2 e^{-a z^2} dz = \frac{1}{2} \pi^{\frac{1}{2}} a^{-\frac{3}{2}}$
	\end{example}
	
	\begin{proof}
		
		
		\begin{equation*}
			\text{设} \quad \Phi(a) = \int_{-\infty}^{+\infty} e^{-a z^2} dz = \sqrt{\frac{\pi}{a}}
		\end{equation*}
		
		\begin{equation*}
			\frac{d\Phi}{da} = -\int_{-\infty}^{+\infty} z^2 e^{-a z^2} dz = \frac{d}{da} \left( \pi^{\frac{1}{2}} a^{-\frac{1}{2}} \right) = \pi^{\frac{1}{2}} a^{-\frac{3}{2}} \cdot \left( -\frac{1}{2} \right)
		\end{equation*}
		
		\begin{equation*}
			\therefore \int_{-\infty}^{+\infty} z^2 e^{-a z^2} dz = \frac{1}{2} \pi^{\frac{1}{2}} a^{-\frac{3}{2}}
		\end{equation*}
	\end{proof}
	
	
	
	\begin{example}\label{ex:5}
		$	\int_{-\infty}^{+\infty} (z + bx)^2 e^{-a z^2} dz=	= \frac{1}{2} \pi^{\frac{1}{2}} a^{-\frac{3}{2}} + b^2 x^2 \sqrt{\frac{\pi}{a}}$
	\end{example}
	
	\begin{proof}
		展开得
		\begin{equation*}
			= \int_{-\infty}^{+\infty} z^2 e^{-a z^2} dz + \int_{-\infty}^{+\infty} 2 z b x e^{-a z^2} dz + b^2 x^2 \int_{-\infty}^{+\infty} e^{-a z^2} dz
		\end{equation*}
		
		\begin{equation*}
			= \frac{1}{2} \pi^{\frac{1}{2}} a^{-\frac{3}{2}} + b^2 x^2 \sqrt{\frac{\pi}{a}}
		\end{equation*}
		
	\end{proof}
	
	
	
	
	
	
	
	
	\subsection{解的导出}
	对于无界的热传导方程\eqref{wujierechuandao}:
	\begin{equation}
		\begin{cases}
			u_t - \Delta u = 0, \\
			u(x, 0) = \varphi(x),
		\end{cases}
	\end{equation}
	
	设初值问题的解 \( u(x, t) \) 和初始数据 \( \varphi(x) \) 都可关于变量 \( x \) 进行 Fourier 变换,并记:
	\begin{equation}
		\hat{u}(\xi, t) = \int_{\mathbb{R}^n} u(x, t) e^{-i x \cdot \xi} dx,
	\end{equation}
	\begin{equation}
		\hat{\varphi}(\xi) = \int_{\mathbb{R}^n} \varphi(x) e^{-i x \cdot \xi} dx.
	\end{equation}
	
	对热传导方程和初始条件进行 Fourier 变换,根据傅里叶的微分性质\eqref{weifen}:
		\begin{equation}
	\hat{u_t}=\int_{\mathbb{R}^n} u_t(x, t) e^{-i x \cdot \xi} dx=\frac{d}{dt} \int_{\mathbb{R}^n} u(x, t) e^{-i x \cdot \xi} dx= \frac{d\hat{u}(\xi, t)}{dt}
		\end{equation}
	
	\[
	\Delta u = \sum_{j=1}^n \frac{\partial^2 u}{\partial x_j^2}
	\]
		\[
	\mathcal{F}\left[\frac{\partial^2 u}{\partial x_j^2}\right](\xi) = -\xi_j^2 \hat{u}(\xi)
		\]
		\begin{equation}
		\mathcal{F}[\Delta u](\xi) = \sum_{j=1}^n (-\xi_j^2) \hat{u}(\xi) = -|\xi|^2 \hat{u}(\xi)
			\end{equation}
		
		
	把$\xi$看作常量,得到关于 \(\hat{u}(\xi, t)\) 的常微分方程初值问题:
	\begin{equation}
		\begin{cases}
			\displaystyle \frac{d\hat{u}(\xi, t)}{dt} + |\xi|^2 \hat{u}(\xi, t) = 0, \\
			\hat{u}(\xi, 0) = \hat{\varphi}(\xi).
		\end{cases}
	\end{equation}
		该方程的解为:
	\begin{equation}
		\hat{u}(\xi, t) = \hat{\varphi}(\xi) e^{-|\xi|^2 t}.
	\end{equation}
	
	
	
		\begin{proof}
	这是一个一阶线性常微分方程,可以通过分离变量法求解。
	将方程改写为:
	\[
	\frac{d\hat{u}}{\hat{u}} = -a^2 |\xi|^2 dt
	\]
	
	对两边积分:
	\[
	\int \frac{d\hat{u}}{\hat{u}} = \int -a^2 |\xi|^2 dt
	\]
	
	得到:
	\[
	\ln|\hat{u}| = -a^2 |\xi|^2 t + C(\xi)
	\]
	其中,\(C(\xi)\) 是积分常数,可能依赖于 \(\xi\)。
	
	
	利用初始条件 \(\hat{u}(\xi, 0) = \hat{\varphi}(\xi)\),代入上式得:
	\[
	C(\xi) = \ln|\hat{\varphi}(\xi)|
	\]
	
	将 \(C(\xi)\) 代入积分结果:
	\[
	\ln|\hat{u}| = -a^2 |\xi|^2 t + \ln|\hat{\varphi}(\xi)|
	\]
	
	对两边取指数:
	\[
	\hat{u}(\xi, t) = \hat{\varphi}(\xi) e^{-a^2 |\xi|^2 t}
	\]

	
		\end{proof}
	
	
对 \(\hat{u}(\xi, t)\) 进行 Fourier 逆变换
	\begin{equation}
		u(x, t) = \mathcal{F}^{-1}[\hat{\varphi}(\xi) e^{-|\xi|^2 t}]
	\end{equation}
	
	利用Fourier 逆变换卷积的性质\eqref{nibianhuanjuanji}:
	\begin{equation}
		u(x, t) = \mathcal{F}^{-1}[\hat{\varphi}(\xi)] * \mathcal{F}^{-1}[e^{-|\xi|^2 t}]
	\end{equation}
	
	\begin{example}
		我们需要证明:
		\[
		\mathcal{F}^{-1}\left[e^{-|\xi|^2 t}\right](x) = (4\pi t)^{-n/2} e^{-\frac{|x|^2}{4t}}
		\]
	\end{example}
	

	
	\begin{proof}
		
		根据傅里叶逆变换的定义:
		\[
		\mathcal{F}^{-1}\left[e^{-|\xi|^2 t}\right](x) = \frac{1}{(2\pi)^n} \int_{\mathbb{R}^n} e^{i x \cdot \xi} e^{-|\xi|^2 t} d\xi
		\]
		
		我们可以将这个积分分解为每个坐标的积分:
		\[
		\mathcal{F}^{-1}\left[e^{-|\xi|^2 t}\right](x) = \frac{1}{(2\pi)^n} \prod_{j=1}^n \int_{-\infty}^\infty e^{i x_j \xi_j} e^{-\xi_j^2 t} d\xi_j
		\]
		
	\begin{remark}
	\[
	\mathcal{F}^{-1}\left[e^{-|\xi|^{2} t}\right](x)=\frac{1}{(2\pi)^{3}}\left(\int_{-\infty}^{\infty} e^{i x_1 \xi_1} e^{-\xi_1^{2} t} d\xi_1\right)\left(\int_{-\infty}^{\infty} e^{i x_2 \xi_2} e^{-\xi_2^{2} t} d\xi_2\right)\left(\int_{-\infty}^{\infty} e^{i x_3 \xi_3} e^{-\xi_3^{2} t} d\xi_3\right)
	\]
	\end{remark}
		
		
		对于每个一维积分,我们有:
		\[
		\int_{-\infty}^\infty e^{i x_j \xi_j} e^{-\xi_j^2 t} d\xi_j
		\]
		
		对指数部分进行配方:
		\[
		-t \xi_j^2 + i x_j \xi_j = -t \left( \xi_j^2 - \frac{i x_j}{t} \xi_j \right)
		\]
		\[
		= -t \left( \xi_j^2 - \frac{i x_j}{t} \xi_j + \left( \frac{x_j}{2t} \right)^2 - \left( \frac{x_j}{2t} \right)^2 \right)
		\]
		\[
		= -t \left( \left( \xi_j - \frac{i x_j}{2t} \right)^2 - \frac{x_j^2}{4t^2} \right)
		\]
		\[
		= -t \left( \xi_j - \frac{i x_j}{2t} \right)^2 + \frac{x_j^2}{4t}
		\]
		
		代入积分中,由命题\eqref{ex:1}:
		\[
		\int_{-\infty}^\infty e^{-t \left( \xi_j - \frac{i x_j}{2t} \right)^2 + \frac{x_j^2}{4t}} d\xi_j = e^{\frac{x_j^2}{4t}} \int_{-\infty}^\infty e^{-t \left( \xi_j - \frac{i x_j}{2t} \right)^2} d\xi_j= \sqrt{\frac{\pi}{t}} e^{\frac{x_j^2}{4t}}
		\]
	
		
		将所有坐标的结果相乘,并乘以系数 \( \frac{1}{(2\pi)^n} \),得到:
		\[
		\mathcal{F}^{-1}\left[e^{-|\xi|^2 t}\right](x) = \frac{1}{(2\pi)^n} \left( \sqrt{\frac{\pi}{t}} \right)^n e^{-\frac{|x|^2}{4t}}
		\]
		\[
		= \frac{1}{(2\pi)^n} \cdot \frac{\pi^{n/2}}{t^{n/2}} e^{-\frac{|x|^2}{4t}}= \frac{1}{(2)^n t^{n/2}} e^{-\frac{|x|^2}{4t}}= (4 \pi t)^{-n/2} e^{-\frac{|x|^2}{4t}}
		\]

	\end{proof}
	

	
	
	
	因此,解可以表示为:
	\begin{equation}
		u(x, t) = (4\pi t)^{-n/2} \int_{\mathbb{R}^n} \varphi(y) e^{-\frac{|x - y|^2}{4t}} dy
	\end{equation}
	
	其中,\( E(x - y, t) = (4\pi t)^{-n/2} e^{-\frac{|x - y|^2}{4t}} \) 称为热传导方程的基本解。
	
	\subsection{热传导方程齐次化}
	和之前波动方程齐次化一样,把边界条件齐次化,用Duhamamel原理把方程齐次化。有界无界的情况,把基本解看作达朗贝尔公式,延拓和上面一模样;分离变量的方法也一样,就是时间常微分函数变成一阶,其他全部的求解过程,验证过程全部一样。
	
	
	
	
	%弱解
%	\href{https://en.wikipedia.org/wiki/Weak_solution}{(Weak Solution)}
	
	
	
\end{document}