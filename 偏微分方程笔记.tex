\documentclass[12pt,a4paper]{article}
\usepackage[UTF8]{ctex}
\usepackage{geometry}
\usepackage{amsmath,amssymb,amsthm}
\usepackage{mathtools,bm}
\usepackage{empheq}
\usepackage{graphicx}
\usepackage{booktabs}
\usepackage[numbers,sort&compress]{natbib}
\usepackage{caption}
\usepackage{enumitem}
\usepackage{chngcntr}

% ========== 页面布局 ==========
\geometry{left=2.5cm,right=2.5cm,top=2.5cm,bottom=2.5cm}
\setlength{\parskip}{0.5em}
\renewcommand{\baselinestretch}{1.2}

% ========== 数学命令 ==========
\newcommand{\diff}{\mathop{}\!\mathrm{d}}
\newcommand{\R}{\mathbb{R}}
\newcommand{\C}{\mathbb{C}}
\newcommand{\Z}{\mathbb{Z}}
\newcommand{\N}{\mathbb{N}}
\DeclareMathOperator{\supp}{supp}

% ========== 编号系统 ==========
\numberwithin{subsection}{section}   % 子小节按章节编号
\numberwithin{subsubsection}{subsection}
\counterwithin{equation}{subsection} % 公式按子小节编号

% ========== 定理环境 ==========
\theoremstyle{plain}
\newtheorem{theorem}{定理}[section]
\newtheorem{lemma}[theorem]{引理}
\newtheorem{proposition}[theorem]{命题}
\newtheorem{corollary}[theorem]{推论}
\newtheorem{solution}{解}[subsection]  % 解按子小节编号

\theoremstyle{definition}
\newtheorem{definition}[theorem]{定义}
\newtheorem{example}{示例}[subsection]  % 示例按子小节编号

\theoremstyle{remark}
\newtheorem{remark}[theorem]{注记}

\theoremstyle{remark}
\newtheorem{verification}[theorem]{验证}

% 加载hyperref包以实现超链接功能
\usepackage[colorlinks=true, linkcolor=black]{hyperref}

\title{偏微分方程笔记}
\author{陈柏均}
\date{2025年5月12日}

\begin{document}
	
	\maketitle


\newpage
	
	\tableofcontents  % 添加目录
	
	
	\newpage
	
	\section{一阶拟线性方程之齐次传输方程} 
	\subsection{变量替换求解常系数齐次传输方程} 
	\subsubsection{问题描述}
	
	
	假设 $a_1 \neq 0$ 且 $a_2 \neq 0$,我们求解常系数传输方程:
	\begin{equation} \label{eq:pde_original}
		a_1 \frac{\partial u}{\partial t} + a_2 \frac{\partial u}{\partial x} = 0
	\end{equation}
	
	\subsubsection{通解} 
	核心思想:通过变量替换,把二元偏微分转化成一元的常微分求解。
	
	其中 $u = u(t,x)$。引入坐标变换 $(\alpha, \beta)$,使得 $u = u(\alpha, \beta)$,且:
	\begin{equation} \label{eq:coordinate_transform}
		\begin{cases}
			\alpha = ax + bt, \\
			\beta = cx + dt.
		\end{cases}
	\end{equation}
	利用链式法则计算偏导数:
	\begin{align}
		\frac{\partial u}{\partial t} 
		&= \frac{\partial u}{\partial \alpha} \frac{\partial \alpha}{\partial t} + \frac{\partial u}{\partial \beta} \frac{\partial \beta}{\partial t} 
		= b\frac{\partial u}{\partial \alpha} + d\frac{\partial u}{\partial \beta}, \label{eq:u_t_chain_rule} \\
		\frac{\partial u}{\partial x} 
		&= \frac{\partial u}{\partial \alpha} \frac{\partial \alpha}{\partial x} + \frac{\partial u}{\partial \beta} \frac{\partial \beta}{\partial x} 
		= a\frac{\partial u}{\partial \alpha} + c\frac{\partial u}{\partial \beta}. \label{eq:u_x_chain_rule}
	\end{align}
	
	将 \eqref{eq:u_t_chain_rule} 和 \eqref{eq:u_x_chain_rule} 代入原方程 \eqref{eq:pde_original}:
	\begin{equation} \label{eq:pde_transformed}
		a_1 \left( b \frac{\partial u}{\partial \alpha} + d \frac{\partial u}{\partial \beta} \right) + a_2 \left( a \frac{\partial u}{\partial \alpha} + c \frac{\partial u}{\partial \beta} \right) = 0.
	\end{equation}
	整理后得到:
	\begin{equation} \label{eq:pde_collected}
		(a_1 b + a_2 a) \frac{\partial u}{\partial \alpha} + (a_1 d + a_2 c) \frac{\partial u}{\partial \beta} = 0.
	\end{equation}
	为消去一个变量,pde转ode,选择让第二项系数为0,把方程 \eqref{eq:pde_collected} 简化为:
	\begin{equation} \label{eq:pde_final}
		\frac{\partial u}{\partial \alpha} = 0.
	\end{equation}
	选择系数
	\begin{equation} \label{eq:constant_choice}
		\begin{cases}
			a = 0, & b = 1, \\
			c = a_1, & d = -a_2.
		\end{cases}
	\end{equation}
	此时坐标变换为:
	\begin{equation} \label{eq:coordinates_specific}
		\begin{cases}
			\alpha = t, \\
			\beta = a_1 x - a_2 t.
		\end{cases}
	\end{equation}
	由\eqref{eq:pde_final} 表明 $u$ 仅依赖于 $\beta$,即通解为:
	\begin{equation} \label{eq:solution}
		u(t,x) = L(a_1 x - a_2 t),
	\end{equation}
	其中
	$L(\cdot)$ 是任意可微函数。
	
		\subsubsection{特解(初始条件或边界条件)} 
	已知初始条件$	u(x, 0) = e^{-x^2}$,求下面常系数运输方程:
	\begin{equation} 
		\frac{\partial u}{\partial t} +  \frac{\partial u}{\partial x} = 0
	\end{equation}
	
	由\eqref{eq:solution}可知
	\begin{equation}
		u(x, t) = f(x - t)= e^{-(x-t)^2}
	\end{equation}
	
	
	\subsection{波的传播求解常系数齐次传输方程} 
	\subsubsection{问题描述}
	在一阶线性方程中,有一种最简单的形如
	
	\begin{equation}
		u_t + b \cdot \mathrm{D}u = 0, \quad x \in \mathbb{R}^n, \ t \in (0, \infty)
	\end{equation}
	
	的方程,称为传输方程,其中,\(b = (b_1, b_2, \cdots, b_n)\) 是已知 \(n\) 维常向量,\(u = u(x, t)\),\(\mathrm{D}u = (u_{x_1}, u_{x_2}, \cdots, u_{x_n})\)。
	
		\subsubsection{通解} 
			\begin{equation}
			\frac{\partial u}{\partial t} + b \frac{\partial u}{\partial x}=(1, b) \cdot \left( \frac{\partial u}{\partial t}, \frac{\partial u}{\partial x} \right) = 0 
		\end{equation}
		$\left( \frac{\partial u}{\partial x}, \frac{\partial u}{\partial y} \right)$为梯度,$(1, b)$为方向,一整个乘积为方向导数,方向导数为0意味着,$u(t, x)=C$在切向量为$(1, b)$这条曲线上,即
		\begin{equation}
			u(t,x)|_{\Gamma} = C
		\end{equation}
		
	
	由方程的形式可以看出,\(u(x, t)\) 沿$(1, b)$微商等于零。事实上,固定一点 \((x, t) \in \mathbb{R}^{n+1}\),记过该直线$\Gamma$的参数方程为 \((x + bs, t + s), s \in \mathbb{R}\),考查函数 \(u\) 在该直线上的值。令
\begin{equation}
	z(s) = u(x + bs, t + s), \quad s \in \mathbb{R}.
	\end{equation}
	
	于是
	\begin{equation}
	\frac{\mathrm{d}z}{\mathrm{d}s} = \mathrm{D}u(x + sb, t + s) \cdot b + u_t(x + sb, t + s) = 0,
	\end{equation}
	
	因此,函数 \(z(s)\) 在过点 \((x, t)\) 且具有方向 \((b, 1) \in \mathbb{R}^{n+1}\) 的直线上取常数值,特征线上的取值和$s$没有关系(和下文中特征线法求解传输方程的$(1, p(x, y))$含义相同)。所以,如果我们知道解 \(u\) 在这条直线上一点的值,则就得到它沿此直线上的值。这就引出下面求解初值问题的方法。
	
	\subsubsection{初值问题之特解} 
	设 $a \in \mathbb{R}^n$ 是已知常向量,$f: \mathbb{R}^n \rightarrow \mathbb{R}$ 是给定函数。考察传输方程的初值问题
	\begin{equation}
		\begin{cases}
			u_t + a \cdot \mathrm{D}u = 0, & (x, t) \in \mathbb{R}^n \times (0, \infty), \\
			u(x, 0) = f(x), & x \in \mathbb{R}^n.
		\end{cases}
	\end{equation}
	
	如上取定 $(x, t)$,过点 $(x, t)$ 且具有方向 $(a, 1)$ 的直线的参数式为 $(x + a s, t + s)$,$s \in \mathbb{R}$。当 $s = -t$ 时,此直线与平面 $\Gamma: \mathbb{R}^n \times \{t = 0\}$ 相交于点 $(x - a t, 0)$。由上文分析知 $u$ 沿此直线取常数值,而由初值条件便得
	\begin{equation}\label{eq:齐次解}
	u(x,t)=	z(0)=z(-t) =u(x - a t, 0) = f(x - a t), \quad x \in \mathbb{R}^n, \ t \geq 0.
	\end{equation}
	
	\begin{remark}
	这表示对于每一个特定的点都有一条特征线,他的函数为特定的$f$。取遍每个特征线就能取遍域内所有点,对于任意的点都有任意的函数表达式。因为上面的式子,at是任意的,所以x-at是任意的,可以取遍整个
	\end{remark}
		
	所以,如果 有解,必由上式子 表示,因此解是唯一的;反之,若 $f$ 一阶连续可微,则可直接验证由 上式子表示的函数 $u(x, t)$ 是问题的解。这就是齐次传输方程初值问题解的存在唯一性。
	
\subsection{波的传播求解常系数非齐次传输方程} 
\subsubsection{问题描述}
	考察非齐次传输方程的初值问题
	\begin{equation}
		\begin{cases}
			u_t + a \cdot \mathrm{D}u = f, & x \in \mathbb{R}^n, t > 0, \\
			u(x,0) = g(x)
		\end{cases}
	\end{equation}
	
\subsubsection{求解}
	受齐次问题解法的启示,我们仍然先取定 \((x, t) \in \mathbb{R}^{n+1}\),对 \(s \in \mathbb{R}\),令 \(z(s) = u(x + a s, t + s)\),则
		\begin{equation}
	\frac{\mathrm{d}z}{\mathrm{d}s} = \mathrm{D}u(x + a s, t + s) \cdot a + u_t(x + a s, t + s) = f(x + a s, t + s).
		\end{equation}
	
	因此,
	\begin{equation}
	\begin{aligned}
		u(x, t) -	u(x-at,0)&= u(x, t)-g(x - a t) \\
		&= z(0) - z(-t) = \int_{-t}^0 \frac{\mathrm{d}z}{\mathrm{d}s} \, \mathrm{d}s \\
		&= \int_{-t}^0 f(x + a s, t + s) \, \mathrm{d}s \\
		&= \int_0^t f(x + a (s - t), s) \, \mathrm{d}s.
	\end{aligned}
\end{equation}
	
	于是,得到问题 的在 \(x \in \mathbb{R}^n\),\(t \geq 0\) 上的解
	\begin{equation}\label{eq:非齐次解1}
		u(x, t) = g(x - a t) + \int_0^t f(x + a (s - t), s) \, \mathrm{d}s.
	\end{equation}
	
	在下一章,这个公式将被用来求解一维波动方程。
	
	
	
	\subsection{特征线法求解变系数齐次传输方程} 
	\subsubsection{通解} 
	一阶线性变系数偏微分方程如下:
	\begin{equation}\label{eq:pde_original2}
		\frac{\partial u}{\partial x} + p(x,y) \frac{\partial u}{\partial y}=(1, p(x, y)) \cdot \left( \frac{\partial u}{\partial x}, \frac{\partial u}{\partial y} \right) = 0 
	\end{equation}
	其中 $p(x, y)$ 是 $x$ 和 $y$ 的函数。
	$\left( \frac{\partial u}{\partial x}, \frac{\partial u}{\partial y} \right)$为梯度,$(1, p(x, y))$为方向,一整个乘积为方向导数,方向导数为0意味着,$u(x, y)=C$在切向量为$(1, p(x, y))$这条曲线上,即
	\begin{equation}
		u(x,y)|_{\Gamma} = C
	\end{equation}
	\begin{equation}
		u(x,y) =  f(C)
	\end{equation}
	$\Gamma$曲线上,任意点$(x, y)$求导($\Gamma$曲线为$XOY$平面上的曲线,故$y$可表示成$x$的函数),可得切向量$(1,\frac{dy}{dx})$
	
	所以我们找到$\Gamma$曲线,把二元偏微分转化成一元的常微分,令
	
	\begin{equation}
		\frac{dy}{dx} = p(x, y)
	\end{equation}
	可解得
	\begin{equation}
		C=\phi(x,y)
	\end{equation}
	得方程解
	\begin{equation}
		u(x, y)=f(C)=f(\phi(x,y))
	\end{equation}
	$(1, \frac{dy}{dx})$
	为该曲线的切向量。我们称这条曲线叫特征线。只需要取遍所有的特征曲线就可以取遍$XOY$平面上所有的点,若有初始条件或者边界条件可以确定每条特征线在$u(x, y)$对应的取值,就可以完整确定$u(x, y)$这个函数。
	
	\begin{example}求解方程
		\begin{equation}
			\frac{\partial u}{\partial x} + x \frac{\partial u}{\partial y} = 0.
		\end{equation}
		
		此时我们有 $p(x, y) = x$,解 $\frac{dy}{dx} = x$,我们得到特征线 $y = \frac{1}{2}x^2 + C$,或 $y - \frac{1}{2}x^2 = C$。从而 $\phi(x, y) = y - \frac{1}{2}x^2$,偏微分方程的通解为 $u(x, y) = f(\phi(x, y))$,其中 $f$ 是任意函数。把它们代回方程,直接验证,便知是解。
	\end{example}
	
	\newpage
	
	\section{一维齐次波动方程的初值问题}
	
	\subsection{d’Alembert 公式}
	
	\subsubsection{问题描述}
	先考察初值问题
	
	\begin{equation}
		\begin{cases}
			u_{tt} - a^2 u_{xx} = 0, & x \in \mathbb{R}, t > 0, \\
			u(x, 0) = \varphi(x), \quad u_t(x, 0) = \psi(x), & x \in \mathbb{R}.
		\end{cases}
	\end{equation}
	
\subsubsection{求解}
	由算子复合作用的概念,易验证下述算子因式分解
	
	\begin{equation}
		\left( \frac{\partial}{\partial t} + a \frac{\partial}{\partial x} \right) \left( \frac{\partial}{\partial t} - a \frac{\partial}{\partial x} \right) u = u_{tt} - a^2 u_{xx} = 0.
	\end{equation}
	
	令
	
	\begin{equation}
		v(x, t) = \left( \frac{\partial}{\partial t} - a \frac{\partial}{\partial x} \right) u.
	\end{equation}
	
	由 (2.1.2), 得
	\begin{equation}
	v_t(x, t) + a v_x(x, t) = 0, \quad x \in \mathbb{R}, t > 0.
\end{equation}
	
	这是一维传输方程,且由 (2.1.3) 知 \(v\) 满足初值条件
	
	\begin{equation}
	v(x, 0) = \psi(x) - a \varphi'(x).
\end{equation}
	
	由 \eqref{eq:齐次解}, 得
	\begin{equation}
		v(x, t) = \psi(x - a t) - a \varphi'(x - a t).
	\end{equation}
	
	将 \(v\) 代入 (2.1.3),得
	\begin{equation}
		u_t(x, t) - a u_x(x, t) = \psi(x - a t) - a \varphi'(x - a t),
	\end{equation}
	其中 \((x, t) \in \mathbb{R} \times (0, \infty)\)。
	
	对此非齐次传输方程,已知 \(u(x, 0) = \varphi(x)\),用公式\eqref{eq:非齐次解1}得到
	\begin{equation}\label{eq:达朗贝尔公式}
		\begin{aligned}
			u(x, t) &= \varphi(x + a t) + \int_0^t \left[ \psi(x - 2 a s + a t) - a \varphi'(x - 2 a s + a t) \right] \mathrm{d}s \\
			&= \varphi(x + a t) + \frac{1}{2 a} \int_{x - a t}^{x + a t} \left[ \psi(y) - a \varphi'(y) \right] \mathrm{d}y \\
			&= \frac{1}{2} \left[ \varphi(x + a t) + \varphi(x - a t) \right] + \frac{1}{2 a} \int_{x - a t}^{x + a t} \psi(y) \mathrm{d}y.
		\end{aligned}
	\end{equation}
	称此式为 d'Alembert (达朗贝尔) 公式.
	
	\section{一维波动方程的初边值问题}
	\subsection{反射法}
	反射法的核心思想:利用达朗贝尔公式把解延拓
	
	\subsubsection{问题描述}
	求解半直线 \(\mathbb{R}_+ = \{x > 0\}\) 上的初边值问题:
	
	\begin{equation}
		\begin{cases}
			u_{tt} - u_{xx} = 0, & x \in \mathbb{R}_+, t > 0, \\
			u(x, 0) = g(x), \quad u_t(x, 0) = h(x), & x \in \mathbb{R}_+, \\
			u(0, t) = 0, & t \geq 0,
		\end{cases}
	\end{equation}
	
	其中,\(g, h\) 是已知函数,满足 \(g(0) = h(0) = 0\)。
	
	\subsubsection{求解}
先把问题转换到全空间 \(\mathbb{R}\) 上去。为此,对函数 \(u, g, h\) 作奇延拓(或称奇反射)如下:
	
	\begin{equation}
		\bar{u}(x, t) = \begin{cases}
			u(x, t), & x \geq 0, t \geq 0, \\
			-u(-x, t), & x \leq 0, t \geq 0,
		\end{cases}
	\end{equation}
	
	\begin{equation}
		\bar{g}(x) = \begin{cases}
			g(x), & x \geq 0, \\
			-g(-x), & x \leq 0,
		\end{cases}
	\end{equation}
	
	\begin{equation}
		\bar{h}(x) = \begin{cases}
			h(x), & x \geq 0, \\
			-h(-x), & x \leq 0.
		\end{cases}
	\end{equation}
	
	则 \(\bar{u}(x, t)\) 满足问题:
	
	\begin{equation}
		\begin{cases}
			\bar{u}_{tt} - \bar{u}_{xx} = 0, & (x, t) \in \mathbb{R} \times (0, \infty), \\
			\bar{u}(x, 0) = \bar{g}(x), \quad \bar{u}_t(x, 0) = \bar{h}(x), & x \in \mathbb{R}.
		\end{cases}
	\end{equation}
	
	\begin{remark}
		还可以用特征线法对问题 (3.1.1) 求解,即用初值问题中方程的特征线作自变量的变换,把方程化为双曲型的第二标准型 \(u_{\xi\eta} = 0\) 的形式,对它积分两次求出通解 \(u = F(\xi) + G(\eta)\),其中,\(F\) 和 \(G\) 是任意二次光滑函数。然后利用初值条件确定通解中的两个任意函数,便得 d'Alembert 公式。
	\end{remark}
	
	
	\subsection{分离变量法}
	\subsubsection{问题描述}
	\begin{equation} \label{eq:wave_equation}
		\frac{\partial^2 u}{\partial t^2} = c^2 \frac{\partial^2 u}{\partial x^2} \qquad 0 < x < l, \quad t > 0
	\end{equation}
	
	边界条件:
	\begin{equation} \label{eq:boundary_conditions}
		u(0, t) = 0 \quad u(l, t) = 0 \qquad \forall t > 0
	\end{equation}
	
	初始条件:
	\begin{equation} \label{eq:initial_conditions}
		\begin{aligned}
			u(x, 0) &= f(x) \\
			\frac{\partial u}{\partial t}(x, 0) &= g(x) \qquad 0 < x < l
		\end{aligned}
	\end{equation}
	
	\subsubsection{核心思想}
	核心思想:分离变量法把偏微分转成为两个常微分。
	
	设 \(u(x, t) = X(x) \cdot T(t)\),假设解为乘积解。
	
	代入方程:
	\begin{equation} \label{eq:substitution}
		\frac{\partial^2 u}{\partial t^2} = X \cdot T'' \qquad \frac{\partial^2 u}{\partial x^2} = X'' \cdot T
	\end{equation}
	
	代入原方程:
	\begin{equation} \label{eq:original_substitution}
		X \cdot T'' = c^2 \cdot X'' \cdot T
	\end{equation}
	
	转化为可分离变量方程:
	\begin{equation} \label{eq:separation}
		\frac{T''}{c^2 T} = \frac{X''}{X}
	\end{equation}
	
	两个线性无关的变量相等,只能同为常数:
	\begin{equation} \label{eq:constant}
		\frac{T''}{c^2 T} = \frac{X''}{X} = k
	\end{equation}
	
	转化为两个常微分方程:
	\begin{equation} \label{eq:ode}
		\begin{cases}
			X'' = kX \\
			T'' = k c^2 T
		\end{cases}
	\end{equation}
	
	\subsubsection{空间常微分方程的求解}
	\begin{equation}
		X'' - kX = 0 \quad X(0) = 0 \quad (X(l) = 0
	\end{equation}
	
情况 1 \quad 若 \(k > 0\)

通解为 \(X(x) = C_1 \cdot \cosh \mu x + C_2 \cdot \sinh \mu x\),其中 \(k = \mu^2\)
	
	代入初始条件 
	\begin{equation}
		X(0) = C_1 = 0 \quad X(l) = C_2 \cdot \sinh \mu l = 0 \quad \therefore C_2 = 0
	\end{equation}
	
	
	\begin{verification}	
		\begin{equation}
			\cosh x = \frac{e^x + e^{-x}}{2} \quad \text{双曲余弦} \quad \sinh x = \frac{e^x - e^{-x}}{2} \quad \text{双曲正弦}
		\end{equation}
		
		\begin{equation}
			e^{ix} = \cos x + i \sin x \quad e^{-ix} = \cos x - i \sin x
		\end{equation}
		
		\begin{equation}
			\therefore \cos x = \frac{e^{ix} + e^{-ix}}{2} \quad \sin x = \frac{e^{ix} - e^{-ix}}{2i}
		\end{equation}
		
		\begin{equation}
			(\cosh x)' = \left( \frac{e^x + e^{-x}}{2} \right)' = \frac{e^x - e^{-x}}{2} = \sinh x
		\end{equation}
		
		\begin{equation}
			(\sinh x)' = \left( \frac{e^x - e^{-x}}{2} \right)' = \frac{e^x + e^{-x}}{2} = \cosh x
		\end{equation}
		
		\begin{equation}
			\therefore X = C_1 \cdot u \cdot \sinh \mu x + C_2 \cdot u \cdot \cosh \mu x
		\end{equation}
		
		\begin{equation}
			X'' = C_1 \cdot \mu^2 \cdot \cosh \mu x + C_2 \cdot \mu^2 \cdot \sinh \mu x
		\end{equation}
		
		\begin{equation}
			X'' - kX = 0 \quad \therefore k = \mu^2
		\end{equation}
		
	\end{verification}	
	
	
情况 2 \quad 若 \(k = 0\)

则 \(X'' = 0\)
	\begin{equation}
		X(x) = C_1 x + C_2 \quad \text{且} \quad X(0) = 0 \quad X(l) = 0
	\end{equation}
	\begin{equation}
		\therefore C_1 = C_2 = 0
	\end{equation}
	
情况 3 \quad 若 \(k < 0\)

即 \(X'' + \mu^2 X = 0\) \quad \(X(0) = 0\) \quad \(X(l) = 0\) \quad \(k = -\mu^2\)
	
	通解:
	\begin{equation}
		X = C_1 \cos \mu x + C_2 \sin \mu x
	\end{equation}
	
	
	
	边界条件:
	\begin{equation}
		X(0) = C_1 = 0 \quad X(l) = C_2 \sin \mu l = 0
	\end{equation}
	
	非平凡解要求:
	\begin{equation}
		\sin \mu l = 0 \quad \therefore \mu l = n\pi \quad n \text{ 为任意正整数}
	\end{equation}
	
	特征值:
	\begin{equation}
		\mu_n = \frac{n\pi}{l}
	\end{equation}
	
	特征函数:
	\begin{equation}
		X_n = C_2 \sin \frac{n\pi}{l} x \quad n = 1, 2, 3, \ldots \quad \text{(C₂吸收正负号)}
	\end{equation}
	
	特征值:
	\begin{equation}
		k = -\mu^2 = -\left(\frac{n\pi}{l}\right)^2
	\end{equation}
	
	\begin{verification}	
		一阶导数:
		\begin{equation}
			X' = -C_1 \mu \sin \mu x + C_2 \mu \cos \mu x
		\end{equation}
		
		二阶导数:
		\begin{equation}
			X'' = -C_1 \mu^2 \cos \mu x - C_2 \mu^2 \sin \mu x
		\end{equation}
		
		满足方程:
		\begin{equation}
			X'' + \mu^2 X = 0
		\end{equation}
	\end{verification}	
	
	
	\subsubsection{时间常微分方程的求解}
	\(T'' + \left(c \cdot \frac{n\pi}{l}\right)^2 \cdot T = 0 \implies T'' + (c \mu_n)^2 T = 0\),其中 \(\lambda_n = c \mu_n = \frac{c n \pi}{l}\)
	
	同理可得通解:
	\begin{equation}
		T = C_3 \cos \lambda_n t + C_4 \sin \lambda_n t
	\end{equation}
	
	\subsubsection{得偏微分方程通解}
	因此:
	\begin{equation}
		u_n(x, t) = X \cdot T = \sin \frac{n\pi}{l} x \cdot (a_n \cos \lambda_n t + b_n \sin \lambda_n t)
	\end{equation}
	
	由于方程为线性齐次,故可用叠加原理:
	\begin{equation}
		u(x, t) = \sum_{n=1}^{\infty} \sin \frac{n\pi}{l} x \cdot (a_n \cos \lambda_n t + b_n \sin \lambda_n t)
	\end{equation}
	
\subsubsection{初始条件求系数}
原函数初始条件求$a_n$
	
	\begin{equation}
		u(x, 0) = f(x) \quad \frac{\partial u}{\partial t}(x, 0) = g(x)
	\end{equation}
	
	由初始条件:
	\begin{equation}
		u(x, 0) = \sum_{n=1}^{\infty} \sin \frac{n\pi}{l} x \cdot a_n = f(x)
	\end{equation}
	
	利用内积公式(需要$f \in L^2$):
	\begin{equation}
		a_n = \frac{\langle f(x), \sin \frac{n\pi}{l} x \rangle}{\langle \sin \frac{n\pi}{l} x, \sin \frac{n\pi}{l} x \rangle} = \frac{\int_0^l f(x) \cdot \sin \frac{n\pi}{l} x \, dx}{\int_0^l \sin^2 \frac{n\pi}{l} x \, dx}
	\end{equation}
	
	化简得:
	\begin{equation}
		a_n = \frac{2}{l} \cdot \int_0^l f(x) \cdot \sin \frac{n\pi}{l} x \, dx
	\end{equation}
	
	偏导初始条件求$b_n$
	
	对 \(u_n\) 求偏导:
	\begin{equation}
		\frac{\partial u_n}{\partial t}(x, t) = \sin \frac{n\pi}{l} x \cdot \left( -a_n \lambda_n \sin \lambda_n t + b_n \lambda_n \cos \lambda_n t \right)
	\end{equation}
	
	在 \(t = 0\) 时:
	\begin{equation}
		\frac{\partial u_n}{\partial t}(x, 0) = \sin \frac{n\pi}{l} x \cdot b_n \lambda_n
	\end{equation}
	
	对总解求偏导:
	\begin{equation}
		\frac{\partial u}{\partial t}(x, 0) = \sum_{n=1}^{\infty} \frac{\partial u_n}{\partial t}(x, 0) = \sum_{n=1}^{\infty} b_n \lambda_n \sin \frac{n\pi}{l} x = g(x)
	\end{equation}
	
	利用内积公式(需要$f \in L^2$):
	\begin{equation}
		b_n \lambda_n = \frac{\langle g(x), \sin \frac{n\pi}{l} x \rangle}{\langle \sin \frac{n\pi}{l} x, \sin \frac{n\pi}{l} x \rangle} = \frac{2}{l} \int_0^l g(x) \cdot \sin \frac{n\pi}{l} x \, dx
	\end{equation}
	
	化简得:
	\begin{equation}
		b_n = \frac{2}{l \lambda_n} \cdot \int_0^l g(x) \cdot \sin \frac{n\pi}{l} x \, dx = \frac{2}{c n \pi} \int_0^l g(x) \cdot \sin \frac{n\pi}{l} x \, dx
	\end{equation}
	
	\subsubsection{用数分知识求系数,条件和前面泛函内积不一样}
	
	考虑函数 \( f(t) \) 的傅里叶级数展开:
	\begin{equation}
		f = \frac{a_0}{2} + \sum_{n=1}^{\infty} \left( a_n \cos nt + b_n \sin nt \right)
	\end{equation}
	
	计算 \( a_0 \):
	\begin{equation}
		\frac{a_0}{2} = f - \sum_{n=1}^{\infty} \left( a_n \cos nt + b_n \sin nt \right)
	\end{equation}
	
	\begin{equation}
		a_0 = 2f - 2 \sum_{n=1}^{\infty} \left( a_n \cos nt + b_n \sin nt \right)
	\end{equation}
	
	对 \( a_0 \) 积分,若积分和求和可换序:
	\begin{equation}
		\frac{1}{2\pi} \int_{-\pi}^{\pi} a_0 \, dt = \frac{1}{\pi} \int_{-\pi}^{\pi} f \, dt - \sum_{n=1}^{\infty} \frac{1}{\pi} a_n \int_{-\pi}^{\pi} \cos nt \, dt - \sum_{n=1}^{\infty} \frac{1}{\pi} b_n \int_{-\pi}^{\pi} \sin nt \, dt
	\end{equation}
	
	化简得:
	\begin{equation}
		a_0 = \frac{1}{\pi} \int_{-\pi}^{\pi} f \, dt
	\end{equation}
	
	计算 \( a_n \):
	\begin{equation}
		f \cos nt = \frac{a_0}{2} \cos nt + \sum_{k=1}^{\infty} \left( a_k \cos kt + b_k \sin kt \right) \cos nt
	\end{equation}
	
	积分得,若积分和求和可换序:
	\begin{equation}
		\int_{-\pi}^{\pi} f \cos nt \, dt = \int_{-\pi}^{\pi} \frac{a_0}{2} \cos nt \, dt + \sum_{k=1}^{\infty} \left( a_k \int_{-\pi}^{\pi} \cos kt \cos nt \, dt + b_k \int_{-\pi}^{\pi} \sin kt \cos nt \, dt \right)
	\end{equation}
	
	化简得:
	\begin{equation}
		\int_{-\pi}^{\pi} f \cos nt \, dt = a_n \pi
	\end{equation}
	
	因此:
	\begin{equation}
		a_n = \frac{1}{\pi} \int_{-\pi}^{\pi} f \cos nt \, dt
	\end{equation}
	
	同理可得:
	\begin{equation}
		b_n = \frac{1}{\pi} \int_{-\pi}^{\pi} f \sin nt \, dt
	\end{equation}
	
	级数收敛性:
	\begin{equation}
		\sum_{n=1}^{\infty} a_n \cos nx < \infty \qquad \sum_{n=1}^{\infty} b_n \sin nx < \infty
	\end{equation}
	详细条件可以去看我的傅里叶分析笔记。
	
	\subsubsection{总结}
	
	一维波动方程:
	\begin{equation}
		\frac{\partial^2 u}{\partial t^2} = c^2 \cdot \frac{\partial^2 u}{\partial x^2} \qquad 0 < x < l, \quad t > 0
	\end{equation}
	
	边界条件:
	\begin{equation}
		u(0, t) = 0 \quad u(l, t) = 0 \qquad \forall t > 0
	\end{equation}
	
	初始条件:
	\begin{equation}
		u(x, 0) = f(x) \quad \frac{\partial u}{\partial t}(x, 0) = g(x) \qquad 0 < x < l
	\end{equation}
	
	解为:
	\begin{equation}
		u(x, t) = \sum_{n=1}^{\infty} \sin \frac{n\pi}{l} x \cdot \left( a_n \cos \lambda_n t + b_n \sin \lambda_n t \right)
	\end{equation}
	
	其中:
	\begin{equation}
		a_n = \frac{2}{l} \int_0^l f(x) \cdot \sin \frac{n\pi}{l} x \, dx
	\end{equation}
	
	\begin{equation}
		b_n = \frac{2}{c n \pi} \int_0^l g(x) \cdot \sin \frac{n\pi}{l} x \, dx
	\end{equation}
	
	\begin{equation}
		\lambda_n = c \mu_n = \frac{c n \pi}{l}
	\end{equation}
	
	
	
\end{document}