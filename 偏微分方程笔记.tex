\documentclass[12pt,a4paper]{article}
\usepackage[UTF8]{ctex}
\usepackage{geometry}
\usepackage{amsmath,amssymb,amsthm}
\usepackage{mathtools,bm}
\usepackage{empheq}
\usepackage{graphicx}
\usepackage{booktabs}
\usepackage[numbers,sort&compress]{natbib}
\usepackage{caption}
\usepackage{enumitem}
\usepackage{chngcntr}

% ========== 页面布局 ==========
\geometry{left=2.5cm,right=2.5cm,top=2.5cm,bottom=2.5cm}
\setlength{\parskip}{0.5em}
\renewcommand{\baselinestretch}{1.2}

% ========== 数学命令 ==========
\newcommand{\diff}{\mathop{}\!\mathrm{d}}
\newcommand{\R}{\mathbb{R}}
\newcommand{\C}{\mathbb{C}}
\newcommand{\Z}{\mathbb{Z}}
\newcommand{\N}{\mathbb{N}}
\DeclareMathOperator{\supp}{supp}

% ========== 编号系统 ==========
\numberwithin{subsection}{section}   % 子小节按章节编号
\numberwithin{subsubsection}{subsection}
\counterwithin{equation}{subsection} % 公式按子小节编号

% ========== 定理环境 ==========
\theoremstyle{plain}
\newtheorem{theorem}{定理}[section]
\newtheorem{lemma}[theorem]{引理}
\newtheorem{proposition}[theorem]{命题}
\newtheorem{corollary}[theorem]{推论}
\newtheorem{solution}{解}[subsection]  % 解按子小节编号

\theoremstyle{definition}
\newtheorem{definition}[theorem]{定义}
\newtheorem{example}{示例}[subsection]  % 示例按子小节编号

\theoremstyle{remark}
\newtheorem{remark}[theorem]{注记}

\theoremstyle{verification}
\newtheorem{verification}[theorem]{检验}

\title{偏微分方程笔记}
\author{陈柏均}
\date{2025年5月12日}

\begin{document}
	
	\maketitle
	
	\tableofcontents  % 添加目录
	
	\section{一阶拟线性方程之齐次传输方程} 
	\subsection{变量替换求解常系数齐次传输方程} 
	
	
	
	假设 $a_1 \neq 0$ 且 $a_2 \neq 0$,我们求解常系数传输方程:
	\begin{equation} \label{eq:pde_original}
		a_1 \frac{\partial u}{\partial t} + a_2 \frac{\partial u}{\partial x} = 0
	\end{equation}
	
	\subsubsection{通解} 
	核心思想:通过变量替换,把二元偏微分转化成一元的常微分求解。
	
	其中 $u = u(t,x)$。引入坐标变换 $(\alpha, \beta)$,使得 $u = u(\alpha, \beta)$,且:
	\begin{equation} \label{eq:coordinate_transform}
		\begin{cases}
			\alpha = ax + bt, \\
			\beta = cx + dt.
		\end{cases}
	\end{equation}
	利用链式法则计算偏导数:
	\begin{align}
		\frac{\partial u}{\partial t} 
		&= \frac{\partial u}{\partial \alpha} \frac{\partial \alpha}{\partial t} + \frac{\partial u}{\partial \beta} \frac{\partial \beta}{\partial t} 
		= b\frac{\partial u}{\partial \alpha} + d\frac{\partial u}{\partial \beta}, \label{eq:u_t_chain_rule} \\
		\frac{\partial u}{\partial x} 
		&= \frac{\partial u}{\partial \alpha} \frac{\partial \alpha}{\partial x} + \frac{\partial u}{\partial \beta} \frac{\partial \beta}{\partial x} 
		= a\frac{\partial u}{\partial \alpha} + c\frac{\partial u}{\partial \beta}. \label{eq:u_x_chain_rule}
	\end{align}
	
	将 \eqref{eq:u_t_chain_rule} 和 \eqref{eq:u_x_chain_rule} 代入原方程 \eqref{eq:pde_original}:
	\begin{equation} \label{eq:pde_transformed}
		a_1 \left( b \frac{\partial u}{\partial \alpha} + d \frac{\partial u}{\partial \beta} \right) + a_2 \left( a \frac{\partial u}{\partial \alpha} + c \frac{\partial u}{\partial \beta} \right) = 0.
	\end{equation}
	整理后得到:
	\begin{equation} \label{eq:pde_collected}
		(a_1 b + a_2 a) \frac{\partial u}{\partial \alpha} + (a_1 d + a_2 c) \frac{\partial u}{\partial \beta} = 0.
	\end{equation}
	为消去一个变量,pde转ode,选择让第二项系数为0,把方程 \eqref{eq:pde_collected} 简化为:
	\begin{equation} \label{eq:pde_final}
		\frac{\partial u}{\partial \alpha} = 0.
	\end{equation}
	选择系数
	\begin{equation} \label{eq:constant_choice}
		\begin{cases}
			a = 0, & b = 1, \\
			c = a_1, & d = -a_2.
		\end{cases}
	\end{equation}
	此时坐标变换为:
	\begin{equation} \label{eq:coordinates_specific}
		\begin{cases}
			\alpha = t, \\
			\beta = a_1 x - a_2 t.
		\end{cases}
	\end{equation}
	由\eqref{eq:pde_final} 表明 $u$ 仅依赖于 $\beta$,即通解为:
	\begin{equation} \label{eq:solution}
		u(t,x) = L(a_1 x - a_2 t),
	\end{equation}
	其中
	$L(\cdot)$ 是任意可微函数。
	\subsubsection{特解(初始条件或边界条件)} 
	已知初始条件$	u(x, 0) = e^{-x^2}$,求下面常系数运输方程:
	\begin{equation} 
		\frac{\partial u}{\partial t} +  \frac{\partial u}{\partial x} = 0
	\end{equation}
	
	由\eqref{eq:solution}可知
	\begin{equation}
		u(x, t) = f(x - t)= e^{-(x-t)^2}
	\end{equation}
	
	
	\subsection{特征线法求解变系数齐次传输方程} 
	\subsubsection{通解} 
	一阶线性变系数偏微分方程如下:
	\begin{equation}\label{eq:pde_original2}
		\frac{\partial u}{\partial x} + p(x,y) \frac{\partial u}{\partial y}=(1, p(x, y)) \cdot \left( \frac{\partial u}{\partial x}, \frac{\partial u}{\partial y} \right) = 0 
	\end{equation}
	其中 $p(x, y)$ 是 $x$ 和 $y$ 的函数。
	$\left( \frac{\partial u}{\partial x}, \frac{\partial u}{\partial y} \right)$为梯度,$(1, p(x, y))$为方向,一整个乘积为方向导数,方向导数为0意味着,$u(x, y)=C$在切向量为$(1, p(x, y))$这条曲线上,即
	\begin{equation}
		u(x,y)|_{\Gamma} = C
	\end{equation}
	\begin{equation}
		u(x,y) =  f(C)
	\end{equation}
	$\Gamma$曲线上,任意点$(x, y)$求导($\Gamma$曲线为$XOY$平面上的曲线,故$y$可表示成$x$的函数),可得切向量$(1,\frac{dy}{dx})$
	
	所以我们找到$\Gamma$曲线,把二元偏微分转化成一元的常微分,令
	
	\begin{equation}
		\frac{dy}{dx} = p(x, y)
	\end{equation}
	可解得
	\begin{equation}
		C=\phi(x,y)
	\end{equation}
	得方程解
	\begin{equation}
		u(x, y)=f(C)=f(\phi(x,y))
	\end{equation}
	$(1, \frac{dy}{dx})$
	为该曲线的切向量。我们称这条曲线叫特征线。只需要取遍所有的特征曲线就可以取遍$XOY$平面上所有的点,若有初始条件或者边界条件可以确定每条特征线在$u(x, y)$对应的取值,就可以完整确定$u(x, y)$这个函数。
	
	\begin{example}求解方程
		\begin{equation}
			\frac{\partial u}{\partial x} + x \frac{\partial u}{\partial y} = 0.
		\end{equation}
		
		此时我们有 $p(x, y) = x$,解 $\frac{dy}{dx} = x$,我们得到特征线 $y = \frac{1}{2}x^2 + C$,或 $y - \frac{1}{2}x^2 = C$。从而 $\phi(x, y) = y - \frac{1}{2}x^2$,偏微分方程的通解为 $u(x, y) = f(\phi(x, y))$,其中 $f$ 是任意函数。把它们代回方程,直接验证,便知是解。
	\end{example}
	
	
	
	\section{一维齐次波动方程之分离变量法}
	\subsection{问题描述}
	\begin{equation} \label{eq:wave_equation}
		\frac{\partial^2 u}{\partial t^2} = c^2 \frac{\partial^2 u}{\partial x^2} \qquad 0 < x < l, \quad t > 0
	\end{equation}
	
	边界条件:
	\begin{equation} \label{eq:boundary_conditions}
		u(0, t) = 0 \quad u(l, t) = 0 \qquad \forall t > 0
	\end{equation}
	
	初始条件:
	\begin{equation} \label{eq:initial_conditions}
		\begin{aligned}
			u(x, 0) &= f(x) \\
			\frac{\partial u}{\partial t}(x, 0) &= g(x) \qquad 0 < x < l
		\end{aligned}
	\end{equation}
	
	\subsection{分离变量法}
	核心思想:分离变量法把偏微分转成为两个常微分。
	
	设 \(u(x, t) = X(x) \cdot T(t)\),假设解为乘积解。
	
	代入方程:
	\begin{equation} \label{eq:substitution}
		\frac{\partial^2 u}{\partial t^2} = X \cdot T'' \qquad \frac{\partial^2 u}{\partial x^2} = X'' \cdot T
	\end{equation}
	
	代入原方程:
	\begin{equation} \label{eq:original_substitution}
		X \cdot T'' = c^2 \cdot X'' \cdot T
	\end{equation}
	
	转化为可分离变量方程:
	\begin{equation} \label{eq:separation}
		\frac{T''}{c^2 T} = \frac{X''}{X}
	\end{equation}
	
	两个线性无关的变量相等,只能同为常数:
	\begin{equation} \label{eq:constant}
		\frac{T''}{c^2 T} = \frac{X''}{X} = k
	\end{equation}
	
	转化为两个常微分方程:
	\begin{equation} \label{eq:ode}
		\begin{cases}
			X'' = kX \\
			T'' = k c^2 T
		\end{cases}
	\end{equation}
	
	\subsection{空间常微分方程的求解}
	\begin{equation}
		X'' - kX = 0 \quad X(0) = 0 \quad (X(l) = 0
	\end{equation}
	
	\subsubsection{情况 1}
	若 \(k > 0\),通解为 \(X(x) = C_1 \cdot \cosh \mu x + C_2 \cdot \sinh \mu x\),其中 \(k = \mu^2\)
	
	代入初始条件 
	\begin{equation}
		X(0) = C_1 = 0 \quad X(l) = C_2 \cdot \sinh \mu l = 0 \quad \therefore C_2 = 0
	\end{equation}
	
	
	\begin{verification}	
		\begin{equation}
			\cosh x = \frac{e^x + e^{-x}}{2} \quad \text{双曲余弦} \quad \sinh x = \frac{e^x - e^{-x}}{2} \quad \text{双曲正弦}
		\end{equation}
		
		\begin{equation}
			e^{ix} = \cos x + i \sin x \quad e^{-ix} = \cos x - i \sin x
		\end{equation}
		
		\begin{equation}
			\therefore \cos x = \frac{e^{ix} + e^{-ix}}{2} \quad \sin x = \frac{e^{ix} - e^{-ix}}{2i}
		\end{equation}
		
		\begin{equation}
			(\cosh x)' = \left( \frac{e^x + e^{-x}}{2} \right)' = \frac{e^x - e^{-x}}{2} = \sinh x
		\end{equation}
		
		\begin{equation}
			(\sinh x)' = \left( \frac{e^x - e^{-x}}{2} \right)' = \frac{e^x + e^{-x}}{2} = \cosh x
		\end{equation}
		
		\begin{equation}
			\therefore X = C_1 \cdot u \cdot \sinh \mu x + C_2 \cdot u \cdot \cosh \mu x
		\end{equation}
		
		\begin{equation}
			X'' = C_1 \cdot \mu^2 \cdot \cosh \mu x + C_2 \cdot \mu^2 \cdot \sinh \mu x
		\end{equation}
		
		\begin{equation}
			X'' - kX = 0 \quad \therefore k = \mu^2
		\end{equation}
		
	\end{verification}	
	
	
	\subsubsection{情况 2}
	若 \(k = 0\),则 \(X'' = 0\)
	\begin{equation}
		X(x) = C_1 x + C_2 \quad \text{且} \quad X(0) = 0 \quad X(l) = 0
	\end{equation}
	\begin{equation}
		\therefore C_1 = C_2 = 0
	\end{equation}
	
	\subsubsection{情况 3} 
	若 \(k < 0\),即 \(X'' + \mu^2 X = 0\) \quad \(X(0) = 0\) \quad \(X(l) = 0\) \quad \(k = -\mu^2\)
	
	通解:
	\begin{equation}
		X = C_1 \cos \mu x + C_2 \sin \mu x
	\end{equation}
	
	
	
	边界条件:
	\begin{equation}
		X(0) = C_1 = 0 \quad X(l) = C_2 \sin \mu l = 0
	\end{equation}
	
	非平凡解要求:
	\begin{equation}
		\sin \mu l = 0 \quad \therefore \mu l = n\pi \quad n \text{ 为任意正整数}
	\end{equation}
	
	特征值:
	\begin{equation}
		\mu_n = \frac{n\pi}{l}
	\end{equation}
	
	特征函数:
	\begin{equation}
		X_n = C_2 \sin \frac{n\pi}{l} x \quad n = 1, 2, 3, \ldots \quad \text{(C₂吸收正负号)}
	\end{equation}
	
	特征值:
	\begin{equation}
		k = -\mu^2 = -\left(\frac{n\pi}{l}\right)^2
	\end{equation}
	
	\begin{verification}	
		一阶导数:
		\begin{equation}
			X' = -C_1 \mu \sin \mu x + C_2 \mu \cos \mu x
		\end{equation}
		
		二阶导数:
		\begin{equation}
			X'' = -C_1 \mu^2 \cos \mu x - C_2 \mu^2 \sin \mu x
		\end{equation}
		
		满足方程:
		\begin{equation}
			X'' + \mu^2 X = 0
		\end{equation}
	\end{verification}	
	
	
	\subsection{时间常微分方程的求解}
	\(T'' + \left(c \cdot \frac{n\pi}{l}\right)^2 \cdot T = 0 \implies T'' + (c \mu_n)^2 T = 0\),其中 \(\lambda_n = c \mu_n = \frac{c n \pi}{l}\)
	
	同理可得通解:
	\begin{equation}
		T = C_3 \cos \lambda_n t + C_4 \sin \lambda_n t
	\end{equation}
	
	\subsection{得偏微分方程通解}
	因此:
	\begin{equation}
		u_n(x, t) = X \cdot T = \sin \frac{n\pi}{l} x \cdot (a_n \cos \lambda_n t + b_n \sin \lambda_n t)
	\end{equation}
	
	由于方程为线性齐次,故可用叠加原理:
	\begin{equation}
		u(x, t) = \sum_{n=1}^{\infty} \sin \frac{n\pi}{l} x \cdot (a_n \cos \lambda_n t + b_n \sin \lambda_n t)
	\end{equation}
	
\subsection{初始条件求系数}
	\subsubsection{原函数初始条件求$a_n$}
	
	\begin{equation}
		u(x, 0) = f(x) \quad \frac{\partial u}{\partial t}(x, 0) = g(x)
	\end{equation}
	
	由初始条件:
	\begin{equation}
		u(x, 0) = \sum_{n=1}^{\infty} \sin \frac{n\pi}{l} x \cdot a_n = f(x)
	\end{equation}
	
	利用内积公式(需要$f \in L^2$):
	\begin{equation}
		a_n = \frac{\langle f(x), \sin \frac{n\pi}{l} x \rangle}{\langle \sin \frac{n\pi}{l} x, \sin \frac{n\pi}{l} x \rangle} = \frac{\int_0^l f(x) \cdot \sin \frac{n\pi}{l} x \, dx}{\int_0^l \sin^2 \frac{n\pi}{l} x \, dx}
	\end{equation}
	
	化简得:
	\begin{equation}
		a_n = \frac{2}{l} \cdot \int_0^l f(x) \cdot \sin \frac{n\pi}{l} x \, dx
	\end{equation}
	
	\subsubsection{偏导初始条件求$b_n$}
	
	对 \(u_n\) 求偏导:
	\begin{equation}
		\frac{\partial u_n}{\partial t}(x, t) = \sin \frac{n\pi}{l} x \cdot \left( -a_n \lambda_n \sin \lambda_n t + b_n \lambda_n \cos \lambda_n t \right)
	\end{equation}
	
	在 \(t = 0\) 时:
	\begin{equation}
		\frac{\partial u_n}{\partial t}(x, 0) = \sin \frac{n\pi}{l} x \cdot b_n \lambda_n
	\end{equation}
	
	对总解求偏导:
	\begin{equation}
		\frac{\partial u}{\partial t}(x, 0) = \sum_{n=1}^{\infty} \frac{\partial u_n}{\partial t}(x, 0) = \sum_{n=1}^{\infty} b_n \lambda_n \sin \frac{n\pi}{l} x = g(x)
	\end{equation}
	
	利用内积公式(需要$f \in L^2$):
	\begin{equation}
		b_n \lambda_n = \frac{\langle g(x), \sin \frac{n\pi}{l} x \rangle}{\langle \sin \frac{n\pi}{l} x, \sin \frac{n\pi}{l} x \rangle} = \frac{2}{l} \int_0^l g(x) \cdot \sin \frac{n\pi}{l} x \, dx
	\end{equation}
	
	化简得:
	\begin{equation}
		b_n = \frac{2}{l \lambda_n} \cdot \int_0^l g(x) \cdot \sin \frac{n\pi}{l} x \, dx = \frac{2}{c n \pi} \int_0^l g(x) \cdot \sin \frac{n\pi}{l} x \, dx
	\end{equation}
	
	\subsubsection{用数分知识求系数,条件和前面泛函内积不一样}
	
	考虑函数 \( f(t) \) 的傅里叶级数展开:
	\begin{equation}
		f = \frac{a_0}{2} + \sum_{n=1}^{\infty} \left( a_n \cos nt + b_n \sin nt \right)
	\end{equation}
	
	计算 \( a_0 \):
	\begin{equation}
		\frac{a_0}{2} = f - \sum_{n=1}^{\infty} \left( a_n \cos nt + b_n \sin nt \right)
	\end{equation}
	
	\begin{equation}
		a_0 = 2f - 2 \sum_{n=1}^{\infty} \left( a_n \cos nt + b_n \sin nt \right)
	\end{equation}
	
	对 \( a_0 \) 积分,若积分和求和可换序:
	\begin{equation}
		\frac{1}{2\pi} \int_{-\pi}^{\pi} a_0 \, dt = \frac{1}{\pi} \int_{-\pi}^{\pi} f \, dt - \sum_{n=1}^{\infty} \frac{1}{\pi} a_n \int_{-\pi}^{\pi} \cos nt \, dt - \sum_{n=1}^{\infty} \frac{1}{\pi} b_n \int_{-\pi}^{\pi} \sin nt \, dt
	\end{equation}
	
	化简得:
	\begin{equation}
		a_0 = \frac{1}{\pi} \int_{-\pi}^{\pi} f \, dt
	\end{equation}
	
	计算 \( a_n \):
	\begin{equation}
		f \cos nt = \frac{a_0}{2} \cos nt + \sum_{k=1}^{\infty} \left( a_k \cos kt + b_k \sin kt \right) \cos nt
	\end{equation}
	
	积分得,若积分和求和可换序:
	\begin{equation}
		\int_{-\pi}^{\pi} f \cos nt \, dt = \int_{-\pi}^{\pi} \frac{a_0}{2} \cos nt \, dt + \sum_{k=1}^{\infty} \left( a_k \int_{-\pi}^{\pi} \cos kt \cos nt \, dt + b_k \int_{-\pi}^{\pi} \sin kt \cos nt \, dt \right)
	\end{equation}
	
	化简得:
	\begin{equation}
		\int_{-\pi}^{\pi} f \cos nt \, dt = a_n \pi
	\end{equation}
	
	因此:
	\begin{equation}
		a_n = \frac{1}{\pi} \int_{-\pi}^{\pi} f \cos nt \, dt
	\end{equation}
	
	同理可得:
	\begin{equation}
		b_n = \frac{1}{\pi} \int_{-\pi}^{\pi} f \sin nt \, dt
	\end{equation}
	
	级数收敛性:
	\begin{equation}
		\sum_{n=1}^{\infty} a_n \cos nx < \infty \qquad \sum_{n=1}^{\infty} b_n \sin nx < \infty
	\end{equation}
	详细条件可以去看我的傅里叶分析笔记。
	
	\subsection{总结}
	
	一维波动方程:
	\begin{equation}
		\frac{\partial^2 u}{\partial t^2} = c^2 \cdot \frac{\partial^2 u}{\partial x^2} \qquad 0 < x < l, \quad t > 0
	\end{equation}
	
	边界条件:
	\begin{equation}
		u(0, t) = 0 \quad u(l, t) = 0 \qquad \forall t > 0
	\end{equation}
	
	初始条件:
	\begin{equation}
		u(x, 0) = f(x) \quad \frac{\partial u}{\partial t}(x, 0) = g(x) \qquad 0 < x < l
	\end{equation}
	
	解为:
	\begin{equation}
		u(x, t) = \sum_{n=1}^{\infty} \sin \frac{n\pi}{l} x \cdot \left( a_n \cos \lambda_n t + b_n \sin \lambda_n t \right)
	\end{equation}
	
	其中:
	\begin{equation}
		a_n = \frac{2}{l} \int_0^l f(x) \cdot \sin \frac{n\pi}{l} x \, dx
	\end{equation}
	
	\begin{equation}
		b_n = \frac{2}{c n \pi} \int_0^l g(x) \cdot \sin \frac{n\pi}{l} x \, dx
	\end{equation}
	
	\begin{equation}
		\lambda_n = c \mu_n = \frac{c n \pi}{l}
	\end{equation}
	
	
	
\end{document}