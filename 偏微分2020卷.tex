\documentclass[12pt,a4paper]{article}
\usepackage[UTF8]{ctex}                    % 中文支持
\usepackage{geometry}                      % 页面布局
\usepackage{amsmath,amssymb,amsthm}        % 数学符号与定理
\usepackage{mathtools,bm}                  % 高级数学工具
\usepackage{graphicx}                      % 图片插入
\usepackage{hyperref}                      % 超链接
\hypersetup{
	pdfborder = {0 0 0}  % 关闭链接边框
}
\usepackage{cleveref}
\usepackage{booktabs}                      % 三线表
\usepackage[numbers,sort&compress]{natbib} % 参考文献
\usepackage{caption}                       % 图表标题
\usepackage[shortlabels]{enumitem}         % 列表环境

% ========== 页面布局 ==========
\geometry{left=2cm, right=2cm, top=2.5cm, bottom=2.5cm}
\setlength{\parskip}{0.5em}                % 段落间距
\renewcommand{\baselinestretch}{1.3}       % 行距

% ========== 数学命令 ==========
\newcommand{\diff}{\mathop{}\!\mathrm{d}}  % 微分符号
\newcommand{\R}{\mathbb{R}}                % 实数集
\newcommand{\C}{\mathbb{C}}                % 复数集
\newcommand{\Z}{\mathbb{Z}}                % 整数集
\newcommand{\Lspace}{L^2(-l,l)}            % L²空间

% ========== 定理环境 ==========
% 定义题目环境
\newtheorem{problem}{题}
% 定义解题环境
\newtheorem*{solution}{解}

\newtheorem{example}{例题}
\newtheorem{corollary}{推论}
\newtheorem{proposition}{命题}
\newtheorem{lemma}{引理}

% ========== 文档信息 ==========
\begin{document}
	
	\begin{center}
				\LARGE 偏微分方程2020卷 \\
		\vspace{0.5cm}
		\large 班级:22数学1 \quad 姓名:陈柏均 \quad 学号:202225110102
	\end{center}
	
	% ========================= 基础题 =========================
	\section*{一、基础题 (本大题共5小题, 每小题4分, 共20分)}
	
	\begin{problem}
		指出方程 $u_t + u_{xxxx} + \sqrt{1+u^2} = 0$ 的阶,并判定它是线性的还是非线性的。
	\end{problem}
	\hrulefill
	\begin{solution}
		\textbf{阶数:}方程中最高阶导数是 $u_{xxxx}$,所以这是一个四阶非偏微分方程。
		
		\noindent
		\textbf{线性性:}该方程是非线性的。因为出现了关于未知函数 $u$ 的非线性项 $\sqrt{1+u^2}$。
	\end{solution}
	\hrulefill
	
	\begin{problem}
		判别方程 $u_{xx} + u_{xy} + u_{yy} + u_x = 0$ 的类型。
	\end{problem}
	\hrulefill
	\begin{solution}
		该方程是一个二阶线性偏微分方程。其类型由主部 $u_{xx} + u_{xy} + u_{yy}$ 的系数决定。
		系数为 $A=1, B=1, C=1$。计算判别式:
		\[
		\Delta = B^2 - 4AC = 1^2 - 4(1)(1) = 1 - 4 = -3 < 0
		\]
		因为 $\Delta < 0$,所以该方程为椭圆型方程。
	\end{solution}
	\hrulefill
	
	\begin{problem}
		验证 $u(x, y) = x^3 - 3xy^2$ 和 $v(x, y) = 3x^2y - y^3$ 是调和的。
	\end{problem}
	\hrulefill
	\begin{solution}
		一个函数是调和的,如果它满足拉普拉斯方程 $\Delta f = f_{xx} + f_{yy} = 0$。
		
		\noindent
		\textbf{对于 $u(x,y)$:}
		\begin{align*}
			u_x &= 3x^2 - 3y^2, & u_{xx} &= 6x \\
			u_y &= -6xy, & u_{yy} &= -6x
		\end{align*}
		因此, $\Delta u = u_{xx} + u_{yy} = 6x - 6x = 0$。所以 $u(x,y)$ 是调和函数。
		
		\noindent
		\textbf{对于 $v(x,y)$:}
		\begin{align*}
			v_x &= 6xy, & v_{xx} &= 6y \\
			v_y &= 3x^2 - 3y^2, & v_{yy} &= -6y
		\end{align*}
		因此, $\Delta v = v_{xx} + v_{yy} = 6y - 6y = 0$。所以 $v(x,y)$ 是调和函数。
	\end{solution}
	\hrulefill
	
	\begin{problem}
		对 Cauchy 问题
		$
		\begin{cases}
			u_{tt} - 9u_{xx} = 0, & -\infty < x < \infty, \ t > 0, \\
			u(x, 0) = \phi(x), \quad u_t(x, 0) = \psi(x), & -\infty < x < \infty,
		\end{cases}
		$
		求关于点 $(3,1)$ 的依赖区域。
	\end{problem}
	\hrulefill
	\begin{solution}
		这是一个一维波动方程,其波速 $c$ 满足 $c^2=9$,所以 $c=3$。
		点 $(x_0, t_0) = (3, 1)$ 的依赖区域是初始直线 $t=0$ 上的一个区间 $[x_0 - ct_0, x_0 + ct_0]$。
		代入数值:
		\[
		[3 - 3 \cdot 1, \quad 3 + 3 \cdot 1] = [0, 6]
		\]
		所以点 $(3,1)$ 的依赖区域是初始轴上的闭区间 $[0, 6]$。
	\end{solution}
	\hrulefill
	
	\begin{problem}
		设 $u,v$ 是 $\Omega \subset \R^N$ 上的连续且具有连续的二阶偏导数,在 $\overline{\Omega}$ 上具有连续的一阶偏导数,则写出第二 Green 公式。
	\end{problem}
	\hrulefill
	\begin{solution}
		第二 Green 公式为:
		\[
		\int_{\Omega} (u \Delta v - v \Delta u) \diff x = \int_{\partial\Omega} \left(u \frac{\partial v}{\partial n} - v \frac{\partial u}{\partial n}\right) \diff S
		\]
		其中 $\Omega$ 是 $\R^N$ 中的区域,$\partial\Omega$ 是其边界,$\frac{\partial}{\partial n}$ 表示沿边界外法线方向的方向导数。
		
		根据第一格林公式,我们有:
		\begin{equation} \label{eq:green1_v}
			\int_{\Omega} (v \Delta u + \nabla v \cdot \nabla u) \diff x = \int_{\partial\Omega} v \frac{\partial u}{\partial n} \diff S
		\end{equation}
		
		\noindent
		交换上式中 $u$ 和 $v$ 的位置,得到:
		\begin{equation} \label{eq:green1_u}
			\int_{\Omega} (u \Delta v + \nabla u \cdot \nabla v) \diff x = \int_{\partial\Omega} u \frac{\partial v}{\partial n} \diff S
		\end{equation}
		
		\noindent
		将方程 \eqref{eq:green1_v} 减去方程 \eqref{eq:green1_u},注意到 $\nabla u \cdot \nabla v = \nabla v \cdot \nabla u$,这两项相减后消去,即可得到\textbf{第二格林公式}:
		\[
		\int_{\Omega} (v \Delta u - u \Delta v) \diff x = \int_{\partial\Omega} \left(v \frac{\partial u}{\partial n} - u \frac{\partial v}{\partial n}\right) \diff S
		\]
	\end{solution}
	
	\newpage
	% ========================= 计算题 =========================
	\section*{二、计算题 (本大题共2小题, 每小题15分, 共30分)}
	
	\begin{problem}
		判断下列方程的类型,并化成标准型:
		\[
		u_{xx} + 2u_{xy} - 3u_{yy} + 2u_x + 6u_y = 0.
		\]
	\end{problem}
	\hrulefill
	\begin{solution}
		\textbf{步骤1. 判断方程类型}
		
		\noindent
		方程的系数为 $A=1, B=2, C=-3$。计算判别式:
		\[
		\Delta = B^2 - 4AC = 2^2 - 4(1)(-3) = 4 + 12 = 16 > 0
		\]
		因为 $\Delta > 0$,所以该方程为双曲型方程。
		
		\hrulefill
		
		\textbf{步骤2. 求解特征方程}
		
		\noindent
		特征方程为 $A \left(\frac{dy}{dx}\right)^2 - B \frac{dy}{dx} + C = 0$,即:
		\[
		\left(\frac{dy}{dx}\right)^2 - 2\frac{dy}{dx} - 3 = 0
		\]
		令 $\lambda = \frac{dy}{dx}$,则有 $\lambda^2 - 2\lambda - 3 = 0$,分解因式得 $(\lambda - 3)(\lambda + 1) = 0$。
		解得两个特征方向:
		\[
		\lambda_1 = 3, \quad \lambda_2 = -1
		\]
		对应的特征线方程为:
		\begin{align*}
			\frac{dy}{dx} = 3 &\implies dy - 3dx = 0 \implies y - 3x = C_1 \\
			\frac{dy}{dx} = -1 &\implies dy + dx = 0 \implies y + x = C_2
		\end{align*}
		
		\hrulefill
		
		\textbf{步骤3. 进行坐标变换}
		
		\noindent
		取新的坐标系:
		\[
		\begin{cases}
			\xi = y - 3x \\
			\eta = y + x
		\end{cases}
		\]
		计算各阶偏导数:
		\begin{align*}
			u_x &= -3u_\xi + u_\eta & u_y &= u_\xi + u_\eta \\
			u_{xx} &= 9u_{\xi\xi} - 6u_{\xi\eta} + u_{\eta\eta} & u_{yy} &= u_{\xi\xi} + 2u_{\xi\eta} + u_{\eta\eta} \\
			u_{xy} &= -3u_{\xi\xi} - 2u_{\xi\eta} + u_{\eta\eta}
		\end{align*}
		
		\hrulefill
		
		\textbf{步骤4. 代入原方程化简}
		
		\noindent
		将偏导数代入原方程:
		\begin{itemize}
			\item \textbf{二阶项}:
			$u_{xx} + 2u_{xy} - 3u_{yy} = (9-6-3)u_{\xi\xi} + (-6-4-6)u_{\xi\eta} + (1+2-3)u_{\eta\eta} = -16u_{\xi\eta}$
			\item \textbf{一阶项}:
			$2u_x + 6u_y = 2(-3u_\xi + u_\eta) + 6(u_\xi + u_\eta) = -6u_\xi + 2u_\eta + 6u_\xi + 6u_\eta = 8u_\eta$
		\end{itemize}
		合并所有项,得到变换后的方程:
		\[ -16u_{\xi\eta} + 8u_\eta = 0 \]
		两边同除以 $-8$,得到标准型:
		\[ 2u_{\xi\eta} - u_\eta = 0 \]
	\end{solution}
	\hrulefill
	
\begin{problem}
	求解 Cauchy 问题:
	\[
	\begin{cases}
		u_t - u_{xx} - 6tu = 0, & -\infty < x < \infty, \ t > 0, \\
		u|_{t=0} = xe^{x}, & -\infty < x < \infty.
	\end{cases}
	\]
\end{problem}
\hrulefill
\begin{solution}
	\textbf{步骤1. 消除含 $u$ 的项}
	
	\noindent
	设变量代换为 $u(x,t) = F(t) V(x,t)$。计算其偏导数:
	\[ u_t = F'(t)V + F(t)V_t, \quad u_{xx} = F(t)V_{xx} \]
	代入原方程:
	\[ F'(t)V + F(t)V_t - F(t)V_{xx} - 6t F(t)V = 0 \]
	整理得 $F(t) [ V_t - V_{xx} ] + [ F'(t) - 6t F(t) ] V = 0$。
	为使方程简化,我们令 $V(x,t)$ 的系数项为零: $F'(t) - 6t F(t) = 0$。
	解此常微分方程:
	\[ \frac{F'(t)}{F(t)} = 6t \implies \int \frac{\diff F}{F} = \int 6t \, \diff t \implies \ln|F| = 3t^2 + C \]
	为方便计算,取 $C=0$ 和正号,得 $F(t) = e^{3t^2}$。
	
	\hrulefill
	
	\textbf{步骤2. 求解新的 Cauchy 问题}
	
	\noindent
	原问题转化为关于 $V(x,t)$ 的标准热传导方程问题:
	\[
	\begin{cases}
		V_t - V_{xx} = 0, & -\infty < x < \infty, \ t > 0, \\
		V(x,0) = u(x,0)/F(0) = xe^x/e^0 = xe^x, & -\infty < x < \infty.
	\end{cases}
	\]
	该问题的解由泊松公式给出 ($a=1$, $n=1$):
	\[ V(x,t) = \frac{1}{\sqrt{4\pi t}} \int_{-\infty}^{\infty} y e^y e^{-\frac{(x-y)^2}{4t}} \diff y \]
	
	\hrulefill
	
	\textbf{步骤3. 计算积分 (详细配方法)}
	
	\noindent
	我们处理指数部分的项,进行配方:
	\begin{align*}
		y - \frac{(x-y)^2}{4t} &= -\frac{1}{4t} [-4ty + (x-y)^2] \\
		&= -\frac{1}{4t} [-4ty + x^2 - 2xy + y^2] \\
		&= -\frac{1}{4t} [y^2 - 2(x+2t)y + x^2] \\
		&= -\frac{1}{4t} \left[ (y - (x+2t))^2 - (x+2t)^2 + x^2 \right] \\
		&= -\frac{(y - (x+2t))^2}{4t} + \frac{(x+2t)^2 - x^2}{4t} \\
		&= -\frac{(y - (x+2t))^2}{4t} + \frac{x^2 + 4xt + 4t^2 - x^2}{4t} \\
		&= -\frac{(y - (x+2t))^2}{4t} + x + t
	\end{align*}
	将此结果代入积分表达式:
	\[
	V(x,t) = \frac{1}{\sqrt{4\pi t}} \int_{-\infty}^{\infty} y e^{-\frac{(y - (x+2t))^2}{4t}} e^{x+t} \diff y = \frac{e^{x+t}}{\sqrt{4\pi t}} \int_{-\infty}^{\infty} y e^{-\frac{(y - (x+2t))^2}{4t}} \diff y
	\]
	作变量代换,令 $z = y - (x+2t)$,则 $y = z + x + 2t$,$\diff y = \diff z$。
	\begin{align*}
		\int_{-\infty}^{\infty} y e^{-\frac{(y - (x+2t))^2}{4t}} \diff y &= \int_{-\infty}^{\infty} (z + x + 2t) e^{-\frac{z^2}{4t}} \diff z \\
		&= \int_{-\infty}^{\infty} z e^{-\frac{z^2}{4t}} \diff z + (x+2t) \int_{-\infty}^{\infty} e^{-\frac{z^2}{4t}} \diff z
	\end{align*}
	第一项的被积函数是奇函数,积分为 $0$。第二项是高斯积分,$\int_{-\infty}^{\infty} e^{-az^2}\diff z = \sqrt{\frac{\pi}{a}}$。这里 $a = \frac{1}{4t}$。
	\[
	(x+2t) \int_{-\infty}^{\infty} e^{-\frac{z^2}{4t}} \diff z = (x+2t)\sqrt{4\pi t}
	\]
	因此,$V(x,t) = \frac{e^{x+t}}{\sqrt{4\pi t}} \left[ (x+2t)\sqrt{4\pi t} \right] = (x+2t)e^{x+t}$。
	
	\hrulefill
	
	\textbf{步骤4. 回代得到最终解}
	
	\noindent
	$u(x,t) = V(x,t) F(t) = (x+2t)e^{x+t} e^{3t^2}$。
	\[
	u(x,t) = (x+2t)e^{x+t+3t^2}
	\]
\end{solution}
	
	\newpage
	% ========================= 解答题 =========================
	\section*{三、解答题 (本大题共1题, 共20分)}
	\begin{problem}
		用分离变量法求解初边值问题:
		\[
		\begin{cases}
			u_{tt} - u_{xx} = 0, & 0 < x < 2, \ t > 0, \\
			u(x, 0) = \frac{1}{4}\sin(\pi x), \quad u_t(x, 0) = 0, & 0 \leq x \leq 2, \\
			u(0, t) = 0, \quad u(2, t) = 0, & t \geq 0.
		\end{cases}
		\]
	\end{problem}
	\hrulefill
	\begin{solution}
		\textbf{步骤1. 分离变量}
		
		\noindent
		设 $u(x,t) = X(x)T(t)$。代入方程 $u_{tt} - u_{xx} = 0$:
		\[ X(x)T''(t) - X''(x)T(t) = 0 \implies \frac{T''(t)}{T(t)} = \frac{X''(x)}{X(x)} = -k \]
		其中 $-k$ 是分离常数。得到两个常微分方程:
		\[ X''(x) + k X(x) = 0 \quad \text{和} \quad T''(t) + k T(t) = 0 \]
		
		\hrulefill
		
		\textbf{步骤2. 求解空间本征值问题}
		
		\noindent
		空间方程的边界条件为 $X(0)=0$ 和 $X(2)=0$。我们对 $k$ 分情况讨论:
		\begin{itemize}
			\item \textbf{情况一:$k < 0$}.
			设 $k = -\mu^2$ ($\mu>0$),方程为 $X'' - \mu^2 X = 0$,通解为 $X(x) = c_1 \cosh(\mu x) + c_2 \sinh(\mu x)$。
			由 $X(0)=0$ 得 $c_1=0$。由 $X(2)=0$ 得 $c_2 \sinh(2\mu) = 0$。因 $\mu>0$, $\sinh(2\mu) > 0$,故 $c_2=0$。只有平凡解,舍去。
			
			\item \textbf{情况二:$k = 0$}.
			方程为 $X''=0$,通解为 $X(x) = c_1 x + c_2$。
			由 $X(0)=0$ 得 $c_2=0$。由 $X(2)=0$ 得 $2c_1=0$,故 $c_1=0$。只有平凡解,舍去。
			
			\item \textbf{情况三:$k > 0$}.
			设 $k = \mu^2$ ($\mu>0$),方程为 $X'' + \mu^2 X = 0$,通解为 $X(x) = c_1 \cos(\mu x) + c_2 \sin(\mu x)$。
			由 $X(0)=0$ 得 $c_1=0$。
			由 $X(2)=0$ 得 $c_2 \sin(2\mu) = 0$。为得到非平凡解,须 $c_2 \neq 0$,故 $\sin(2\mu)=0$。
			因此 $2\mu = n\pi$,即 $\mu_n = \frac{n\pi}{2}$ ($n=1, 2, 3, \dots$)。
		\end{itemize}
		由此得到本征值 $k_n = \mu_n^2 = \left(\frac{n\pi}{2}\right)^2$。
		对应的本征函数为 $X_n(x) = \sin\left(\frac{n\pi x}{2}\right)$。
		
		\hrulefill
		
		\textbf{步骤3. 求解时间方程}
		
		\noindent
		时间方程为 $T_n'' + k_n T_n = 0$,即 $T_n'' + \left(\frac{n\pi}{2}\right)^2 T_n = 0$。
		其通解为 $T_n(t) = a_n \cos\left(\frac{n\pi t}{2}\right) + b_n \sin\left(\frac{n\pi t}{2}\right)$。
		
		\hrulefill
		
		\textbf{步骤4. 叠加并利用初始条件}
		
		\noindent
		解的级数形式为 $u(x,t) = \sum_{n=1}^\infty \left[ a_n \cos\left(\frac{n\pi t}{2}\right) + b_n \sin\left(\frac{n\pi t}{2}\right) \right] \sin\left(\frac{n\pi x}{2}\right)$。
		利用 $u(x,0) = \frac{1}{4}\sin(\pi x)$:
		\[ u(x,0) = \sum_{n=1}^\infty a_n \sin\left(\frac{n\pi x}{2}\right) = \frac{1}{4}\sin(\pi x) \]
		比较系数,当 $\frac{n\pi}{2} = \pi$ 即 $n=2$ 时,$a_2 = \frac{1}{4}$。当 $n \neq 2$ 时,$a_n=0$。
		
		对时间求导:$u_t(x,t) = \sum_{n=1}^\infty \left[ -a_n\frac{n\pi}{2}\sin\left(\frac{n\pi t}{2}\right) + b_n\frac{n\pi}{2}\cos\left(\frac{n\pi t}{2}\right) \right] \sin\left(\frac{n\pi x}{2}\right)$。
		利用 $u_t(x,0) = 0$:
		\[ u_t(x,0) = \sum_{n=1}^\infty b_n \frac{n\pi}{2} \sin\left(\frac{n\pi x}{2}\right) = 0 \]
		因此,所有 $b_n=0$。
		
		\hrulefill
		
		\textbf{步骤5. 最终解}
		
		\noindent
		将求得的系数代回,只有 $n=2$ 的项非零:
		\[ u(x,t) = a_2 \cos\left(\frac{2\pi t}{2}\right) \sin\left(\frac{2\pi x}{2}\right) = \frac{1}{4}\cos(\pi t)\sin(\pi x) \]
	\end{solution}
	
	\newpage
	% ========================= 证明题 =========================
	\section*{四、证明题 (本大题共1题, 共15分)}
	\begin{problem}
		设 $u(x_1, \dots, x_n) = f(r)$ (其中 $r = \sqrt{x_1^2 + \dots + x_n^2}$) 是 $n$ 维 ($n \geq 3$) Laplace 方程 $\Delta_n u = \sum_{i=1}^n \frac{\partial^2 u}{\partial x_i^2} = 0$ 的解。试证明 $f(r) = C_1 + \frac{C_2}{r^{n-2}}$,其中 $C_1, C_2$ 为任意常数。
	\end{problem}
	\hrulefill
	\begin{solution}
		\textbf{步骤1. 计算径向函数的偏导数}
		
		\noindent
		我们有 $r^2 = \sum_{i=1}^n x_i^2$。对 $x_i$ 求导:
		\[
		\frac{\partial r}{\partial x_i} = \frac{x_i}{r}
		\]
		\[
		\frac{\partial^2 r}{\partial x_i^2} = \frac{\partial}{\partial x_i}\left(\frac{x_i}{r}\right) = \frac{1 \cdot r - x_i \frac{\partial r}{\partial x_i}}{r^2} = \frac{r - x_i \frac{x_i}{r}}{r^2} = \frac{r^2 - x_i^2}{r^3} = \frac{1}{r} - \frac{x_i^2}{r^3}
		\]
		利用链式法则计算 $u=f(r)$ 的偏导数:
		\[
		\frac{\partial u}{\partial x_i} = f'(r) \frac{\partial r}{\partial x_i}
		\]
		\[
		\frac{\partial^2 u}{\partial x_i^2} = f''(r)\left(\frac{\partial r}{\partial x_i}\right)^2 + f'(r)\frac{\partial^2 r}{\partial x_i^2} = f''(r)\frac{x_i^2}{r^2} + f'(r)\left(\frac{1}{r} - \frac{x_i^2}{r^3}\right)
		\]
		
		\hrulefill
		
		\textbf{步骤2. 推导拉普拉斯算子}
		
		\noindent
		将所有二阶导数相加得到拉普拉斯算子:
		\begin{align*}
			\Delta_n u = \sum_{i=1}^n \frac{\partial^2 u}{\partial x_i^2} &= \sum_{i=1}^n \left[ f''(r)\frac{x_i^2}{r^2} + f'(r)\left(\frac{1}{r} - \frac{x_i^2}{r^3}\right) \right] \\
			&= f''(r) \sum_{i=1}^n \frac{x_i^2}{r^2} + f'(r) \sum_{i=1}^n \left(\frac{1}{r} - \frac{x_i^2}{r^3}\right) \\
			&= f''(r) \frac{\sum x_i^2}{r^2} + f'(r) \left(\frac{n}{r} - \frac{\sum x_i^2}{r^3}\right) \\
			&= f''(r) \frac{r^2}{r^2} + f'(r) \left(\frac{n}{r} - \frac{r^2}{r^3}\right) = f''(r) + \frac{n-1}{r} f'(r)
		\end{align*}
		
		\hrulefill
		
		\textbf{步骤3. 求解常微分方程}
		
		\noindent
		因为 $\Delta_n u = 0$,我们得到关于 $f(r)$ 的常微分方程:
		\[ f''(r) + \frac{n-1}{r} f'(r) = 0 \]
		令 $g(r) = f'(r)$,方程变为一阶方程: $g'(r) + \frac{n-1}{r} g(r) = 0$。
		分离变量:
		\[ \frac{\diff g}{g} = -(n-1) \frac{\diff r}{r} \]
		两边积分:$\ln|g| = -(n-1)\ln|r| + \ln|A| \implies g(r) = \frac{A}{r^{n-1}}$。
		
		\hrulefill
		
		\textbf{步骤4. 积分得到 $f(r)$}
		
		\noindent
		我们已经得到 $f'(r) = g(r) = A r^{-(n-1)}$。再次对 $r$ 积分:
		\[ f(r) = \int A r^{-(n-1)} \diff r \]
		因为 $n \ge 3$, $-(n-1) \neq -1$。
		\[ f(r) = A \frac{r^{-n+2}}{-n+2} + C_1 \]
		令常数 $C_2 = \frac{A}{2-n}$,则解的形式为:
		\[ f(r) = C_1 + \frac{C_2}{r^{n-2}} \]
		证明完毕。
	\end{solution}
	
\end{document}