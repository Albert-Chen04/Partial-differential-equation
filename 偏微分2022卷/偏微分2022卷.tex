\documentclass[12pt,a4paper]{article}
\usepackage[UTF8]{ctex}                    % 中文支持
\usepackage{geometry}                      % 页面布局
\usepackage{amsmath,amssymb,amsthm}        % 数学符号与定理
\usepackage{mathtools,bm}                  % 高级数学工具
\usepackage{graphicx}                      % 图片插入
\usepackage{hyperref}                      % 超链接
\hypersetup{
	pdfborder = {0 0 0}  % 关闭链接边框
}
\usepackage{cleveref}
\usepackage{booktabs}                      % 三线表
\usepackage[numbers,sort&compress]{natbib} % 参考文献
\usepackage{caption}                       % 图表标题
\usepackage[shortlabels]{enumitem}         % 列表环境

% ========== 页面布局 ==========
\geometry{left=2cm, right=2cm, top=2.5cm, bottom=2.5cm}
\setlength{\parskip}{0.5em}                % 段落间距
\renewcommand{\baselinestretch}{1.3}       % 行距

% ========== 数学命令 ==========
\newcommand{\diff}{\mathop{}\!\mathrm{d}}  % 微分符号
\newcommand{\R}{\mathbb{R}}                % 实数集
\newcommand{\C}{\mathbb{C}}                % 复数集
\newcommand{\Z}{\mathbb{Z}}                % 整数集
\newcommand{\Lspace}{L^2(-l,l)}            % L²空间

% ========== 定理环境 ==========
% 定义题目环境
\newtheorem{problem}{题}
% 定义解题环境
\newtheorem*{solution}{解}

\newtheorem{example}{例题}
\newtheorem{corollary}{推论}
\newtheorem{proposition}{命题}
\newtheorem{lemma}{引理}

% ========== 文档信息 ==========
\begin{document}
	
	\begin{center}
				\LARGE 偏微分方程2022卷 \\
	\vspace{0.5cm}
	\large 班级:22数学1 \quad 姓名:陈柏均 \quad 学号:202225110102
	\end{center}

	
	% ========================= 基础题 =========================
	\section*{一、基础题 (本大题共5小题, 每小题4分, 共20分)}
	
	\begin{problem}
		指出方程 $(x^2 + y^2 + 3xy) u_{xy} + (\sin x + \cos y) u_{xxyy} = 2x^4 + 3x^2 \sin y + y^3$ 的阶,并判定它是线性的还是非线性的。
	\end{problem}
	\hrulefill
	\begin{solution}
		\textbf{阶数:}方程中最高阶导数是 $u_{xxyy}$,其阶数为 $2+2=4$。所以这是一个四阶偏微分方程。
		
		\noindent
		\textbf{线性性:}该方程是线性的。因为未知函数 $u$ 及其各阶偏导数都是一次的,并且其系数和方程右端的项都仅与自变量 $x, y$ 有关。
	\end{solution}
	\hrulefill
	
	\begin{problem}
		写出方程 $u_{xy} + u_{yz} + u_{zz} = 0$ 的特征方程。
	\end{problem}
	\hrulefill
	\begin{solution}
		设特征曲面为 $\phi(x, y, z) = C$。特征方程由主部系数决定,其形式为:
		\[
		\phi_x \phi_y + \phi_y \phi_z + \phi_z^2 = 0
		\]
	\end{solution}
	\hrulefill
	
	\begin{problem}
		验证 $v(x, y) = \dfrac{y}{x^2 + y^2}$ 是调和的。
	\end{problem}
	\hrulefill
	\begin{solution}
		一个函数是调和的,如果它满足拉普拉斯方程 $\Delta v = v_{xx} + v_{yy} = 0$。
		\begin{align*}
			v_x &= \frac{\partial}{\partial x} \left( y(x^2+y^2)^{-1} \right) = -y(x^2+y^2)^{-2}(2x) = \frac{-2xy}{(x^2+y^2)^2} \\
			v_{xx} &= \frac{-2y(x^2+y^2)^2 - (-2xy) \cdot 2(x^2+y^2)(2x)}{(x^2+y^2)^4} = \frac{-2y(x^2+y^2) + 8x^2y}{(x^2+y^2)^3} = \frac{6x^2y-2y^3}{(x^2+y^2)^3} \\
			v_y &= \frac{1 \cdot (x^2+y^2) - y \cdot (2y)}{(x^2+y^2)^2} = \frac{x^2-y^2}{(x^2+y^2)^2} \\
			v_{yy} &= \frac{-2y(x^2+y^2)^2 - (x^2-y^2) \cdot 2(x^2+y^2)(2y)}{(x^2+y^2)^4} = \frac{-2y(x^2+y^2) - 4y(x^2-y^2)}{(x^2+y^2)^3} = \frac{-6x^2y+2y^3}{(x^2+y^2)^3}
		\end{align*}
		因此,
		\[
		v_{xx} + v_{yy} = \frac{6x^2y-2y^3}{(x^2+y^2)^3} + \frac{-6x^2y+2y^3}{(x^2+y^2)^3} = 0
		\]
		所以,$v(x,y)$ 是调和函数(在 $(0,0)$ 点外)。
	\end{solution}
	\hrulefill
	
	\begin{problem}
		对于 Cauchy 问题
		$
		\begin{cases}
			u_{tt} - 16u_{xx} = 0, & -\infty < x < \infty, \ t > 0, \\
			u(x, 0) = \phi(x), \quad u_t(x, 0) = \psi(x), & -\infty < x < \infty,
		\end{cases}
		$
		求关于点 $(0,2)$ 的依赖区域。
	\end{problem}
	\hrulefill
	\begin{solution}
		这是一个一维波动方程,波速 $c$ 满足 $c^2=16$,所以 $c=4$。
		点 $(x_0, t_0) = (0, 2)$ 的依赖区域是初始直线 $t=0$ 上的一个区间 $[x_0 - ct_0, x_0 + ct_0]$。
		代入数值:
		\[
		[0 - 4 \cdot 2, \quad 0 + 4 \cdot 2] = [-8, 8]
		\]
		所以点 $(0,2)$ 的依赖区域是初始轴上的闭区间 $[-8, 8]$。
	\end{solution}
	\hrulefill
	
	\begin{problem}
		写出一维波动方程 Cauchy 问题
		$
		\begin{cases}
			u_{tt} - u_{xx} = 0, & -\infty < x < \infty, \ t > 0, \\
			u(x, 0) = \phi(x), \quad u_t(x, 0) = \psi(x), & -\infty < x < \infty
		\end{cases}
		$
		的 D'Alembert (达朗贝尔) 公式。
	\end{problem}
	\hrulefill
	\begin{solution}
		该方程的波速 $c=1$。D'Alembert 公式为:
		\[
		u(x,t) = \frac{1}{2}[\phi(x+t) + \phi(x-t)] + \frac{1}{2}\int_{x-t}^{x+t} \psi(s) \diff s
		\]
	\end{solution}
	
	\newpage
	% ========================= 计算题 =========================
	\section*{二、计算题 (本大题共3题, 前两题每题各15分, 第三题10分, 共40分)}
	
	\begin{problem}
		判断下列方程的类型,并化成标准型:
		\[ 3u_{xx} + 2u_{xy} - u_{yy} + u_x + u_y = 0. \]
	\end{problem}
	\hrulefill
	\begin{solution}
		\textbf{步骤1. 判断类型}
		
		\noindent
		系数 $A=3, B=2, C=-1$。判别式 $\Delta = B^2 - 4AC = 2^2 - 4(3)(-1) = 4+12=16>0$。方程为双曲型。
		
		\noindent
		\textbf{步骤2. 求解特征方程}
		
		\noindent
		特征方程为 $A(\frac{dy}{dx})^2 - B\frac{dy}{dx} + C = 0 \implies 3(\frac{dy}{dx})^2 - 2\frac{dy}{dx} - 1 = 0$。
		分解得 $(3\frac{dy}{dx}+1)(\frac{dy}{dx}-1)=0$。
		特征方向为 $\lambda_1 = 1$ 和 $\lambda_2 = -1/3$。
		对应的特征线方程为 $y-x=C_1$ 和 $y+\frac{1}{3}x=C_2$ (或 $3y+x=C_2$)。
		
		\noindent
		\textbf{步骤3. 进行坐标变换}
		
		\noindent
		取新坐标 $\xi = y-x, \eta=3y+x$。计算偏导数:
		\begin{align*}
			u_x &= u_\xi \xi_x + u_\eta \eta_x = -u_\xi + u_\eta \\
			u_y &= u_\xi \xi_y + u_\eta \eta_y = u_\xi + 3u_\eta \\
			u_{xx} &= \frac{\partial}{\partial x}(-u_\xi + u_\eta) = -(-u_{\xi\xi} + u_{\xi\eta}) + (-u_{\eta\xi} + u_{\eta\eta}) = u_{\xi\xi} - 2u_{\xi\eta} + u_{\eta\eta} \\
			u_{xy} &= \frac{\partial}{\partial y}(-u_\xi + u_\eta) = -(u_{\xi\xi} + 3u_{\xi\eta}) + (u_{\eta\xi} + 3u_{\eta\eta}) = -u_{\xi\xi} - 2u_{\xi\eta} + 3u_{\eta\eta} \\
			u_{yy} &= \frac{\partial}{\partial y}(u_\xi + 3u_\eta) = (u_{\xi\xi} + 3u_{\xi\eta}) + 3(u_{\eta\xi} + 3u_{\eta\eta}) = u_{\xi\xi} + 6u_{\xi\eta} + 9u_{\eta\eta}
		\end{align*}
		
		\noindent
		\textbf{步骤4. 代入原方程化简}
		
		\noindent
		\textbf{二阶项}: $3u_{xx} + 2u_{xy} - u_{yy} = 3(u_{\xi\xi} - 2u_{\xi\eta} + u_{\eta\eta}) + 2(-u_{\xi\xi} - 2u_{\xi\eta} + 3u_{\eta\eta}) - (u_{\xi\xi} + 6u_{\xi\eta} + 9u_{\eta\eta}) = (3-2-1)u_{\xi\xi} + (-6-4-6)u_{\xi\eta} + (3+6-9)u_{\eta\eta} = -16u_{\xi\eta}$。
		
		\noindent
		\textbf{一阶项}: $u_x + u_y = (-u_\xi + u_\eta) + (u_\xi + 3u_\eta) = 4u_\eta$。
		
		\noindent
		合并得到 $-16u_{\xi\eta} + 4u_\eta = 0$。两边同除以 $-4$,得到标准型:
		\[ 4u_{\xi\eta} - u_\eta = 0 \]
	\end{solution}
	\hrulefill
	
	\begin{problem}
		求解热传导方程 $u_t - 4u_{xx} = 0, (-\infty < x < \infty, t>0)$ 的 Cauchy 问题,已知初值条件为 $u|_{t=0} = (x+1)^2$。
	\end{problem}
	\hrulefill
	\begin{solution}
		\textbf{步骤1. 使用泊松公式}
		
		\noindent
		这是一个标准的热传导方程 Cauchy 问题。方程形式为 $u_t - a^2 u_{xx}=0$,这里 $a^2=4$。初始条件为 $\phi(y)=(y+1)^2$。
		根据泊松公式,解为:
		\[ u(x,t) = \frac{1}{\sqrt{4\pi a^2 t}} \int_{-\infty}^\infty \phi(y) e^{-\frac{(x-y)^2}{4a^2 t}} \diff y = \frac{1}{\sqrt{16\pi t}} \int_{-\infty}^\infty (y+1)^2 e^{-\frac{(x-y)^2}{16t}} \diff y \]
		
		\hrulefill
		
		\textbf{步骤2. 变量代换与展开}
		
		\noindent
		令 $z=y-x$, 则 $y=z+x$, $\diff y = \diff z$。此时 $y+1 = z+x+1$。
		\begin{align*}
			u(x,t) &= \frac{1}{\sqrt{16\pi t}} \int_{-\infty}^\infty (z+x+1)^2 e^{-\frac{z^2}{16t}} \diff z \\
			&= \frac{1}{4\sqrt{\pi t}} \int_{-\infty}^\infty ((x+1)+z)^2 e^{-\frac{z^2}{16t}} \diff z \\
			&= \frac{1}{4\sqrt{\pi t}} \int_{-\infty}^\infty \left[ (x+1)^2 + 2(x+1)z + z^2 \right] e^{-\frac{z^2}{16t}} \diff z \\
			&= \frac{1}{4\sqrt{\pi t}} \left[ \int_{-\infty}^\infty z^2 e^{-\frac{z^2}{16t}} \diff z + \int_{-\infty}^\infty 2(x+1)z e^{-\frac{z^2}{16t}} \diff z + \int_{-\infty}^\infty (x+1)^2 e^{-\frac{z^2}{16t}} \diff z \right]
		\end{align*}
		
		\hrulefill
		
		\textbf{步骤3. 计算各积分项}
		
		\noindent
		我们分项计算上述三个积分。
		
		\begin{itemize}
			\item \textbf{第一项:$\int z^2 e^{-\frac{z^2}{16t}} \diff z$}
			
			我们使用对参数求导的方法。设 $A(\alpha) = \int_{-\infty}^\infty e^{-\alpha z^2} \diff z = \sqrt{\frac{\pi}{\alpha}} = \pi^{1/2}\alpha^{-1/2}$。
			两边对 $\alpha$ 求导:
			\[ \frac{\diff A}{\diff \alpha} = \int_{-\infty}^\infty \frac{\partial}{\partial \alpha} (e^{-\alpha z^2}) \diff z = \int_{-\infty}^\infty -z^2 e^{-\alpha z^2} \diff z \]
			另一方面,对 $A(\alpha)$ 的闭式解求导:
			\[ \frac{\diff A}{\diff \alpha} = \pi^{1/2} \left(-\frac{1}{2}\alpha^{-3/2}\right) = -\frac{1}{2}\pi^{1/2}\alpha^{-3/2} \]
			因此,$\int_{-\infty}^\infty z^2 e^{-\alpha z^2} \diff z = -\frac{\diff A}{\diff \alpha} = \frac{1}{2}\pi^{1/2}\alpha^{-3/2}$。
			令 $\alpha = \frac{1}{16t}$,则
			\[ \int_{-\infty}^\infty z^2 e^{-\frac{z^2}{16t}} \diff z = \frac{1}{2}\pi^{1/2}\left(\frac{1}{16t}\right)^{-3/2} = \frac{1}{2}\pi^{1/2}(16t)^{3/2} = \frac{1}{2}\pi^{1/2}(64t^{3/2}) = 32\pi^{1/2}t^{3/2} \]
			
			\item \textbf{第二项:$\int 2(x+1)z e^{-\frac{z^2}{16t}} \diff z$}
			
			被积函数 $z e^{-\frac{z^2}{16t}}$ 是关于 $z$ 的奇函数,在对称区间 $(-\infty, \infty)$ 上的积分为 $0$。
			
			\item \textbf{第三项:$\int (x+1)^2 e^{-\frac{z^2}{16t}} \diff z$}
			
			这是一个标准的高斯积分:
			\[ (x+1)^2 \int_{-\infty}^\infty e^{-\frac{z^2}{16t}} \diff z = (x+1)^2 \sqrt{\frac{\pi}{1/(16t)}} = (x+1)^2 \sqrt{16\pi t} \]
		\end{itemize}
		
		\hrulefill
		
		\textbf{步骤4. 合并得到最终解}
		
		\noindent
		将计算结果代回 $u(x,t)$ 的表达式:
		\begin{align*}
			u(x,t) &= \frac{1}{\sqrt{16\pi t}} \left[ 32\pi^{1/2}t^{3/2} + 0 + (x+1)^2 \sqrt{16\pi t} \right] \\
			&= \frac{32\pi^{1/2}t^{3/2}}{\sqrt{16\pi t}} + \frac{(x+1)^2 \sqrt{16\pi t}}{\sqrt{16\pi t}} \\
			&= \frac{32\sqrt{\pi}t^{3/2}}{4\sqrt{\pi}\sqrt{t}} + (x+1)^2 \\
			&= 8t + (x+1)^2
		\end{align*}
		最终解为 $u(x,t) = (x+1)^2 + 8t$。
	\end{solution}
	\hrulefill
	
	\begin{problem}
		求方程 $u_{xx}+u_{yy}+u_{zz} - (x^2+y^2+z^2) = 0$ 的球对称解 $u(x,y,z)=u(r)$,其中 $r=\sqrt{x^2+y^2+z^2}$。
	\end{problem}
	\hrulefill
	\begin{solution}
		\textbf{步骤1. 将偏微分方程转化为常微分方程}
		
		\noindent
		对于球对称函数 $u(r)$,原方程变为 $\Delta u = r^2$。我们首先推导拉普拉斯算子 $\Delta u$ 作用于 $u(r)$ 的表达式。
		\[
		\frac{\partial r}{\partial x} = \frac{x}{r}, \quad \frac{\partial r}{\partial y} = \frac{y}{r}, \quad \frac{\partial r}{\partial z} = \frac{z}{r}
		\]
		利用链式法则:
		\[
		u_x = u_r \frac{\partial r}{\partial x} = u_r \frac{x}{r}
		\]
		\[
		u_{xx} = \frac{\partial}{\partial x}\left(u_r \frac{x}{r}\right) = u_{rr}\left(\frac{\partial r}{\partial x}\right)^2 + u_r \frac{\partial^2 r}{\partial x^2} = u_{rr}\frac{x^2}{r^2} + u_r\left(\frac{1}{r} - \frac{x^2}{r^3}\right)
		\]
		将三个二阶偏导数相加:
		\begin{align*}
			\Delta u = u_{xx}+u_{yy}+u_{zz} &= \sum_{cyc} \left( u_{rr}\frac{x^2}{r^2} + u_r\left(\frac{1}{r} - \frac{x^2}{r^3}\right) \right) \\
			&= u_{rr} \frac{x^2+y^2+z^2}{r^2} + u_r \left(\frac{3}{r} - \frac{x^2+y^2+z^2}{r^3}\right) \\
			&= u_{rr} + u_r \left(\frac{3}{r} - \frac{r^2}{r^3}\right) = u_{rr} + \frac{2}{r}u_r
		\end{align*}
		因此,原方程化为常微分方程:
		\[
		u_{rr} + \frac{2}{r}u_r = r^2
		\]
		
		\hrulefill
		
		\textbf{步骤2. 使用变量代换求解常微分方程}
		
		\noindent
		作变量代换,令 $g(r) = r u(r)$,则 $u(r) = \frac{g(r)}{r}$。
		计算 $u$ 对 $r$ 的导数:
		\begin{align*}
			u_r &= \frac{g_r r - g}{r^2} = \frac{g_r}{r} - \frac{g}{r^2} \\[6pt]
			u_{rr} &= \frac{\partial}{\partial r}\left(\frac{g_r}{r} - \frac{g}{r^2}\right) = \left(\frac{g_{rr}r - g_r}{r^2}\right) - \left(\frac{g_r r^2 - g(2r)}{r^4}\right) \\
			&= \frac{g_{rr}}{r} - \frac{g_r}{r^2} - \frac{g_r}{r^2} + \frac{2g}{r^3} = \frac{g_{rr}}{r} - \frac{2g_r}{r^2} + \frac{2g}{r^3}
		\end{align*}
		将 $u_r, u_{rr}$ 代入常微分方程:
		\[
		\left( \frac{g_{rr}}{r} - \frac{2g_r}{r^2} + \frac{2g}{r^3} \right) + \frac{2}{r} \left( \frac{g_r}{r} - \frac{g}{r^2} \right) = r^2
		\]
		化简得:
		\[
		\frac{g_{rr}}{r} - \frac{2g_r}{r^2} + \frac{2g}{r^3} + \frac{2g_r}{r^2} - \frac{2g}{r^3} = r^2 \implies \frac{g_{rr}}{r} = r^2
		\]
		于是得到关于 $g(r)$ 的一个更简单的方程:
		\[
		g_{rr} = r^3
		\]
		
		也可用
		\begin{equation*}
			\frac{\partial^2}{\partial r^2}(ru) = \frac{\partial}{\partial r}(u + ru_r) = u_r + u_r + r u_{rr} = r u_{rr} + 2u_r
		\end{equation*}
		\hrulefill
		
		\textbf{步骤3. 积分并回代}
		
		\noindent
		对 $g_{rr} = r^3$ 积分两次:
		\begin{align*}
			g_r &= \int r^3 \diff r = \frac{1}{4}r^4 + C_1 \\
			g(r) &= \int \left(\frac{1}{4}r^4 + C_1\right) \diff r = \frac{1}{20}r^5 + C_1 r + C_2
		\end{align*}
		将 $u(r) = g(r)/r$ 代回,得到原方程的球对称解:
		\[
		u(r) = \frac{1}{r}\left(\frac{1}{20}r^5 + C_1 r + C_2\right) = \frac{1}{20}r^4 + C_1 + \frac{C_2}{r}
		\]
		其中 $C_1, C_2$ 是任意常数。
	\end{solution}
	
	\newpage
	% ========================= 解答题 =========================
	\section*{三、解答题 (本大题共1题, 共25分)}

\begin{problem}
	用分离变量法求解初边值问题:
	\[
	\begin{cases}
		u_{tt} - u_{xx} = 0, & 0 < x < 1, \ t > 0, \\
		u(x,0) = \cos\left(\frac{3\pi}{2}x\right), \quad u_t(x,0) = \cos\left(\frac{\pi}{2}x\right) + \cos\left(\frac{5\pi}{2}x\right), & 0 \leq x \leq 1, \\
		u_x(0,t) = 0, \quad u(1,t) = 0, & t \geq 0.
	\end{cases}
	\]
\end{problem}
\hrulefill
\begin{solution}
	\textbf{步骤1. 分离变量}
	
	\noindent
	设解的形式为 $u(x,t) = X(x)T(t)$。代入方程 $u_{tt} - u_{xx} = 0$:
	\[ X(x)T''(t) - X''(x)T(t) = 0 \implies \frac{T''(t)}{T(t)} = \frac{X''(x)}{X(x)} = -k \]
	其中 $-k$ 是分离常数。由此得到两个常微分方程:
	\[ X''(x) + k X(x) = 0 \quad \text{和} \quad T''(t) + k T(t) = 0 \]
	
	\hrulefill
	
	\textbf{步骤2. 求解空间本征值问题}
	
	\noindent
	空间方程的边界条件由原问题给出:$u_x(0,t) = X'(0)T(t)=0 \implies X'(0)=0$ 和 $u(1,t)=X(1)T(t)=0 \implies X(1)=0$。
	我们求解 Sturm-Liouville 问题: $X'' + k X = 0, \ X'(0)=0, \ X(1)=0$。
	
	\begin{itemize}
		\item \textbf{情况一:$k < 0$}.
		设 $k = -\mu^2$ ($\mu>0$),方程为 $X'' - \mu^2 X = 0$,通解为 $X(x) = c_1 \cosh(\mu x) + c_2 \sinh(\mu x)$。
		其导数为 $X'(x) = c_1\mu\sinh(\mu x) + c_2\mu\cosh(\mu x)$。
		由 $X'(0)=0$ 得 $c_2\mu=0 \implies c_2=0$。
		由 $X(1)=0$ 得 $c_1 \cosh(\mu) = 0$。因 $\mu>0$, $\cosh(\mu) > 1$,故 $c_1=0$。只有平凡解,舍去。
		
		\item \textbf{情况二:$k = 0$}.
		方程为 $X''=0$,通解为 $X(x) = c_1 x + c_2$。
		$X'(x) = c_1$。由 $X'(0)=0$ 得 $c_1=0$。
		由 $X(1)=0$ 得 $c_1 \cdot 1 + c_2 = c_2 = 0$。只有平凡解,舍去。
		
		\item \textbf{情况三:$k > 0$}.
		设 $k = \mu^2$ ($\mu>0$),方程为 $X'' + \mu^2 X = 0$,通解为 $X(x) = c_1 \cos(\mu x) + c_2 \sin(\mu x)$。
		$X'(x) = -c_1\mu\sin(\mu x) + c_2\mu\cos(\mu x)$。
		由 $X'(0)=0$ 得 $c_2\mu=0 \implies c_2=0$。
		由 $X(1)=0$ 得 $c_1 \cos(\mu) = 0$。为得到非平凡解,须 $c_1 \neq 0$,故 $\cos(\mu)=0$。
		因此 $\mu_n = \frac{\pi}{2} + n\pi = (n+\frac{1}{2})\pi$ for $n=0, 1, 2, \dots$。
	\end{itemize}
	本征值为 $k_n = \mu_n^2 = ((n+\frac{1}{2})\pi)^2$。
	对应的本征函数为 $X_n(x) = \cos\left((n+\frac{1}{2})\pi x\right)$。
	
	\hrulefill
	
	\textbf{步骤3. 求解时间方程及叠加}
	
	\noindent
	时间方程 $T_n'' + k_n T_n = 0$ 的解为 $T_n(t) = a_n \cos(\mu_n t) + b_n \sin(\mu_n t)$。
	根据叠加原理,解的形式为 $u(x,t) = \sum_{n=0}^\infty \left[a_n \cos(\mu_n t) + b_n \sin(\mu_n t)\right] \cos(\mu_n x)$。
	
	\hrulefill
	
	\textbf{步骤4. 利用初始条件确定系数}
	
	\noindent
	利用初始位移 $u(x,0) = \cos(\frac{3\pi}{2}x)$:
	\[ u(x,0) = \sum_{n=0}^\infty a_n \cos\left((n+\frac{1}{2})\pi x\right) = \cos\left(\frac{3\pi}{2}x\right) \]
	通过比较傅里叶级数的系数,当 $(n+\frac{1}{2})\pi = \frac{3\pi}{2} \implies n=1$ 时,$a_1=1$。其他所有 $a_n=0$。
	
	\noindent
	对时间求导:$u_t(x,t) = \sum_{n=0}^\infty \left[-a_n\mu_n \sin(\mu_n t) + b_n\mu_n \cos(\mu_n t)\right] \cos(\mu_n x)$。
	利用初始速度 $u_t(x,0) = \cos(\frac{\pi}{2}x) + \cos(\frac{5\pi}{2}x)$:
	\[ u_t(x,0) = \sum_{n=0}^\infty b_n\mu_n \cos(\mu_n x) = \cos\left(\frac{\pi}{2}x\right) + \cos\left(\frac{5\pi}{2}x\right) \]
	比较系数:
	\begin{itemize}
		\item 对于 $\cos(\frac{\pi}{2}x)$ 项: $n=0$, $\mu_0=\frac{\pi}{2}$。$b_0\mu_0 = 1 \implies b_0\left(\frac{\pi}{2}\right)=1 \implies b_0 = \frac{2}{\pi}$。
		\item 对于 $\cos(\frac{5\pi}{2}x)$ 项: $n=2$, $\mu_2=\frac{5\pi}{2}$。$b_2\mu_2 = 1 \implies b_2\left(\frac{5\pi}{2}\right)=1 \implies b_2 = \frac{2}{5\pi}$。
		\item 其他所有 $b_n=0$ (除了 $n=0,2$)。
	\end{itemize}
	
	\hrulefill
	
	\textbf{步骤5. 写出最终解}
	
	\noindent
	将所有非零系数代回级数,得到最终解:
	\begin{align*}
		u(x,t) &= a_1\cos(\mu_1 t)\cos(\mu_1 x) + b_0\sin(\mu_0 t)\cos(\mu_0 x) + b_2\sin(\mu_2 t)\cos(\mu_2 x) \\
		&= \cos\left(\frac{3\pi}{2}t\right)\cos\left(\frac{3\pi}{2}x\right) + \frac{2}{\pi}\sin\left(\frac{\pi}{2}t\right)\cos\left(\frac{\pi}{2}x\right) + \frac{2}{5\pi}\sin\left(\frac{5\pi}{2}t\right)\cos\left(\frac{5\pi}{2}x\right)
	\end{align*}
\end{solution}
	
	\newpage
	% ========================= 证明题 =========================
	\section*{四、证明题 (本大题共1题, 共15分)}
	
	\begin{problem}
		证明下述 Neumann 边值问题
		\[
		\begin{cases}
			\Delta u = g(x), & x \in \Omega \subset \R^n, \\
			\dfrac{\partial u}{\partial \nu} \Big|_{\Gamma} = f(y), & y \in \Gamma = \partial\Omega,
		\end{cases}
		\]
		有解的必要条件是
		\[ \int_{\Omega} g(x) \diff x = \int_{\Gamma} f(y) \diff S. \]
	\end{problem}
	\hrulefill
	\begin{solution}
		证明:
		
		\noindent
		假设该 Neumann 问题有(足够光滑的)解 $u(x)$。
		
		\noindent
		对泊松方程 $\Delta u = g(x)$ 在区域 $\Omega$ 上进行积分,可得:
		\[
		\int_{\Omega} \Delta u \diff x = \int_{\Omega} g(x) \diff x
		\]
		根据高斯散度定理(或格林第一恒等式),我们可以将左边的体积分转化为面积分:
		\[
		\int_{\Omega} \Delta u \diff x = \int_{\Omega} \mathrm{div}(\nabla u) \diff x = \int_{\partial \Omega} (\nabla u \cdot \mathbf{n}) \diff S
		\]
		其中 $\mathbf{n}$ 是边界 $\partial\Omega = \Gamma$ 上的单位外法向量。
		
		\noindent
		沿外法线方向的方向导数定义为 $\frac{\partial u}{\partial \nu} = \nabla u \cdot \mathbf{n}$。因此,
		\[
		\int_{\partial \Omega} (\nabla u \cdot \mathbf{n}) \diff S = \int_{\Gamma} \frac{\partial u}{\partial \nu} \diff S
		\]
		将上述等式联立,我们得到:
		\[
		\int_{\Gamma} \frac{\partial u}{\partial \nu} \diff S = \int_{\Omega} g(x) \diff x
		\]
		最后,利用给定的边界条件 $\frac{\partial u}{\partial \nu} \Big|_{\Gamma} = f(y)$,代入上式左端,即得:
		\[
		\int_{\Gamma} f(y) \diff S = \int_{\Omega} g(x) \diff x
		\]
		这就是解存在的必要条件,也称为相容性条件。证明完毕。
		
	\end{solution}
	
\end{document}