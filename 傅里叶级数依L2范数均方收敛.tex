\documentclass[12pt,a4paper]{article}
\usepackage[UTF8]{ctex}
\usepackage{geometry}
\usepackage{amsmath,amssymb,amsthm}
\usepackage{mathtools,bm}
\usepackage{empheq}
\usepackage{graphicx}
\usepackage{booktabs}
\usepackage[numbers,sort&compress]{natbib}
\usepackage{caption}
\usepackage{enumitem}
\usepackage{chngcntr}

% ========== 页面布局 ==========
\geometry{left=2.5cm,right=2.5cm,top=2.5cm,bottom=2.5cm}
\setlength{\parskip}{0.5em}
\renewcommand{\baselinestretch}{1.2}

% ========== 数学命令 ==========
\newcommand{\diff}{\mathop{}\!\mathrm{d}}
\newcommand{\R}{\mathbb{R}}
\newcommand{\C}{\mathbb{C}}
\newcommand{\Z}{\mathbb{Z}}
\newcommand{\N}{\mathbb{N}}
\DeclareMathOperator{\supp}{supp}

% ========== 编号系统 ==========
\numberwithin{subsection}{section}   % 子小节按章节编号
\numberwithin{subsubsection}{subsection}
\counterwithin{equation}{subsection} % 公式按子小节编号

% ========== 定理环境 ==========
\theoremstyle{plain}
\newtheorem{theorem}{定理}[section]
\newtheorem{lemma}[theorem]{引理}
\newtheorem{proposition}[theorem]{命题}
\newtheorem{corollary}[theorem]{推论}
\newtheorem{solution}{解}[subsection]  % 解按子小节编号

\theoremstyle{definition}
\newtheorem{definition}[theorem]{定义}
\newtheorem{example}{例题}[subsection]  % 示例按子小节编号

\theoremstyle{remark}
\newtheorem{remark}{注记}[subsection] % 直接在定义时指定计数器的层级关系

\theoremstyle{remark}
\newtheorem{verification}[theorem]{验证}

% 加载hyperref包以实现超链接功能
\usepackage[colorlinks=true, linkcolor=black]{hyperref}

\title{傅里叶级数依$L^2$范数均方收敛}
\author{陈柏均}
\date{2025年3月11日}

\begin{document}
	\maketitle

	\tableofcontents  % 添加目录
	
	
	\newpage
	
\section{三角函数系的完备性}	
\begin{proof}
要证明级数和能表示$L^2$空间中的任意函数(三角函数系能张成整个空间),即证明级数部分和的极限可以等于$L^2$空间中的任意函数,即级数部分和(三角多项式)在$L^2$空间中稠密。

	根据魏尔斯特拉斯定理,三角多项式在连续函数中稠密,连续函数在$L^2$空间中稠密,故得证。还可以用三角函数系的正交补为0来证明。
	
这个证明很多种方法,因为$L^2$空间是希尔伯特空间。所以三角函数系是$L^2$空间的一组基。
\end{proof}
	
	\section{Bessel不等式的第二种证明}
	

\begin{proof}
	考虑在$L^2$空间中,对于标准正交基$\{e_k\}_{k=1}^\infty$和函数$f$,我们有如下等式:
	
	\[
	\left\| f - \sum_{k=1}^n (f, e_k) e_k \right\|_2^2 = \left( f - \sum_{k=1}^n (f, e_k) e_k,\ f - \sum_{k=1}^n (f, e_k) e_k \right)
	\]

	\[
		= \ (f, f) - \left( f, \sum_{k=1}^n (f, e_k) e_k \right) - \left( \sum_{k=1}^n (f, e_k) e_k, f \right) + \left( \sum_{k=1}^n (f, e_k) e_k, \sum_{k=1}^n (f, e_k) e_k \right)
	\]

	 第二项:
	 \[
	 \left( f, \sum_{k=1}^n (f, e_k) e_k \right) = \sum_{k=1}^n \overline{(f, e_k)} (f, e_k) = \sum_{k=1}^n |(f, e_k)|^2
	 \]
	 
	 第三项:
	 \[
	 \left( \sum_{k=1}^n (f, e_k) e_k, f \right)  = \sum_{k=1}^n (f, e_k) {(f, e_k)} = \sum_{k=1}^n |(f, e_k)|^2
	 \]
		
第四项

\[
\left( \sum_{k=1}^n (f, e_k) e_k, \sum_{k=1}^n (f, e_k) e_k \right)
\]
由于 \(e_k\) 是标准正交基底,因此交叉项消失,只剩下两边下标相同的项

\[
= \sum_{k=1}^n \left( (f, e_k) e_k, (f, e_k) e_k \right) = \sum_{k=1}^n (f, e_k) \overline{(f, e_k)} (e_k, e_k) = \sum_{k=1}^n |(f, e_k)|^2
\]

最终得
\[
\left\| f - \sum_{k=1}^n (f, e_k) e_k \right\|_2^2 = \|f\|_2^2 - \sum_{k=1}^n |(f, e_k)|^2 \geq 0
\]

		\[
		\left( \sum_{k=1}^n (f, e_k) e_k, \sum_{k=1}^n (f, e_k) e_k \right) = \sum_{k=1}^n |(f, e_k)|^2
		\]

	
	综合以上结果,我们得到关键不等式:
	
	\[
	\left\| f - \sum_{k=1}^n (f, e_k) e_k \right\|_2^2 = \|f\|_2^2 - \sum_{k=1}^n |(f, e_k)|^2 \geq 0
	\]
	
	由此直接导出Bessel不等式:
	
	\[
	\sum_{k=1}^n |(f, e_k)|^2 \leq \|f\|_2^2
	\]
	
	当$n \to \infty$时,级数仍然收敛:
	
	\[
	\lim_{n \to \infty} \sum_{k=-n}^n |c_k|^2 < +\infty
	\]
	
\end{proof}
	
	\section{完备赋范空间中级数收敛性的证明}
	
	\begin{proof}
	本部分将证明在完备赋范空间中,傅里叶级数的部分和序列收敛于原函数。基本思路分为三步:
	\begin{enumerate}
		\item 证明部分和序列是Cauchy列
		\item 利用空间完备性得到收敛性
		\item 验证极限函数与原函数相等
	\end{enumerate}
	
	考虑标准正交基$\{e_k\}_{k=-\infty}^\infty$,其中$e_k(x) = \frac{1}{\sqrt{2\pi}}e^{ikx}$(已标准化)。定义部分和:
	

		\[
	S_n(f) = \sum_{k=-n}^n c_k \cdot e_k \quad \quad c_k = (f, e_k)
	\]
	
	\[
	\| S_m(f) - S_n(f) \|_2^2 = \left\| c_{n+1} e_{n+1} + \cdots + c_m e_m + c_{-m} e_{-m} + \cdots + c_{-(n+1)} e_{-(n+1)} \right\|_2^2
	\]
	
	\[
	= \int_{-\pi}^\pi \left| c_{n+1} e_{n+1} + \cdots + c_m e_m + c_{-m} e_{-m} + \cdots + c_{-(n+1)} e_{-(n+1)} \right|^2 dx
	\]
	中间项正交消掉
	
	\[
	= \int_{-\pi}^\pi |c_{n+1} e_{n+1}|^2 dx + \cdots + \int_{-\pi}^\pi |c_m e_m|^2 dx + \cdots + \int_{-\pi}^\pi |c_{-m} e_{-m}|^2 dx
	\]
	
	\[
	\leq \sum_{k=n+1}^m  |c_k|^2 + \sum_{k=-m}^{-(n+1)} |c_k|^2
	\]
	
	由bessel不等式
	\[
 \sum_{k=-n}^n |c_k|^2 \leq \|f\|_2^2 < +\infty \quad f \in L_2
	\]
	
	由柯西准则可知:
	
	\[
	\sum_{k=n+1}^m |c_k|^2 < \frac{\varepsilon}{2}, \quad \sum_{k=-m}^{-(n+1)} |c_k|^2 < \frac{\varepsilon}{2}
	\]
	
	故:
	\[
	\| S_m(f) - S_n(f) \|_2^2 < \varepsilon 
	\]
	
	这表明$\{S_n(f)\}$是Cauchy列。由于$L^2$空间完备,存在$g \in L^2$使得:

\[
\lim_{n \to \infty} S_n(f) = g
\]
	

最后验证$g = f$,对任意基元素$e_j$:
	
	\[
	(g, e_j) = \left( \lim_{n \to \infty} S_n(f), e_j \right)
	\]
	
	由内积的连续性,极限与内积可交换:
	
	\[
	= \lim_{n \to \infty} (S_n(f), e_j) = \lim_{n \to \infty} \left( \sum_{k=-n}^n (f, e_k) e_k, e_j \right) = (f, e_j)
	\]
	
	得:
	
	\[
	(g, e_j) = (f, e_j) \Rightarrow (g - f, e_j) = 0 \Rightarrow g = f
	\]
	
\end{proof}
	
综上所述,故$f$在$L^2$空间下的傅里叶级数依$L^2$范数的均方收敛,则必能按范数收敛。
	


	
\end{document}