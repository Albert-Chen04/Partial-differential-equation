\documentclass[12pt,a4paper]{article}
\usepackage[UTF8]{ctex}
\usepackage{geometry}
\usepackage{amsmath,amssymb,amsthm}
\usepackage{mathtools,bm}
\usepackage{empheq}
\usepackage{graphicx}
\usepackage{booktabs}
\usepackage[numbers,sort&compress]{natbib}
\usepackage{caption}
\usepackage{enumitem}
\usepackage{chngcntr}

% ========== 页面布局 ==========
\geometry{left=2.5cm,right=2.5cm,top=2.5cm,bottom=2.5cm}
\setlength{\parskip}{0.5em}
\renewcommand{\baselinestretch}{1.2}

% ========== 数学命令 ==========
\newcommand{\diff}{\mathop{}\!\mathrm{d}}
\newcommand{\R}{\mathbb{R}}
\newcommand{\C}{\mathbb{C}}
\newcommand{\Z}{\mathbb{Z}}
\newcommand{\N}{\mathbb{N}}
\DeclareMathOperator{\supp}{supp}

% ========== 编号系统 ==========
\numberwithin{subsection}{section}   % 子小节按章节编号
\numberwithin{subsubsection}{subsection}
\counterwithin{equation}{subsection} % 公式按子小节编号

% ========== 定理环境 ==========
\theoremstyle{plain}
\newtheorem{theorem}{定理}[section]
\newtheorem{lemma}[theorem]{引理}
\newtheorem{proposition}[theorem]{命题}
\newtheorem{corollary}[theorem]{推论}
\newtheorem{solution}{解}[subsection]  % 解按子小节编号

\theoremstyle{definition}
\newtheorem{definition}[theorem]{定义}
\newtheorem{example}{示例}[subsection]  % 示例按子小节编号

\theoremstyle{remark}
\newtheorem{remark}[theorem]{注记}

\title{一阶偏微分方程之传输方程}
\author{陈柏均}
\date{2025年5月11日}

\begin{document}
	\maketitle
	
\section{一阶拟传输方程} 
	\subsection{变量替换求解常系数传输方程} 
	\subsubsection{通解} 
	解决方法:通过变量替换,把二元偏微分转化成一元的常微分求解。
	
	假设 $a_1 \neq 0$ 且 $a_2 \neq 0$,我们求解常系数传输方程:
	\begin{equation} \label{eq:pde_original}
		a_1 \frac{\partial u}{\partial t} + a_2 \frac{\partial u}{\partial x} = 0
	\end{equation}
	其中 $u = u(t,x)$。引入坐标变换 $(\alpha, \beta)$,使得 $u = u(\alpha, \beta)$,且:
	\begin{equation} \label{eq:coordinate_transform}
		\begin{cases}
			\alpha = ax + bt, \\
			\beta = cx + dt.
		\end{cases}
	\end{equation}
	特别地,指定 $\beta = a_1x - a_2t$。
	
	利用链式法则计算偏导数:
	\begin{align}
		\frac{\partial u}{\partial t} 
		&= \frac{\partial u}{\partial \alpha} \frac{\partial \alpha}{\partial t} + \frac{\partial u}{\partial \beta} \frac{\partial \beta}{\partial t} 
		= b\frac{\partial u}{\partial \alpha} + d\frac{\partial u}{\partial \beta}, \label{eq:u_t_chain_rule} \\
		\frac{\partial u}{\partial x} 
		&= \frac{\partial u}{\partial \alpha} \frac{\partial \alpha}{\partial x} + \frac{\partial u}{\partial \beta} \frac{\partial \beta}{\partial x} 
		= a\frac{\partial u}{\partial \alpha} + c\frac{\partial u}{\partial \beta}. \label{eq:u_x_chain_rule}
	\end{align}
	
	将 \eqref{eq:u_t_chain_rule} 和 \eqref{eq:u_x_chain_rule} 代入原方程 \eqref{eq:pde_original}:
	\begin{equation} \label{eq:pde_transformed}
		a_1 \left( b \frac{\partial u}{\partial \alpha} + d \frac{\partial u}{\partial \beta} \right) + a_2 \left( a \frac{\partial u}{\partial \alpha} + c \frac{\partial u}{\partial \beta} \right) = 0.
	\end{equation}
	整理后得到:
	\begin{equation} \label{eq:pde_collected}
		(a_1 b + a_2 a) \frac{\partial u}{\partial \alpha} + (a_1 d + a_2 c) \frac{\partial u}{\partial \beta} = 0.
	\end{equation}
为消去一个变量,pde转ode,选择让第二项系数为0,把方程 \eqref{eq:pde_collected} 简化为:
	\begin{equation} \label{eq:pde_final}
		\frac{\partial u}{\partial \alpha} = 0.
	\end{equation}
   选择系数
	\begin{equation} \label{eq:constant_choice}
		\begin{cases}
			a = 0, & b = 1, \\
			c = a_1, & d = -a_2.
		\end{cases}
	\end{equation}
	此时坐标变换为:
	\begin{equation} \label{eq:coordinates_specific}
		\begin{cases}
			\alpha = t, \\
			\beta = a_1 x - a_2 t.
		\end{cases}
	\end{equation}
由\eqref{eq:pde_final} 表明 $u$ 仅依赖于 $\beta$,即通解为:
	\begin{equation} \label{eq:solution}
		u(t,x) = L(a_1 x - a_2 t),
	\end{equation}
	其中
	 $L(\cdot)$ 是任意可微函数。
	\subsubsection{特解(初始条件或边界条件)} 
已知初始条件$	u(x, 0) = e^{-x^2}$,求下面常系数运输方程:
\begin{equation} 
	\frac{\partial u}{\partial t} +  \frac{\partial u}{\partial x} = 0
\end{equation}

由\eqref{eq:solution}可知
\begin{equation}
	u(x, t) = f(x - t)= e^{-(x-t)^2}
\end{equation}

	
		\subsection{特征线法求解变系数传输方程} 
	\subsubsection{通解} 
	一阶线性变系数偏微分方程如下:
	\begin{equation}\label{eq:pde_original2}
		\frac{\partial u}{\partial x} + p(x,y) \frac{\partial u}{\partial y}=(1, p(x, y)) \cdot \left( \frac{\partial u}{\partial x}, \frac{\partial u}{\partial y} \right) = 0 
	\end{equation}
	其中 $p(x, y)$ 是 $x$ 和 $y$ 的函数。
	$\left( \frac{\partial u}{\partial x}, \frac{\partial u}{\partial y} \right)$为梯度,$(1, p(x, y))$为方向,一整个乘积为方向导数,方向导数为0意味着,$u(x, y)=C$在切向量为$(1, p(x, y))$这条曲线上,即
\begin{equation}
	u(x,y)|_{\Gamma} = C
		\end{equation}
	\begin{equation}
		u(x,y) =  f(C)
	\end{equation}
$\Gamma$曲线上,任意点$(x, y)$求导($\Gamma$曲线为$XOY$平面上的曲线,故$y$可表示成$x$的函数),可得切向量$(1,\frac{dy}{dx})$

所以我们找到$\Gamma$曲线,把二元偏微分转化成一元的常微分,令

	\begin{equation}
		\frac{dy}{dx} = p(x, y)
	\end{equation}
	可解得
	\begin{equation}
	C=\phi(x,y)
\end{equation}
得方程解
\begin{equation}
	u(x, y)=f(C)=f(\phi(x,y))
\end{equation}
$(1, \frac{dy}{dx})$
 为该曲线的切向量。我们称这条曲线叫特征线。只需要取遍所有的特征曲线就可以取遍$XOY$平面上所有的点,若有初始条件或者边界条件可以确定每条特征线在$u(x, y)$对应的取值,就可以完整确定$u(x, y)$这个函数。
	
	\begin{example}求解方程
	\begin{equation}
		\frac{\partial u}{\partial x} + x \frac{\partial u}{\partial y} = 0.
	\end{equation}
	
	此时我们有 $p(x, y) = x$,解 $\frac{dy}{dx} = x$,我们得到特征线 $y = \frac{1}{2}x^2 + C$,或 $y - \frac{1}{2}x^2 = C$。从而 $\phi(x, y) = y - \frac{1}{2}x^2$,偏微分方程的通解为 $u(x, y) = f(\phi(x, y))$,其中 $f$ 是任意函数。把它们代回方程,直接验证,便知是解。
		\end{example}

	
\end{document}