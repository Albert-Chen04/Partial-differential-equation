\documentclass[12pt,a4paper]{article}
\usepackage[UTF8]{ctex}                    % 中文支持
\usepackage{geometry}                      % 页面布局
\usepackage{amsmath,amssymb,amsthm}        % 数学符号与定理
\usepackage{mathtools,bm}                  % 高级数学工具
\usepackage{graphicx}                      % 图片插入
\usepackage{hyperref}                      % 超链接
\usepackage{tikz}
\hypersetup{
	pdfborder = {0 0 0}  % 关闭链接边框
}
\usepackage{cleveref}
\usepackage{booktabs}                      % 三线表
\usepackage[numbers,sort&compress]{natbib} % 参考文献
\usepackage{caption}                       % 图表标题
\usepackage[shortlabels]{enumitem}         % 列表环境

% ========== 页面布局 ==========
\geometry{left=2.5cm, right=2.5cm, top=2.5cm, bottom=2.5cm}
\setlength{\parskip}{0.5em}                % 段落间距
\renewcommand{\baselinestretch}{1.2}       % 行距

% ========== 数学命令 ==========
\newcommand{\diff}{\mathop{}\!\mathrm{d}}  % 微分符号
\newcommand{\R}{\mathbb{R}}                % 实数集
\newcommand{\C}{\mathbb{C}}                % 复数集
\newcommand{\Z}{\mathbb{Z}}                % 整数集
\newcommand{\Lspace}{L^2(-l,l)}            % L²空间

% ========== 定理环境 ==========
% 定义题目环境
\newtheorem{problem}{题}
% 定义解题环境
\newtheorem*{solution}{解}

\newtheorem{example}{例题}

\newtheorem{corollary}{推论}

\newtheorem{proposition}{命题}

\newtheorem{lemma}{引理}

% ========== 文档信息 ==========


\vspace{1cm}



%\title{偏微分方程第四次作业}
%\author{22数学1陈柏均202225110102}
%\date{\today}


\begin{document}
	
	\begin{center}
		\LARGE 偏微分方程小测 \\
		\vspace{0.5cm}
		\large 班级:22数学1 \quad 姓名:陈柏均 \quad 学号:202225110102
	\end{center}
	
	%\maketitle
	
	% ========================= 第1题 =========================
	\begin{problem}
		判断下列方程的类型,并化成标准型:
		\[
		u_{xx} + 4u_{xy} - 5u_{yy} + 2u_x + 6u_y = 0.
		\]
	\end{problem}
		\hrulefill
	\begin{solution}
		\textbf{步骤1. 判断方程类型}
		
		\noindent
		方程的系数为 $A=1, B=4, C=-5$。计算判别式:
		\[
		\Delta = B^2 - 4AC = 4^2 - 4(1)(-5) = 16 + 20 = 36 > 0
		\]
		因为 $\Delta > 0$,所以该方程为双曲型方程。
		
		\hrulefill
		
		\textbf{步骤2. 求解特征方程}
		
		\noindent
		特征方程为 $A \left(\frac{dy}{dx}\right)^2 - B \frac{dy}{dx} + C = 0$,即:
		\[
		\left(\frac{dy}{dx}\right)^2 - 4\frac{dy}{dx} - 5 = 0
		\]
		令 $\lambda = \frac{dy}{dx}$,则有 $\lambda^2 - 4\lambda - 5 = 0$,分解因式得 $(\lambda - 5)(\lambda + 1) = 0$。
		解得两个特征方向:
		\[
		\lambda_1 = 5, \quad \lambda_2 = -1
		\]
		对应的特征线方程为:
		\begin{align*}
			\frac{dy}{dx} = 5 &\implies dy - 5dx = 0 \implies y - 5x = C_1 \\
			\frac{dy}{dx} = -1 &\implies dy + dx = 0 \implies y + x = C_2
		\end{align*}
		
		\hrulefill
		
		\textbf{步骤3. 进行坐标变换}
		
		\noindent
		取新的坐标系:
		\[
		\begin{cases}
			\xi = y - 5x \\
			\eta = y + x
		\end{cases}
		\]
		利用链式法则计算各阶偏导数:
		\begin{align*}
			u_x &= u_\xi \xi_x + u_\eta \eta_x = -5u_\xi + u_\eta \\[6pt]
			u_y &= u_\xi \xi_y + u_\eta \eta_y = u_\xi + u_\eta \\[6pt]
			u_{xx} &= \frac{\partial}{\partial x}(-5u_\xi + u_\eta) = -5(u_{\xi\xi}\xi_x + u_{\xi\eta}\eta_x) + (u_{\eta\xi}\xi_x + u_{\eta\eta}\eta_x) \\[6pt]
			&= -5(-5u_{\xi\xi} + u_{\xi\eta}) + (-5u_{\eta\xi} + u_{\eta\eta}) = 25u_{\xi\xi} - 10u_{\xi\eta} + u_{\eta\eta} \\[6pt]
			u_{xy} &= \frac{\partial}{\partial y}(-5u_\xi + u_\eta) = -5(u_{\xi\xi}\xi_y + u_{\xi\eta}\eta_y) + (u_{\eta\xi}\xi_y + u_{\eta\eta}\eta_y) \\[6pt]
			&= -5(u_{\xi\xi} + u_{\xi\eta}) + (u_{\eta\xi} + u_{\eta\eta}) = -5u_{\xi\xi} - 4u_{\xi\eta} + u_{\eta\eta} \\[6pt]
			u_{yy} &= \frac{\partial}{\partial y}(u_\xi + u_\eta) = (u_{\xi\xi}\xi_y + u_{\xi\eta}\eta_y) + (u_{\eta\xi}\xi_y + u_{\eta\eta}\eta_y) \\[6pt]
			&= (u_{\xi\xi} + u_{\xi\eta}) + (u_{\eta\xi} + u_{\eta\eta}) = u_{\xi\xi} + 2u_{\xi\eta} + u_{\eta\eta}
		\end{align*}
		
		\hrulefill
		
		\textbf{步骤4. 代入原方程化简}
		
		\noindent
		将上述偏导数代入原方程:
		\begin{itemize}
			\item \textbf{二阶项}:
			\begin{align*}
				& u_{xx} + 4u_{xy} - 5u_{yy} \\[6pt]
				&= (25u_{\xi\xi} - 10u_{\xi\eta} + u_{\eta\eta}) + 4(-5u_{\xi\xi} - 4u_{\xi\eta} + u_{\eta\eta}) - 5(u_{\xi\xi} + 2u_{\xi\eta} + u_{\eta\eta}) \\[6pt]
				&= (25 - 20 - 5)u_{\xi\xi} + (-10 - 16 - 10)u_{\xi\eta} + (1 + 4 - 5)u_{\eta\eta} \\[6pt]
				&= -36u_{\xi\eta}
			\end{align*}
			\item \textbf{一阶项}:
			\begin{align*}
				& 2u_x + 6u_y \\[6pt]
				&= 2(-5u_\xi + u_\eta) + 6(u_\xi + u_\eta) \\[6pt]
				&= -10u_\xi + 2u_\eta + 6u_\xi + 6u_\eta \\[6pt]
				&= -4u_\xi + 8u_\eta
			\end{align*}
		\end{itemize}
		合并所有项,得到变换后的方程:
		\[
		-36u_{\xi\eta} - 4u_\xi + 8u_\eta = 0
		\]
		两边同除以 $-4$,得到标准型:
		\[
		9u_{\xi\eta} + u_\xi - 2u_\eta = 0
		\]
	\end{solution}
	
	\newpage
	% ========================= 第2题 =========================
	\begin{problem}
		求方程 $u_{xx} + u_{yy} + u_{zz} - u = 0$ 的球对称特解。
	\end{problem}
		\hrulefill
	\begin{solution}
		\textbf{步骤1. 引入球坐标系}
		
		\noindent
	这是对球坐标系下拉普拉斯算子作用于球对称函数 $u=u(r)$ 的详细推导,其中 $r = \sqrt{x^2+y^2+z^2}$。
			
			首先,计算 $r$ 对各变量的一阶和二阶偏导数。
			由 $r = (x^2+y^2+z^2)^{\frac{1}{2}}$ 可得:
			\[
			\frac{\partial r}{\partial x} = \frac{1}{2}(x^2+y^2+z^2)^{-\frac{1}{2}} \cdot 2x = \frac{x}{r}
			\]
			同理,$\frac{\partial r}{\partial y} = \frac{y}{r}$,$\frac{\partial r}{\partial z} = \frac{z}{r}$。
			\[
			\frac{\partial^2 r}{\partial x^2} = - \frac{x^2}{r^3} + \frac{1}{r}
			\]
			利用链式法则,计算 $u(r)$ 的一阶和二阶偏导数:
			\begin{align*}
				u_x &= u_r \frac{\partial r}{\partial x} \\[6pt]
				u_{xx} &= \frac{\partial}{\partial x}\left(u_r \frac{\partial r}{\partial x}\right) = \left(\frac{\partial u_r}{\partial r}\frac{\partial r}{\partial x}\right)\frac{\partial r}{\partial x} + u_r \frac{\partial^2 r}{\partial x^2} = u_{rr}\left(\frac{\partial r}{\partial x}\right)^2 + u_r \frac{\partial^2 r}{\partial x^2}
			\end{align*}
			同理可得 $u_{yy}$ 和 $u_{zz}$ 的表达式。
			
			将三者相加,得到拉普拉斯算子 $\Delta u = u_{xx} + u_{yy} + u_{zz}$:
			\begin{align*}
				\Delta u &= u_{rr} \left[ \left(\frac{\partial r}{\partial x}\right)^2 + \left(\frac{\partial r}{\partial y}\right)^2 + \left(\frac{\partial r}{\partial z}\right)^2 \right] + u_r \left( \frac{\partial^2 r}{\partial x^2} + \frac{\partial^2 r}{\partial y^2} + \frac{\partial^2 r}{\partial z^2} \right) \\[6pt]
				% 计算两个求和项
				% 第一个求和项
				&= u_{rr} \left[ \frac{x^2}{r^2} + \frac{y^2}{r^2} + \frac{z^2}{r^2} \right] + u_r \left( \frac{r^2-x^2}{r^3} + \frac{r^2-y^2}{r^3} + \frac{r^2-z^2}{r^3} \right) \\[6pt]
				% 化简
				&= u_{rr} \left( \frac{x^2+y^2+z^2}{r^2} \right) + u_r \left( \frac{3r^2 - (x^2+y^2+z^2)}{r^3} \right) \\[6pt]
				&= u_{rr} \left( \frac{r^2}{r^2} \right) + u_r \left( \frac{3r^2 - r^2}{r^3} \right) \\
				&= u_{rr} + u_r \left( \frac{2r^2}{r^3} \right) \\[6pt]
				&= u_{rr} + \frac{2}{r} u_r
			\end{align*}
			
		
		
		球对称解意味着解 $u$ 只与到原点的距离 $r = \sqrt{x^2+y^2+z^2}$ 有关,即 $u=u(r)$。
		在球坐标系下,拉普拉斯算子 $\Delta = \frac{\partial^2}{\partial x^2} + \frac{\partial^2}{\partial y^2} + \frac{\partial^2}{\partial z^2}$ 作用于球对称函数 $u(r)$ 的形式为:
		\[
		\Delta u = u_{rr} + \frac{2}{r} u_r
		\]
		于是原方程变为一个关于 $r$ 的常微分方程:
		\[
		u_{rr} + \frac{2}{r} u_r - u = 0
		\]
		
		\hrulefill
		
		\textbf{步骤2. 使用变量代换简化方程}
		
		\noindent
		为了求解这个方程,作变量代换,令 $v(r) = r u(r)$。则 $u(r) = \frac{v(r)}{r}$。
		计算 $u$ 对 $r$ 的导数:
		\begin{align*}
			u_r &= \frac{v_r r - v}{r^2} = \frac{v_r}{r} - \frac{v}{r^2} \\[6pt]
			u_{rr} &= \frac{v_{rr} r - v_r}{r^2} - \frac{v_r r^2 - v (2r)}{r^4} = \frac{v_{rr}}{r} - \frac{v_r}{r^2} - \frac{v_r}{r^2} + \frac{2v}{r^3} \\[6pt]
			&= \frac{v_{rr}}{r} - \frac{2v_r}{r^2} + \frac{2v}{r^3}
		\end{align*}
		将 $u, u_r, u_{rr}$ 代入原方程:
		\[
		\left( \frac{v_{rr}}{r} - \frac{2v_r}{r^2} + \frac{2v}{r^3} \right) + \frac{2}{r} \left( \frac{v_r}{r} - \frac{v}{r^2} \right) - \frac{v}{r} = 0
		\]
		化简得:
		\[
		\frac{v_{rr}}{r} - \frac{2v_r}{r^2} + \frac{2v}{r^3} + \frac{2v_r}{r^2} - \frac{2v}{r^3} - \frac{v}{r} = 0
		\]
		\[
		\frac{v_{rr}}{r} - \frac{v}{r} = 0
		\]
		在 $r \neq 0$ 的情况下,方程简化为:
		\[
		v_{rr} - v = 0
		\]
		也可用
		\begin{equation*}
			\frac{\partial^2}{\partial r^2}(ru) = \frac{\partial}{\partial r}(u + ru_r) = u_r + u_r + r u_{rr} = r u_{rr} + 2u_r
		\end{equation*}

		
		
		\hrulefill
		
		\textbf{步骤3. 求解并回代}
		
		\noindent
		这是一个常系数线性常微分方程,其通解为:
		\[
		v(r) = C_1 e^r + C_2 e^{-r}
		\]
		或者写成
			\[
		v(r) = C_1 \cosh r + C_2 \sinh r
			\]
		其中 $C_1, C_2$ 是任意常数。
		将 $u(r) = v(r)/r$ 代回,得到原方程的球对称特解:
		\[
		u(r) = \frac{C_1 e^r + C_2 e^{-r}}{r}
		\]
		或者写成
		\[
			u(r) = \frac{C_1 \cosh r + C_2 \sinh r}{r}
		\]
	\end{solution}
	
	\newpage
	% ========================= 第3题 =========================
	\begin{problem}
		求解 Cauchy 问题:
		\[
		\begin{cases}
			u_t - u_{xx} - 4tu = 0, & -\infty < x < \infty, \ t > 0, \\[6pt]
			u|_{t=0} = xe^x, & -\infty < x < \infty.
		\end{cases}
		\]
	\end{problem}
		\hrulefill
	\begin{solution}
		\textbf{步骤1. 消除含 $u$ 的项}
		
		\noindent
设变量代换为:
\[
u(x,t) = F(t) V(x,t)
\]

计算其偏导数:
\begin{align*}
	u_t &= F'(t) V(x,t) + F(t) V_t(x,t) \\[6pt]
	u_{xx} &= F(t) V_{xx}(x,t)
\end{align*}

代入原方程 $u_t - u_{xx} - 4tu = 0$ 并整理:
\begin{align*}
	& F'(t)V + F(t)V_t - F(t)V_{xx} - 4t F(t)V = 0 \\[6pt]
	\implies \quad & F(t) \left[ V_t - V_{xx} \right] + \left[ F'(t) - 4t F(t) \right] V = 0
\end{align*}

为使方程简化,我们令 $V(x,t)$ 的系数项为零:
\[
F'(t) - 4t F(t) = 0
\]
这样,原方程就变为标准的热传导方程:
\[
V_t - V_{xx} = 0
\]

现在求解关于 $F(t)$ 的常微分方程:
\begin{align*}
	F'(t) &= 4t F(t) \\[6pt]
	\frac{dF}{F} &= 4t \, dt \\[6pt]
	\ln|F| &= 2t^2 + C \\[6pt]
	F(t) &= C_1 e^{2t^2}
\end{align*}

为方便计算,取 $C_1=1$,得到:
\[
F(t) = e^{2t^2}
\]
此时,初始时刻有 $F(0)=1$。

最后,变换初始条件 $u(x,0)=xe^x$:
\begin{align*}
	F(0) V(x,0) &= xe^x \\
	\implies \quad V(x,0) &= xe^x
\end{align*}

		\hrulefill
		
		\textbf{步骤2. 求解新的 Cauchy 问题}
		
		\noindent
		经过代换,原问题转化为关于 $V(x,t)$ 的标准热传导方程的 Cauchy 问题:
		\[
		\begin{cases}
			V_t - V_{xx} = 0, & -\infty < x < \infty, \ t > 0, \\[6pt]
			V(x,0) = xe^x, & -\infty < x < \infty.
		\end{cases}
		\]
		该问题的解由热核(Poisson公式)给出:
			\[
		V(x,t)= (4 \pi a t)^{-\frac{n}{2}} \int_{-\infty}^{+\infty} e^{-\frac{|x-y|^2}{4at}} \cdot V(y,0) \, dy
		\]
		其中$a=1$,$n=1$($a$为方程系数,$n$为空间变量$x$的维度)
		\[
		V(x,t) = \frac{1}{\sqrt{4\pi t}} \int_{-\infty}^{\infty} V(y,0) e^{-\frac{(x-y)^2}{4t}} dy = \frac{1}{\sqrt{4\pi t}} \int_{-\infty}^{\infty} y\cdot e^y e^{-\frac{(x-y)^2}{4t}} dy
		\]
		
		\hrulefill
		
		\textbf{步骤3. 计算积分}
		
		\noindent
		我们处理指数部分的项,进行配方:
		\begin{align*}
			y - \frac{(x-y)^2}{4t} &= \frac{4ty - (x^2 - 2xy + y^2)}{4t} = -\frac{1}{4t} [y^2 - (2x+4t)y + x^2] \\[6pt]
			&= -\frac{1}{4t} \left[ \left(y - (x+2t)\right)^2 - (x+2t)^2 + x^2 \right] \\
			&= -\frac{(y - (x+2t))^2}{4t} + \frac{(x+2t)^2 - x^2}{4t} \\[6pt]
			&= -\frac{(y - (x+2t))^2}{4t} + \frac{x^2 + 4xt + 4t^2 - x^2}{4t} \\
			&= -\frac{(y - (x+2t))^2}{4t} + x + t
		\end{align*}
		将此结果代入积分表达式:
		\[
		V(x,t) = \frac{1}{\sqrt{4\pi t}} \int_{-\infty}^{\infty} y e^{-\frac{(y - (x+2t))^2}{4t}} e^{x+t} dy = \frac{e^{x+t}}{\sqrt{4\pi t}} \int_{-\infty}^{\infty} y e^{-\frac{(y - (x+2t))^2}{4t}} dy
		\]
		作变量代换,令 $z = y - (x+2t)$,则 $y = z + x + 2t$,$dy = dz$。
		\begin{align*}
			\int_{-\infty}^{\infty} y e^{-\frac{(y - (x+2t))^2}{4t}} dy &= \int_{-\infty}^{\infty} (z + x + 2t) e^{-\frac{z^2}{4t}} dz \\[6pt]
			&= \int_{-\infty}^{\infty} z e^{-\frac{z^2}{4t}} dz + (x+2t) \int_{-\infty}^{\infty} e^{-\frac{z^2}{4t}} dz
		\end{align*}
		第一项的被积函数是奇函数,积分为 $0$。第二项是高斯积分,$\int_{-\infty}^{\infty} e^{-az^2}dz = \sqrt{\frac{\pi}{a}}$。这里 $a = \frac{1}{4t}$。
		\[
		(x+2t) \int_{-\infty}^{\infty} e^{-\frac{z^2}{4t}} dz = (x+2t)\sqrt{4\pi t}
		\]
		因此,
		\[
		V(x,t) = \frac{e^{x+t}}{\sqrt{4\pi t}} \left[ 0 + (x+2t)\sqrt{4\pi t} \right] = (x+2t)e^{x+t}
		\]
		
		\hrulefill
		
		\textbf{步骤4. 回代得到最终解}
		
		\noindent
		将 $V(x,t)$ 的解代回 $u(x,t) = V(x,t) e^{2t^2}$:
		\[
		u(x,t) = (x+2t)e^{x+t} e^{2t^2} = (x+2t)e^{x+t+2t^2}
		\]
	\end{solution}
	
	\newpage
	% ========================= 第4题 =========================
	\begin{problem}
		用分离变量法求解初边值问题:
		\[
		\begin{cases}
			u_{tt} - u_{xx} = 0, & 0 < x < 2, \ t > 0, \\[6pt]
			u|_{t=0} = 2\sin(\pi x), \quad u_t|_{t=0} = 0, & 0 \leq x \leq 2, \\[6pt]
			u|_{x=0} = 0, \quad u|_{x=2} = 0, & t \geq 0.
		\end{cases}
		\]
	\end{problem}
		\hrulefill
	\begin{solution}
	\textbf{步骤1. 分离变量}
	
设解的形式为 $u(x,t) = X(x)T(t)$。
\[
u_{tt} = X T'', \quad u_{xx} = X'' T
\]
代入方程 $u_{tt} - u_{xx} = 0$ 并分离变量,设分离常数为 $k$:
\[
\frac{T''}{T} = \frac{X''}{X} = k
\]
由此得到两个常微分方程:
\[
\begin{cases}
	X'' - kX = 0 \\
	T'' - kT = 0
\end{cases}
\]

\hrulefill

\textbf{步骤2. 求解空间方程}

空间方程及其边界条件为:
\[
X'' - kX = 0, \quad X(0) = 0, \quad X(2) = 0
\]
分三种情况讨论 $k$:

\begin{itemize}
	\item \textbf{情况一:$k > 0$}.
	设 $k = \mu^2$ ($\mu>0$),通解为 $X(x) = c_1 \cosh(\mu x) + c_2 \sinh(\mu x)$。
	由 $X(0)=0$ 得 $c_1=0$。由 $X(2)=0$ 得 $c_2 \sinh(2\mu) = 0$,因 $\sinh(2\mu) \neq 0$,故 $c_2=0$。只有平凡解,舍去。
	
	\item \textbf{情况二:$k = 0$}.
	方程为 $X''=0$,通解为 $X(x) = c_1 x + c_2$。
	由 $X(0)=0$ 得 $c_2=0$。由 $X(2)=0$ 得 $2c_1=0$,故 $c_1=0$。只有平凡解,舍去。
	
	\item \textbf{情况三:$k < 0$}.
	设 $k = -\mu^2$ ($\mu>0$),方程为 $X'' + \mu^2 X = 0$,通解为 $X(x) = c_1 \cos(\mu x) + c_2 \sin(\mu x)$。
	由 $X(0)=0$ 得 $c_1=0$。
	由 $X(2)=0$ 得 $c_2 \sin(2\mu) = 0$。为得到非平凡解,须 $\sin(2\mu)=0$。
	因此 $2\mu = n\pi$,即 $\mu_n = \frac{n\pi}{2}$ ($n=1, 2, 3, \dots$)。
\end{itemize}
由此得到本征值 $k_n = -\mu_n^2 = -\left(\frac{n\pi}{2}\right)^2$。
对应的本征函数为 $X_n(x) = \sin\left(\frac{n\pi x}{2}\right)$。

\hrulefill

\textbf{步骤3. 求解时间方程}

时间方程为 $T_n'' - k_n T_n = 0$。代入 $k_n$:
\[
T_n'' - \left(-\left(\frac{n\pi}{2}\right)^2\right) T_n = 0 \implies T_n'' + \left(\frac{n\pi}{2}\right)^2 T_n = 0
\]
其通解为:
\[
T_n(t) = a_n \cos\left(\frac{n\pi t}{2}\right) + b_n \sin\left(\frac{n\pi t}{2}\right)
\]

\hrulefill

\textbf{步骤4. 叠加并利用初始条件}

根据叠加原理,解可以写成级数形式:
\[
u(x,t) = \sum_{n=1}^\infty \left[ a_n \cos\left(\frac{n\pi t}{2}\right) + b_n \sin\left(\frac{n\pi t}{2}\right) \right] \sin\left(\frac{n\pi x}{2}\right)
\]
利用初始条件 $u(x,0) = 2\sin(\pi x)$:
\[
u(x,0) = \sum_{n=1}^\infty a_n \sin\left(\frac{n\pi x}{2}\right) = 2\sin(\pi x)
\]
比较系数可知,当 $n=2$ 时,$a_2=2$。当 $n \neq 2$ 时,$a_n=0$。

对时间求导:
\[
u_t(x,t) = \sum_{n=1}^\infty \left[ -a_n \frac{n\pi}{2} \sin\left(\frac{n\pi t}{2}\right) + b_n \frac{n\pi}{2} \cos\left(\frac{n\pi t}{2}\right) \right] \sin\left(\frac{n\pi x}{2}\right)
\]
利用初始条件 $u_t(x,0) = 0$:
\[
u_t(x,0) = \sum_{n=1}^\infty b_n \frac{n\pi}{2} \sin\left(\frac{n\pi x}{2}\right) = 0
\]
因此,所有 $b_n=0$。

\hrulefill

\textbf{步骤5. 最终解}

将求得的系数代回级数,只有 $n=2$ 的项被保留:
\[
u(x,t) = a_2 \cos\left(\frac{2\pi t}{2}\right) \sin\left(\frac{2\pi x}{2}\right)
\]
化简得到最终解:
\[
u(x,t) = 2\cos(\pi t)\sin(\pi x)
\]
	\end{solution}
	
\end{document}